\textbf{¿Qué ventajas y desventajas encuentras al trabajar con un Sistema de Bases de Datos
considerando que se planea implantar este sistema en una empresa de telemarketing?} \\

En primera instancia tenemos que reconocer que el telemarketing es una técnica publicitaria que es utilizada por las empresas para contactar con potenciales clientes y hablarles acerca de sus productos o servicios . \\

Hay muchas empresas a lo largo del mundo que hacen de esta estrategia su principal manera de operar, catalogándose en mayor o menor medida como empresas de telemarketing y si bien cada empresa decide como gestionar sus recursos para así poder llegar a su publico objetivo, el telemarketing cuenta con una serie de ventajas y desventajas enumeradas de la siguiente manera las cuales pueden hacer que una empresa se decante por este sistema o no: \\


\begin{table}[h!]
    \centering
    \begin{tabular}{|p{6cm}||p{6cm}|}
        \hline
        \textcolor{blue}{\textbf{Ventajas}} & \textcolor{Red}{\textbf{Desventajas}} \\ \hline 
        Se tiene un trato mas directo con los potenciales clientes & Si se utiliza de manera inadecuada puede afectar negativamente a la reputación de la empresa \\ \hline
        Permite entrar en contacto con un gran número de clientes en poco tiempo & En algunos países existen ciertas restricciones para esta clase de técnicas \\ \hline 
        Permite ofrecer una gran cantidad de información sobre el producto o servicio en cuestión  & Requiere una inversión en formar a los agentes y que estos sigan las buenas practicas de la empresa \\ \hline
        Se puede llamar a cualquier parte del mundo, lo cual asegura que nuestra empresa pueda darse a conocer en otras fronteras & La tasa de conversión es baja\\ \hline
    \end{tabular}
    \caption{Ventajas y desventajas de un Sistema de Telemarketing \cite{boada_cyberclick_2023}} 
\end{table}

Así mismo el telemarketing puede utilizarse de distintas maneras y estas varían de acuerdo a la campaña y objetivos de la empresa. Pero volviendo al punto principal de la pregunta, algunas de las posibles ventajas y desventajas de implemmentar un sistema de base de datos en alguna empresa de telemarketing podrian ser las siguientes:\\


\begin{table}[h!]
    \centering
    \begin{tabular}{|p{6cm}||p{6cm}|}
        \hline
        \textcolor{green}{\textbf{Ventajas}} & \textcolor{purple}{\textbf{Desventajas}} \\ \hline
        Organización y gestión eficiente de datos & Altos costos de implementación y mantenimiento continuo. \\ \hline
        Generación de análisis detallados e historial de datos. & Trabajo extra para integración con sistemas existentes y necesidad de capacitación del personal. \\ \hline
         Automatiza las tareas repetitivas y hay una reducción de errores manuales. & Riesgo de interrupciones por fallos del sistema y necesidad de ajustes por actualizaciones. \\ \hline
        Tiene la capacidad para manejar más datos y mas personalización según necesidades específicas. &  Riesgos de seguridad y necesidad de cumplir con normativas de privacidad. \\ \hline
        Existe un control de acceso a datos sensibles y así como también un respaldo de datos. & Existen riesgos por parte de entradas de datos incorrectos y problemas de duplicados. \\ \hline
    \end{tabular}
    \caption{Ventajas y desventajas de un Sistema de BD en una empresa de telemarketing}
    \cite{adSalsa_2024}
\end{table}
