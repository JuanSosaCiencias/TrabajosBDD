\begin{center}   
\textbf{Indica las responsabilidades que tiene un Sistema Manejador de Bases de Datos y para cada responsabilidad, explica los problemas que surgirían si dicha responsabilidad no se cumpliera.}\\
\end{center}

\begin{enumerate}
    \item[1.] \textbf{Almacenamiento y Recuperación de Datos:}
    El SGBD organiza y almacena grandes volúmenes de datos, permitiendo una recuperación eficiente de la información.\\
    \textbf{Problema:} Si no se cumple esta responsabilidad, la empresa podría enfrentar dificultades para acceder a datos vitales, lo que ralentizaría operaciones diarias y la toma de decisiones importantes. \cite{Uni-Europea} \\
             
    \item[2.] \textbf{Control de Concurrencia:}
     Gestionar accesos simultáneos a la base de datos, evitando conflictos y asegurando que múltiples usuarios puedan acceder a la información sin comprometer su integridad.\\
     \textbf{Problema:} Si no se controla el acceso concurrente, podrían generarse inconsistencias en los datos cuando varios usuarios intentan modificar la misma información al mismo tiempo. \\

    \item[3.] \textbf{Seguridad:}
    Restringir el acceso a los datos, garantizando que solo usuarios autorizados puedan acceder o modificar la base de datos.\\
    \textbf{Problema:} Sin una adecuada seguridad, los datos sensibles podrían verse comprometidos, lo que afectaría la privacidad de la información y podría resultar en violaciones legales y pérdida de confianza. \\

    \item[4.] \textbf{Integridad de los datos:}
    Asegurar que los datos almacenados sigan reglas específicas, manteniéndose correctos y fiables a lo largo del tiempo.\\
    \textbf{Problema:} Si la integridad no está garantizada, la base de datos podría llenarse de datos erróneos o inconsistentes, afectando los análisis y decisiones basadas en esa información. \\

    \item[5.] \textbf{Copia de Seguridad y Recuperación:}
    Realizar backups periódicos y estar preparado para restaurar los datos en caso de fallo del sistema.\\
    \textbf{Problema:} Sin copias de seguridad, un fallo del sistema podría resultar en la pérdida irreparable de datos, impactando gravemente las operaciones comerciales. \\

    \item[6.] \textbf{Optimización del Rendimiento:}
    Monitorear y ajustar el rendimiento del sistema para que las consultas y transacciones sean rápidas y eficientes.\\
    \textbf{Problema:} Un rendimiento deficiente llevaría a tiempos de respuesta lentos, lo que afectaría la productividad general y la experiencia del usuario. 
    
    \textcolor{blue}{Toda la información de esta sección se saco del texto proporcionado\cite{Emagister}} \\
    
\end{enumerate}