\begin{center}
    \textbf{Para cada política que investigaron, ¿cuáles son sus ventajas y desventajas\end{center}
\begin{itemize}

    \item \textbf{CASCADE}:
    
    \textbf{Ventaja}: Facilita la gestión automática de datos relacionados, eliminando o actualizando los registros dependientes de manera automática, lo que reduce el trabajo manual y asegura la coherencia entre las tablas.
    
    \textbf{Desventaja}: Es peligrosa, ya que una eliminación en cascada podría borrar muchos registros relacionados de forma accidental, lo que podría resultar en la pérdida de diversos datos importantes.
    
    \vspace{0.5cm}
    
    \item \textbf{SET NULL}:
    
    \textbf{Ventaja}: Permite mantener los registros relacionados, pero indicando que la relación con la tabla principal ya no es válida, asignando un valor \texttt{NULL}. Es útil cuando se desea conservar los datos secundarios sin una relación activa.
    
    \textbf{Desventaja}: Puede generar inconsistencias si se permite que muchos campos se queden en \texttt{NULL}, lo que complica analizar los datos. Además, no es aplicable si la columna no permite valores \texttt{NULL}.
    
    \vspace{0.5cm}
    
    \item \textbf{SET DEFAULT}:
    
    \textbf{Ventaja}: Asigna un valor predeterminado en los registros relacionados cuando se elimina o actualiza un registro en la tabla principal, lo que puede ser útil para mantener datos válidos en los registros secundarios.
    
    \textbf{Desventaja}: Podría introducir datos incorrectos o inconsistentes si el valor predeterminado no refleja adecuadamentevlos datos. Esto podría generar confusión o errores en los reportes.
    
    \vspace{0.5cm}
    
    \item \textbf{RESTRICT}:
    
    \textbf{Ventaja}: Protege contra la eliminación o actualización accidental de registros importantes, asegurando que no se puedan eliminar registros si existen dependencias en otras tablas. Esto previene la creación de registros huérfanos.
    
    \textbf{Desventaja}: Puede ser demasiado restrictiva en algunos casos, bloqueando operaciones que podrían ser necesarias, lo que podría generar errores al intentar gestionar los datos.
    
    \vspace{0.5cm}
    
    \item \textbf{NO ACTION}:
    
    \textbf{Ventaja}: Similar a \texttt{RESTRICT}, impide la eliminación o actualización de registros relacionados, pero la validación ocurre al final de la transacción, lo que podría ser útil en transacciones complejas.
    
    \textbf{Desventaja}: Al validar la integridad referencial al final de la transacción, pueden generarse errores que sean difíciles de manejar, especialmente si ocurren al final de un proceso complejo.

    \vspace{0.5cm}
    
\end{itemize}
