\begin{center}
    \textbf{¿Qué es una política de mantenimiento de llaves foráneas?}   
\end{center}

Bien sabemos que en PostgreSQL, una \textbf{llave foránea} es una restricción que se utiliza para garantizar la integridad entre dos tablas. Entonces, en pocas palabras, en una tabla se referencia el campo clave primaria en otra tabla. \\

Son reglas que definen qué sucede con los registros relacionados entre dos tablas, a través de llaves foráneas, cuando se hacen cambios en la tabla que contiene la clave primaria. \\

Nos sirven para que exista coherencia entre los datos de ambas tablas cuando se eliminan o actualizan registros. \\

Las principales políticas son:

\begin{itemize}
    \item \textbf{ON DELETE}: Esta política especifica qué sucede cuando se elimina un registro de la tabla referenciada (la que contiene la clave primaria).
    \item \textbf{ON UPDATE}: Esta política especifica qué sucede cuando se actualiza un valor de clave primaria en la tabla referenciada.
\end{itemize}

PostgreSQL define diferentes políticas que pueden configurarse cuando
se crean las restricciones de llaves foráneas, las cuales determinan
cómo se gestionan los cambios en los registros referenciados. \\

Las opciones disponibles para ambas políticas son:

\begin{itemize}
    \item \textbf{CASCADE}: Si se elimina o actualiza un registro en la tabla referenciada, se eliminarán o actualizarán automáticamente los registros correspondientes en la tabla que contiene la llave foránea.
    \item \textbf{SET NULL}: Si se elimina o actualiza un registro en la tabla referenciada, los campos de la llave foránea en la tabla dependiente se establecerán en \texttt{NULL}.
    \item \textbf{SET DEFAULT}: Si se elimina o actualiza un registro en la tabla referenciada, los campos de la llave foránea en la tabla dependiente se establecerán en un valor por defecto predefinido.
    \item \textbf{RESTRICT}: Impide la eliminación o actualización del registro referenciado si hay registros dependientes en la tabla que contiene la llave foránea.
    \item \textbf{NO ACTION}: Similar a RESTRICT, pero la validación ocurre al final de la transacción. PostgreSQL no permite que se complete la transacción si la llave foránea se ve comprometida.
\end{itemize}

Estas políticas permiten mantener la consistencia e integridad de los datos en una base de datos relacional, asegurando que no haya referencias huérfanas o inconsistentes entre las tablas. \\
