\begin{center}
\textbf{Investiga tres noticias relacionadas con las hojas de cálculo, en donde se muestre que hay errores en el procesamiento/manejo/distribución de datos y realiza una discusión breve de cada una de ellas, contrastándolas con el enfoque de las bases de datos. ¿Por qué consideras qué a pesar de los ejemplos mostrados, las hojas de cálculo siguen siendo la herramienta más prevaleciente en un ambiente de negocios típico?} \\
\end{center}

La primera noticia que tomamos como referencia nos habla acerca de la perdida financiera de 86 millones de euros por parte del \textit{Norges Bank} de Noruega, debido a una falla en una hoja de cálculo, que de acuerdo al medio que cubrió la noticia todo fue debido a un error humano en el cual se vio involucrado un empleado que según sus propias palabras \textit{"Fue un error manual. Utilicé la fecha equivocada, el 1 de diciembre en lugar del 1 de noviembre, como estaba claramente marcado"}. Por lo cual como podemos apreciar se trato de una casilla mal colocada en una hoja de Excel y debido a esto, se registro un impacto de 0.7 puntos en el fondo soberano noruego lo cual provoco la perdida de 900 millones de coronas o 86 millones de euros como se menciono al principio. 
\textcolor{blue}{Toda la información de esta sección se ha tomado del texto proporcionado\cite{dinero_en_imagen_2024}}
\\

Ahora pasando a la segunda noticia, esta nos habla sobre una situación que aconteció en la pandemia, mas en especifico en reino unido. Según el medio del cual es sacada esta noticia, las autoridades sanitarias británicas anunciaron que casi 16,000 casos de coronavirus no se informaron durante una semana. De acuerdo con \textit{Public Health England (PHE)}, que contabilizaba diariamente las nuevas infecciones por coronavirus en Excel, se omitieron 15,841 casos nuevos entre el 25 de septiembre y el 2 de octubre del 2020. Este fallo se debió a que la hoja de calculo de Excel había alcanzado su limite. \\

Esto ocasiono que el numero de personas infectadas de los casos nuevos diarios fueran significativamente más altos de lo que parecían, lo que daba como especulación que propagación de la infección en Inglaterra pareciese ser marcadamente mayor de lo que se informó. Lo cual afecto de manera mediática en como fue percibida la propagación del virus en reino unido, todo debido a un fallo en el software que se utilizaba para almacenar y consultar los datos.\textcolor{blue} \\
{Toda la información de esta sección se ha tomado del texto proporcionado\cite{deyden-2020}}
\\

Pasando ahora con la tercera noticia que se solicito, esta nos habla acerca de la necesidad de migrar del sistema convencional en el cual se utiliza Excel para las cuestiones relacionadas al almacenamiento y consulta de información en el ámbito del \textit{Big Data} ya que las hojas de calculo ya no son suficientes para las necesidades y requerimientos de las empresas en dicho ámbito. Por ejemplo, se sabe que hasta 2018, el volumen de información en el mundo se calculaba en 33 zettabytes. Al cierre de 2020, casi se duplicó al llegar a 59 zettabytes y esta cifra sigue creciendo conforme pasan los años, por lo cual se necesitan nuevos sistemas que soporten una mayor cantidad de datos como lo son las bases de datos. \\

Es por eso que el volumen casi incontrolable de información que se va generando segundo a segundo ha provocado que las hojas de calculo cada vez se queden más y más atrás.  \cite{escobar-2023} \\

Ahora bien, si contrastamos las 3 noticias anteriores con el enfoque de las bases de datos, se pueden observar en general algunas diferencias en términos de robustez, escalabilidad y manejo de errores vemos por ejemplo que en la primera noticia, por ejemplo; en la primera noticia, el error mas que nada humano de ingresar una fecha incorrecta podría haberse evitado mediante el uso de una base de datos con reglas de validación más estrictas y control de versiones, pudiendo así evitar el ingreso de datos erróneos.\\

En la segunda noticia, la limitación de Excel al alcanzar su capacidad máxima se podría haberse evitado utilizando una base de datos relacional o no relacional que puede manejar millones de registros sin problemas de rendimiento. Por ultimo en la tercera noticia, se discute la necesidad de migrar de hojas de cálculo a sistemas más avanzados para el manejo de Big Data y como ya lo mencionábamos las bases de datos están diseñadas para gestionar volúmenes masivos de información, permitiendo no solo su almacenamiento, sino también su procesamiento y análisis en tiempo real. \cite{kde-2023}\\

Ahora, respondiendo a la pregunta. Nosotros pensamos que las hojas de calculo siguen siendo la herramienta mas prevaleciente en un ambiente de negocios típico porque la infraestructura esta principalmente diseñada para tal, así como el personal de dichas instituciones pudiera estar mas familiarizado con el uso de Excel que con el manejo de una base de datos, todo esto aunado a que a lo largo de los años, Excel de la mano de Microsoft ha hecho que la mayoría de las empresas a nivel mundial utilicen su producto generando de esta manera un \textit{estatus quo} en las empresas y el ambiente de las mismas en general. \\

En conclusión, podemos decir que las hojas de calculo son mas utilizadas que las bases de datos debido a su popularidad y fácil acceso comparado con una base de datos tradicional, aunado a la infraestructura de las empresas mismas y el personal encargado de manejar los datos dentro de las organizaciones.\\
