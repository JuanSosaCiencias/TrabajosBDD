\begin{center}   
\textbf{Investiga cuáles son las responsabilidades de un DBA. Si asumimos que el DBA nunca está
interesado en ejecutar sus propias consultas, ¿necesita entender y/o conocer el modelo de datos
lógico de la base de datos? Justifica tu respuesta} \\
\end{center}

Las responsabilidades de un DBA, consiste en gestionar el software de las bases de datos, determina la organización y la almacena digitalmente, en la cual verifica la integridad de los datos y responsabilizándose de su seguridad.


\begin{table}[h]
\centering
\begin{tabular}{|p{6cm}|p{6cm}|}
\hline
\textbf{Tareas de BD} & \textbf{Permisos a los usuarios} \\
\hline
Se encarga del diseño, desarrollo y mantenimiento de las BD, de los cambios, de comprobar el funcionamiento correcto y de la eficacia del acceso de las bases de datos & Permite a los usuarios guardar, ordenar, extraer los datos y compartirlos a través de una red interna o por internet \\
\hline
\end{tabular}
\caption{Descripción de tareas y permisos}
\end{table}

Respondiendo a la pregunta de \textquotedblleft Si asumimos que el DBA
nunca está interesado en ejecutar sus propias consultas, ¿necesita entender y/o conocer el modelo de datos lógico de la base de datos?\textquotedblright
\\
\textcolor{blue}{Tod ala información de esta sección se sacó del trexto proporcionado\cite{Barcelona-Activa}} \\


Bueno, no necesariamente. Si el DBA (Administrador de Base de Datos) no está interesado en ejecutar sus propias consultas; entonces su principal enfoque estaría en la administración, seguridad, rendimiento y disponibilidad de la base de datos. Sin embargo el conocer el modelo de datos lógico puede ser útil para entender mejor cómo están organizados los datos, cómo se relacionan entre sí y cómo optimizar el rendimiento del sistema. Claramente, esto le ayudaría a tomar decisiones más informadas sobre cómo administrar la base de datos de manera eficiente. Pero bien sabemos que podría realizar su trabajo sin dicha actividad, sin embargo como se dijo anteriormente, es un plus que se tiene para realizar mejor la Base de Datos. \\
