\begin{center}
\textbf{Supón que deseas crear una red social orientada al uso empresarial, negocios y empleo, similar
a LinkedIn. Considera cada una de las desventajas indicadas en el documento “Purpose of Database Systems”, cuando se administran los datos en un sistema de archivos. Discute la relevancia de cada uno de los puntos indicados, con respecto al almacenamiento de datos de los perfiles profesionales: el usuario que lo subió, las fechas de los empleos que ha tenido, sus habilidades, sus postulaciones y los usuarios que vieron su perfil, cantidad de publicaciones realizadas, cantidad de reacciones, cantidad de compartidos, entre otros.}\\
\end{center}

En el documento nos explican como, para hacer nuestra base de información una universidad utilizo una serie de archivos y para modificar y manipular esta información se tienen que tener programas que hagan ciertas funciones como añadir estudiantes, materias o registrar alumnos, esto supone un problema a la hora de querer hacer cambios grandes o añadir mas información pues por ejemplo si necesitamos crear un curso nuevo o cambiar las reglas de la escuela habrá que cambiar varios archivos de programas y sera difícil de automatizar.\\

En nuestro caso suponemos que queremos crear un pseudo Linkedin; es claro, que en este caso, va a haber bastante mas manipulación de datos que en el caso ejemplo entonces para empezar, tendremos este problema de tener que reescribir programas cada que queramos cambiar cosas.\\

Ahora vamos a repasar las otras desventajas para ver como afectan a nuestro caso:
\begin{itemize}
    \item \textbf{Inconsistencia y redundancia de datos:} El texto explica que debido a que los programas probablemente se escriba entre varias personas a lo largo del tiempo quizás en diferentes lenguajes y con diferentes estructuras, es altamente probable que los archivos tengan grandes inconsistencias entre datos guardados y contengan varias veces la misma información, en nuestro ejemplo, por ejemplo al tener un programador que crea su archivo para registrar perfiles profesionales y otro que escribe por ejemplo el programa para agregar una persona en un perfil de empresa, es muy probable que ambos repitan el proceso de guardar muchos de los mismos datos y especialmente si no están coordinados van a guardar 2 veces el mismo perfil; aun así, el problema no acaba ahí, pues sabemos que la información al no ser estática, probablemente sera modificada y quizás por un tercer programador que si no trata con cuidado la información va a pasar que empiece a generar inconsistencias dentro del mismo archivo o a lo largo de varios archivos.\\
    \item \textbf{Dificultad para acceder a la información:} Este titulo se explica solo, pero, por ejemplo si una empresa quiere buscar a programadores que sepan utilizar lenguaje de consulta SQL, como en realidad no existe una manera simple de consultar esto, podemos o obtener todos los programadores de todo (si es que existe tal lista) que consumiría muchos recursos y seria luego tedioso filtrar, o alternativamente podríamos poner a alguien a hacer un programa para conseguir esta lista pero obviamente esto también tardaría y probablemente costaría mas.\\
    \item \textbf{Datos aislados:} Como ya dijimos los archivos de texto tienen limites informáticos mucho mas chicos que aquellos que tenemos si usamos BDD por tanto cuando estos se saturan hay que partir nuestra información, en nuestro ejemplo seria tener que dividir nuestra base de usuarios entre 2 o mas archivos.\\
    \item \textbf{Integridad de los datos:} El problema es que en los archivos es difícil mantener el formato de nuestros datos, entonces por ejemplo si en un archivo se guardan los nombres en formato <Nombre APaterno AMaterno> esto se tiene que mantener por el programador y siempre que haya un nuevo archivo o nuevo programa se debe verificar que se tenga esta restricción.\\
    \item \textbf{Falta de atomicidad:} Es importante que se mantengan partes independientes para prevenir hacer errores, lo mas importante es que si una operación se va a realizar debe poderse realizar completa o no realizarse, por ejemplo, si un usuario mete su CV a una postulación, no debe pasar que el lo suba y no le llegue, o que lo pueda subir mas de una vez porque no se registro.\\
    \item \textbf{Anomalías de concurrencia:} Sucede que en muchos caos ocupamos que muchos usuarios puedan acceder a la información al mismo tiempo y en muchos casos que modifiquen esta información, es así que surgen complicaciones, porque digamos en nuestro ejemplo que varios administradores de la pagina de nuestro "linkedin" quieren modificar la cantidad de personas que quieren en una postulación, si ambos modifican el dato y no se trata con cuidado es posible que solo se quede uno de los datos, pero también es posible que se solapen ambos cambios y quede mal el dato.\\
    \item \textbf{Seguridad:} Esto es simple, los archivos tienen medidas de seguridad muy débiles, controlar quien puede ver y modificarlos es muy complicado, además, varias de las medidas de seguridad por defecto son mucho mas débiles que medidas usadas en BDD, en nuestro ejemplo es esencial que pocas personas tengan acceso a los datos de acceso de los usuarios y es esencial que se tenga fino control sobre quien puede modificar que datos en que momento.
\end{itemize}

\textcolor{blue}{Toda la información de esta sección se saco del texto proporcionado. \cite{PuprposeOfDBS}}