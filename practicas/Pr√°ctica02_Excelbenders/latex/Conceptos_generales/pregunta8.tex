\begin{center}
\textbf{Investiga por qué surgieron los sistemas NoSQL en la década de 2000 y compara a través de una tabla sus características vs. los sistemas de bases de datos relacionales.}
\end{center}

 Este tipo de sistemas se desarrollaron para abordar las limitaciones de los sistemas relacionales, que a menudo enfrentaban desafíos en términos de escalabilidad, flexibilidad y rendimiento al manejar grandes volúmenes de datos no estructurados. A medida que las aplicaciones web y móviles crecieron en popularidad, se necesitaban soluciones capaces de escalar horizontalmente, manejar datos diversos sin esquemas fijos y ofrecer alta disponibilidad con mejor tolerancia a fallos. Los sistemas NoSQL respondieron a estas necesidades al proporcionar modelos de datos más flexibles, mayor rendimiento y capacidades avanzadas para manejar datos en tiempo real y grandes volúmenes de información.

 Veamos la comparación de algunas características :

 \begin{table}[h!]
\centering
\small 
\begin{tabularx}{\textwidth}{|X|X|X|}
\hline
\textbf{Característica} & \textbf{\textcolor{myPurple}{Sistemas NoSQL}} & \textbf{\textcolor{lightblue}{Sistemas de Bases de Datos Relacionales}} \\
\hline
\textbf{Modelo de Datos} & Flexible (clave-valor, columnares, documentos, grafos) & Estructurado (tablas con esquema fijo) \\
\hline
\textbf{Escalabilidad} & Horizontal (agregar más servidores) & Vertical (mejorar hardware del servidor) \\
\hline
\textbf{Rendimiento} & Optimizado para alta disponibilidad y baja latencia & Puede ser más lento con grandes volúmenes de datos \\
\hline
\textbf{Flexibilidad del Esquema} & Esquema dinámico, datos semi-estructurados o no estructurados (más flexible) & Esquema fijo y predefinido (menos flexible) \\
\hline
\textbf{Garantías ACID} & A menudo sacrificadas por rendimiento y disponibilidad & Fuerte soporte para transacciones ACID \\
\hline
\textbf{Disponibilidad} & Diseñado para replicación y particionamiento distribuidos & Requiere configuraciones adicionales para alta disponibilidad \\
\hline
\textbf{Usos} & Aplicaciones web grandes, big data, datos en tiempo real & Aplicaciones empresariales tradicionales, sistemas transaccionales \\
\hline
\textbf{Cuándo usar} & Cuando el volumen de datos no crece tanto con el tiempo & Cuando el volumen de datos crece rápidamente \\
\hline
\end{tabularx}
\label{tab:nosql_vs_rdbms}
\end{table}

\textcolor{blue}{Toda la información de esta sección se ha tomado del texto proporcionado \cite{nosql}.} \\
