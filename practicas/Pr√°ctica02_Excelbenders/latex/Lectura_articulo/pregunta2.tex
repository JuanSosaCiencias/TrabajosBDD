\begin{center}
\textbf{Realizar un ensayo donde expresen sus comentarios (cada integrante del equipo deberá indicar este punto de manera individual en el documento que redacten) sobre la lectura, considerando los puntos vistos en el pdf.}\\
\end{center}

\begin{center}
    \textbf{Del Monte Ortega Maryam Michelle} 
\end{center}

El artículo planteado ofrece una crítica profunda y bien fundamentada sobre la analogía que equipara los datos con el petróleo. Esta metáfora, utilizada en discursos tecnológicos y económicos, busca enfatizar el valor crucial de los datos en la economía moderna, similar al papel que jugó el petróleo durante la Revolución Industrial. Sin embargo, Stach argumenta que esta comparación, aunque poderosa, es engañosa y puede llevar a malentendidos significativos sobre la naturaleza y gestión de los datos.

\textbf{Objetivo del Autor}

El objetivo del autor es desmitificar la analogía entre datos y petróleo y resaltar las diferencias fundamentales que hacen que los datos requieran un enfoque de administración único y especializado. Stach explica que, aunque los datos son un recurso valioso, su naturaleza es escencialmente distinta a la del petróleo. Los datos no son un recurso finito que se consume al usarse, por el contrario, son infinitos y pueden ser replicados, compartidos y reutilizados sin agotarse. Esta diferencia crucial implica que las estrategias de gestión de datos deben ser distintas de las empleadas para manejar recursos físicos como el petróleo.

\textbf{Temática Central y Temas Laterales}

La temática central del artículo es la crítica a la comparación entre datos y petróleo, destacando que, aunque ambos son recursos valiosos, sus características y desafíos de manejo son radicalmente diferentes. Por otro lado, entre los temas laterales que desarrolla el autor se encuentran la heterogeneidad de los datos, la necesidad de personalización en su procesamiento, y los retos relacionados con su almacenamiento y distribución. Estos temas son altamente relevantes para la práctica profesional en la administración de bases de datos, ya que subrayan la importancia de desarrollar estrategias de gestión de datos que sean adaptativas y específicas para cada contexto. 

Stach señala que los datos son heterogéneos, provienen de múltiples fuentes y formatos, y requieren procesos de limpieza y transformación antes de ser útiles. Esto contrasta con el petróleo, que, aunque también requiere refinamiento, sigue un proceso más estandarizado. Además, los datos pueden ser manipulados o alterados sin ser fácilmente detectados, lo que plantea desafíos significativos en términos de integridad y confiabilidad. En términos de almacenamiento, a diferencia del petróleo que se consume, los datos no se agotan, lo que significa que los volúmenes de datos que deben ser gestionados están en constante crecimiento. Esto exige soluciones de almacenamiento escalables y eficientes.

\textbf{Relación con Fundamentos de Bases de Datos}

El artículo tiene una conexión directa con conceptos basicos como la gestión eficiente de grandes volúmenes de datos, la integridad de la información y la seguridad de los sistemas de bases de datos. La discusión sobre la heterogeneidad de los datos y la necesidad de un refinamiento específico se alinea con la práctica de diseñar sistemas que sean capaces de manejar diferentes tipos de datos y satisfacer las necesidades específicas de los usuarios finales. Este enfoque es crucial en un entorno donde la personalización y la adaptabilidad son esenciales para el éxito de las aplicaciones basadas en datos.

\textbf{Comentarios Personales}

Desde mi perspectiva, el artículo de Stach ofrece una visión crítica y necesaria sobre cómo debemos abordar la administración de datos en la era digital. Estoy de acuerdo con el autor en que la metáfora del petróleo puede ser engañosa si se toma al pie de la letra, ya que puede llevar a suposiciones erróneas sobre cómo deben ser gestionados los datos. Por ejemplo, la idea de que los datos deben ser "extraídos" y "refinados" de manera similar al petróleo ignora la complejidad y diversidad de las fuentes de datos modernas, así como la necesidad de enfoques personalizados para su procesamiento y análisis. Sin embargo, también reconozco el valor de la metáfora como herramienta comunicativa. Al comparar los datos con el petróleo, se destaca su importancia crítica en la economía digital y se subraya la necesidad de invertir en infraestructuras y tecnologías que permitan su gestión eficiente. Esta comparación puede servir como punto de partida para discusiones más profundas sobre las estrategias necesarias para manejar los datos de manera efectiva, siempre y cuando se tenga en cuenta que las soluciones deben ser innovadoras y adaptadas a las características únicas de los datos.

El reconocimiento de las diferencias entre datos y petróleo es esencial para desarrollar estrategias de gestión que sean verdaderamente efectivas.

La naturaleza infinita y replicable de los datos significa que debemos abordar su gestión con un enfoque que priorice la seguridad, la privacidad y la integridad. Además, la heterogeneidad de los datos requiere soluciones de almacenamiento y procesamiento que sean flexibles y adaptables a diferentes contextos y necesidades.

\textbf{Conclusión}

El artículo de Stach nos invita a reconsiderar cómo entendemos y manejamos los datos en el mundo moderno. Al desafiar la comparación simplista con el petróleo, nos impulsa a desarrollar un enfoque de administración de datos que sea tan complejo y dinámico como el propio recurso. Este enfoque debe tener en cuenta no solo la importancia económica de los datos, sino también los retos únicos que presentan en términos de heterogeneidad, integridad y seguridad. Al hacerlo, podemos asegurarnos de que los datos continúen siendo un motor de innovación. Esta reflexión es crucial para cualquier persona involucrada en la gestión de datos en la era digital.

\begin{center}
    \textbf{Sosa Romo Juan Mario}
\end{center}

\textbf{Resumen:}
\begin{quote}
    El articulo básicamente trata el tema visto mucho en últimos tiempos que compara los datos con el petroleo y porque la comparación, aunque tenga ciertas bases verdaderas tiene algunos puntos que la pueden hacer una mala comparación.
\end{quote}

\textbf{Objetivo del autor y relación con FBDD}
\begin{quote}
    En el caso de este articulo, el autor busca explicar el porque esta comparación pierde mucha validez por características fundamentales de ambas entidades.
    \vspace{.2cm}   
    
    La cosa es que el valor relativo de las cosas para nosotros no solo se basan de que tan útil es la cosa si no en una combinación de esto y la relación que habrá entre oferta y demanda, es entonces que surge la primera gran diferencia entre los datos y el petroleo, el petroleo al ser un recurso finito y escaso y los datos son potencialmente infinitos y crecen de manera exponencial por tanto su valor mal manejado no es alto mientras que el del petroleo aun así es alto.
    \vspace{.2cm}   

    Además de esto el autor explica que el petroleo no procesado aun así es bastante valioso, mientras que los datos no procesados pierden muchísimo valor.\\
    Ahora, ¿como se relaciona todo esto con la materia? bueno el articulo toca sobre diferentes partes de la correcta gestion de datos como lo son la integridad y seguridad de estos mismos, escencialmente, datos en crudo son poco utiles pero datos procesados y estructurados si pueden llegar a compararse con el petroleo.
\end{quote}

\textbf{Temática e Importancia:}
\begin{quote}
    La temática ya la mencionamos, es la comparación petroleo, datos; aunque también toca un poco los temas de características de los datos, la necesidad de personalización y el valor potencial que tienen los datos si se les trata de la manera correcta. 
    \vspace{.2cm}   
    
    La importancia de estos temas es fundamental pues es verdad que nos encontramos en la época de la información donde, en todos los trabajos podemos generar y trabajar con grandes cantidades de datos entonces saber hacerlo puede generar grandes ventajas individuales o para el proceso.
\end{quote}

\textbf{Opinion personal:}
\begin{quote}
    Creo que el autor hace una buena critica a la comparación y definitivamente es cierto que los datos no son el nuevo petroleo; sin embargo, creo que se enfoca demasiado en los específicos e ignora un poco que la comparación solo esta para elevar el nivel de importancia que le asignamos a la información, ciertamente esta ultima tiene muchas "desventajas", respecto al petroleo pero a mi perspectiva, los datos presentan una oportunidad mucho mas interesante que el petroleo.
\end{quote}

\begin{center}
    \textbf{Castillo Hernández Antonio}
\end{center}

Veamos en primera instancia que en el artículo se nos da conocer una realidad en la cual los datos son comparables con el crudo/petroleo a nivel de importancia para muchas empresas, tal es así que en el 2011 el foro económico mundial dijo que los datos serian el \textit{“nuevo petroleo”}, dicha metáfora hace referencia a la creciente importancia de los datos en el mundo actual. 

Tal es así que en el artículo se menciona que los datos en la era moderna son casi tan importantes como el petroleo y suponen una revolución industrial casi como lo supuso dicho recurso natural en el siglo 19.\\

\textbf{Objetivo que quiso plantear el Autor:}\\

En general, pienso que el autor del artículo trata de convencernos de que si bien el petroleo y los datos son algo parecidos, ya que por ejemplo, en este se mencionan algunas características compartidas como lo son: el que ambos deben de ser descubiertos y extraídos, ambos deben ser refinados para transformarlos en un recurso usable, y además también una vez refinados y transformados ambos recursos, son almacenados y para mas tarde ser liberados a los clientes. Y si bien el manejo de ambos recursos involucra los mismos pasos, el autor sobre todo hace énfasis en tratar de remarcar las diferencias principales que los datos tienen sobre el crudo, destacando que los datos exigen un enfoque de gestión especializado. Sobre todo, la principal diferencia es que los datos se tratan de un recurso que no se consume ni se agota con el uso, los datos son infinitos y esto hace que se aborde de una manera distinta. Ahora bien, esto se relaciona con la materia de bases de datos ya que lo que precisamente se manejan en las bases de datos, son colecciones de distintos tipos datos, por lo cual todo lo relevante o lo que concierne a los datos es importante para nuestra materia. \\

\textbf{Temática central del Artículo:} \\

El tema principal que se aborda en el artículo es la comparación entre los datos en los distintos ámbitos en los que se usan y el petróleo, argumentando que si bien, ambos son recursos valiosos, los desafíos para el manejo de ambos son completamente distintos.

Además cabe destacar que algunos de los temas que se abordan en el artículo tienen que ver con el manejo de los datos, la seguridad de los datos, la forma en que se \textit{"refinan"} los datos, y el estudio de los mismos. Todos estos siendo temas de suma relevancia para la correcta gestión de los datos en Bases de datos.\\

Ahora bien, todo esto tiene una relación bastante estrecha con nuestra practica profesional, ya que en el artículo se mencionan varios de los procesos necesarios para hacer un correcto filtro y procesado de los datos con la finalidad de que estos nos puedan ser de utilidad. Se mencionan varias cosas relevantes a la privacidad y protección de los mismos así como también se muestra el trayecto que los datos realizan en orden de la función que estos desempeñan. Por lo tanto, todos los temas aquí presentados en el artículo, son de suma relevancia para nuestra practica profesional.\\

\textbf{Consideraciones Personales:} \\

Considero que los puntos a tratar en el artículo son de bastante relevancia en nuestra materia y practica profesional, ya que en los mismos se mencionan varios de los procesos necesarios para \textit{"refinar"} los datos. Además la metáfora de que los datos son el “nuevo petroleo”, cada vez cobra mas fuerza, sin embargo como nos lo comunica el autor a lo largo de todo el articulo, es necesario esclarecer bien a que se refiere dicha metáfora para no caer en ambigüedades y mal interpretar las cosas, ya que si bien los datos están teniendo el mismo valor a nivel de importancia que el petroleo, la forma de manejarlos y procesarlos junto con algunas de sus características, son muy diferentes. \\

A manera de conclusión, dado que vivimos en la era moderna de la información, los datos cada vez cobran mas y mas relevancia dentro de nuestras vidas cotidianas, por lo cual es de vital importancia tener algo de conocimiento acerca de temas relacionados a los mismos, además los temas tratados en el articulo son bastante relevantes ya que forman parte de cierta manera de nuestra formación académica como estudiantes de Ciencias de la computación. Por consiguiente, diría que el presente articulo en efecto nos proporciona una visión diferente de ver a los datos y a los procesos relacionados con ellos.\\

\begin{center}
    \textbf{Erik Eduardo Gómez López} 
\end{center}

En el artículo proporcionado, Christoph Stach busca desafiar la metáfora prevalente que equipara los datos con el petróleo, afirmando que, si bien ambos son recursos valiosos, su manejo e implicaciones difieren significativamente. El autor intenta persuadir a los lectores de que entender estas diferencias es importante para una administración de datos efectiva en el contexto de la cuarta revolución industrial y la era digital. Esta discusión es justamente relevante para la materia de Fundamentos de Bases de Datos, ya que enfatiza las complejidades y características únicas de los datos que los profesionales de bases de datos deben enfrentar.

El tema central del artículo de Stach gira en torno a los desafíos  asociados con los datos como recurso. Identifica diez características específicas que distinguen a los datos de las mercancías tangibles como el petróleo, tales como la naturaleza no consumible de los datos, su capacidad de ser duplicados sin pérdida, y la alta velocidad a la que se generan. Estas características requieren enfoques innovadores para la gestión de datos que difieren de los métodos tradicionales utilizados para las mercancías físicas. Por ejemplo, mientras el petróleo puede ser extraído y consumido, los datos permanecen disponibles para su procesamiento y análisis continuo, lo que lleva a un volumen cada vez mayor que debe ser gestionado eficientemente.

Stach también aborda temas secundarios relacionados con la refinación de datos, la seguridad, la privacidad y el valor económico de los datos. Destaca la necesidad de procesos de preparación de datos adaptados al uso previsto de los mismos, así como la importancia de garantizar la calidad y autenticidad de los datos. Estos temas son particularmente pertinentes para los profesionales en el campo de la gestión de bases de datos, ya que subrayan la necesidad de desarrollar estrategias robustas de administración de datos que aborden los desafíos únicos que plantea el big data.

Desde mi perspectiva, los conocimientos presentados en el artículo de Stach resuenan profundamente con los principios del diseño y la gestión de bases de datos. Comprender las características distintivas de los datos informa cómo abordo el modelado de datos, las soluciones de almacenamiento y los procesos de recuperación de datos. Por ejemplo, el reconocimiento de que los datos pueden duplicarse sin pérdida influye en las decisiones relacionadas con las arquitecturas de almacenamiento de datos, ya que requiere mecanismos eficientes de indexación y recuperación para manejar grandes volúmenes de datos sin comprometer el rendimiento.

Además, el énfasis en la seguridad y privacidad de los datos se alinea con la creciente importancia del cumplimiento de las regulaciones de protección de datos, como el Reglamento General de Protección de Datos (GDPR). Debemo asegurarnos de que las prácticas de gestión de datos no solo faciliten el acceso efectivo a los datos, sino que también protejan la información sensible y respeten los derechos de los titulares de los datos.

En conclusión, el artículo de Stach sirve como un recordatorio crítico de las complejidades involucradas en la administración moderna de datos. Al esclarecer las diferencias entre los datos y el petróleo, el autor anima a los profesionales de bases de datos a replantear los enfoques tradicionales y adoptar estrategias innovadoras que aborden los desafíos únicos de manejar datos en el panorama digital actual. Esta perspectiva es esencial para garantizar que los datos se gestionen de manera efectiva, segura y de una forma que maximice su valor como activo estratégico en la era de la información.

\begin{center}
    \textbf{Julio César Islas Espino}    
\end{center}

Este artículo fue de suma importancia pues compara el petróleo crudo con los datos en sí a mí se me hace una comparación un poco ambigua pues hay puntos en los que no es necesario comprarlos pues son distintos en varios aspectos pero eso es lo que trata del artículo los compara y él nos da datos de artículos nos da datos científicos de cómo son casi iguales pero no son lo mismo como por ejemplo en los mismos pasos pero tiene unas características especiales los datos que la hace diferente al petróleo.

El autor nos da varios puntos en los cuales nos dice por qué esta comparación pierde validez, una de las cuáles son las tantas características que tiene en los datos.

Nos dice que los datos son infinitos y en cambio que los recursos las materias son finitos pues los datos se pueden generar más y más y más y por esa razón se necesitan grandes almacenes de información grandes buffers para que esa información de los datos pueda ser almacenada esa es una gran diferencia que nos pone y por supuesto el valor que la oferta y la demanda que es muy distinto la oferta y la demanda en la materia que en los datos pues en los datos conforme sea más único y más original va a cambiar el valor de los datos.

Nos da un enfoque primordialmente en la comparación del petróleo con los datos aunque también nos da un enfoque muy práctico en las características de los datos lo que tienen qué es lo que hacen la personalización de ellos y cómo tratarlos de manera correcta.


Me gusta como el autor hace la comparación del petróleo con los datos pues en si todas las comparaciones que hace se me hacen muy correctas aunque algunas no tanto para comparar, siendo sincero, pero sin embargo veo que se enfoca demasiado en a fuerzas comparar el petróleo con algo que tiene muchas pero muchas características que son los datos.
Hoy para concluir se me hace de vital importancia que haya hecho una comparación tan buena comparando con datos y científicos el petróleo con los datos pues conforme va avanzando la tecnología va avanzando los datos y va modernizándose las cosas y por supuesto hay muchos temas de relación que tiene y estos datos conforme pasa el tiempo se van haciendo más de nuestra vida cotidiana y son cosas que debemos aprender de ello que como estudiante de ciencias de la computación es algo que a futuro lo vamos a ocupar sí o sí.
