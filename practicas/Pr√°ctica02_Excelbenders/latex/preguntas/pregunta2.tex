\begin{center}
    \textbf{Describe cual es el mas conveniente utilizar.}
    \vspace{.5cm}
\end{center}

Depende de la complejidad de los datos y las necesidades del proyecto.

Un \textbf{Sistema de Datos} es adecuado cuando se requiere una estructura simple y poco procesamiento. Si solo necesitas almacenar archivos en carpetas y no hay relaciones complejas entre los datos, un sistema de archivos es una opción eficiente. Además, es más fácil de implementar y mantener en proyectos pequeños o específicos.

En cambio, una \textbf{Base de Datos} es preferible cuando se necesita manipular grandes volúmenes de datos estructurados, realizar consultas avanzadas, o cuando se requieren funcionalidades como seguridad, control de acceso, integridad referencial o escalabilidad. Las bases de datos permiten automatizar procesos complejos, mejorar la productividad y reducir errores, facilitando la toma de decisiones en empresas que manejan datos críticos y en constante evolución.\\

\cite{sooluciona}\\

Adicionalmente a esto, en el transcurso de esta practica notamos lo tedioso y complicado que es 
modelar datos en un sistema de archivos, ya que se debe tener en cuenta la estructura de los archivos,
la forma en la que se van a leer, la forma en la que se van a escribir, etc. Por otro lado, en una base de datos
relacional, se puede modelar los datos de una forma más sencilla y se puede acceder a ellos de una forma más sencilla.\\ 
Al final incluso con esta pequeña BDD y agregando solo 15 registros, notamos que se podia tardar un monton simplemente
porque tiene que ir validando dato por dato y checando que no haya duplicados y asi, aunque cabe aclarar, nuestra implementacion
fue lejos de la deseada.\\
\vspace{.5cm}
