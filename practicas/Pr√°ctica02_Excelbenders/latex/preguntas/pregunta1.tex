\begin{center}
    \textbf{Menciona 5 diferencias sentre almacenar la informacion utilizando un sistema de archivos o almacenarla utilizando una BDD.}
    \begin{enumerate}
    \item \textbf{Estructura de Datos:} Los sistemas de archivos son simples colecciones de archivos sin relación entre ellos, mientras que             las bases de datos organizan los datos de manera estructurada y con relaciones lógicas.
    \item \textbf{Redundancia:} La redundancia de datos es alta en los sistemas de archivos, ya que los mismos datos pueden aparecer en                 múltiples lugares. En las bases de datos, la redundancia se minimiza mediante normalización.
    \item \textbf{Consistencia de Datos:} Los sistemas de archivos tienen problemas de inconsistencia cuando los datos se modifican en                 varios archivos. En las bases de datos, las actualizaciones se reflejan de manera consistente en todas las instancias de los datos.
    \item \textbf{Seguridad:} Los sistemas de archivos suelen ofrecer menos seguridad, mientras que las bases de datos incluyen medidas de             seguridad avanzadas como control de acceso y encriptación. 
    \item \textbf{Copia de Seguridad y recuperación:} Los sistemas de archivos no cuentan con mecanismos automatizados de respaldo y                     recuperación, mientras que las bases de datos generalmente incluyen estas funciones para proteger la información.\\
    \cite{guru99, sooluciona}
\end{enumerate}
\end{center}
