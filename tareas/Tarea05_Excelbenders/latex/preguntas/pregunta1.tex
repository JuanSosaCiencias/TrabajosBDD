Dada una relación \textbf{R(A, B, C, D, E, G)} y el siguiente conjunto de dependencias funcionales \textbf{F}:

\begin{center}
    \textbf{F = \{AB $\rightarrow$ C, BC $\rightarrow$ D, D $\rightarrow$ EG, CG $\rightarrow$ BD, C $\rightarrow$ A, ACD $\rightarrow$ B, BE $\rightarrow$ C, CE $\rightarrow$ AG \}} 
\end{center}

Para las siguientes sentencias, determina si son \textbf{verdaderas} o \textbf{falsas}. Para aquellas sentencias que resulten falsas, deberás \textbf{explicar} por qué consideras que no se cumplen:

\begin{table}[h!]
    \centering
    \renewcommand{\arraystretch}{1.5}
    \begin{tabular}{|c|m{5cm}|c|c|m{6cm}|}
    \hline
    \textbf{No.} & \textbf{Sentencia} & \textbf{Verdadera} & \textbf{Falsa} & \textbf{Justificación} \\ \hline

    1 & La cerradura de $BC$ es $\{A, D,$ $ E, G\}$ & & \textcolor{blue}{\checkmark} & \{BC\}+ = \{BCDEGA\} \\ \hline

    2 & Todos los atributos de $R$ están en la cerradura de $BC$ & \textcolor{blue}{\checkmark} & & \\ \hline

    3 & La cerradura de $AC$ es $\{A, C\}$ & \textcolor{blue}{\checkmark} & & \\ \hline

    4 & $ABC$ es una superllave de $R$ & \textcolor{blue}{\checkmark} & & Como BC es llave pues tiene todos los atributos de R, agregar A significa que es superllave. \\ \hline

    5 & $ABC$ es una llave candidata de $R$ & & \textcolor{blue}{\checkmark} & Contiene \textbf{redundancia} podemos eliminar A sin destruir la propiedad de \textbf{identificación única}. \\ \hline

    6 & $BC$ es la única llave candidata de $R$ &  & \textcolor{blue}{\checkmark} & Porque \{AB\}+ = \{ABCDEG\} cumple con \textbf{identificación única} y \textbf{no redundancia}.\\ \hline
    \end{tabular}
\end{table}