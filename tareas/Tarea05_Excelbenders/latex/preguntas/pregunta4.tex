Para cada uno de los esquemas que se muestran a continuación, con su respectivo conjunto de dependencias funcionales:

\begin{itemize}
	\item[a)] \( R(A, B, C, D, E, F, G) \) con \( F = \{AB \to C, AB \to F, A \to D, A \to E, B \to G\} \).
	\item[b)] \( R(A, B, C, D, E, F, G) \) con \( F = \{A \to B, CD \to FG, G \to E, B \to D, A \to C, E \to A\} \).
\end{itemize}

\subsubsection*{Inciso a)}
\begin{itemize}
	\item Indica alguna llave candidata para la relación \( R \).

		Una llave candidata para la relación \( R \) podría ser \( AB \), ya que:
		\begin{itemize}
			\item \( AB \rightarrow C \)
			\item \( AB \rightarrow F \)
			\item \( A \rightarrow D \) y \( A \rightarrow E \) implican que, conociendo \( A \), se puede determinar \( D \) y \( E \).
			\item \( B \rightarrow G \) implica que, conociendo \( B \), se puede determinar \( G \).
		\end{itemize}
		Por lo tanto, \( AB \) es una llave candidata.
	\vspace*{.3cm}
	\item Indica las violaciones a 3NF que encuentres en \( R \).
	
	Las siguientes dependencias violan la 3NF:
	\begin{itemize}
		\item \( A \rightarrow D \): \( D \) no es parte de una llave candidata y \( A \) no es superllave.
		\item \( A \rightarrow E \): \( E \) no es parte de una llave candidata y \( A \) no es superllave.
		\item \( B \rightarrow G \): \( G \) no es parte de una llave candidata y \( B \) no es superllave.
	\end{itemize}
	\vspace*{.3cm}
	
	\item Encuentra el conjunto mínimo de dependencias funcionales equivalente a \( F \).
	\begin{itemize}
		\item Paso a paso:
		\begin{itemize}
			\item \( AB \rightarrow C \): No es redundante, ya que necesitamos \( AB \) para determinar \( C \).
			\item \( AB \rightarrow F \): No es redundante, necesitamos \( AB \) para determinar \( F \).
			\item \( A \rightarrow D \): No es redundante, ya que no podemos deducir \( D \) de ninguna otra dependencia sin \( A \).
			\item \( A \rightarrow E \): No es redundante, ya que necesitamos \( A \) para determinar \( E \).
			\item \( B \rightarrow G \): No es redundante, necesitamos \( B \) para determinar \( G \).
		\end{itemize}
		Por lo tanto, el conjunto mínimo equivalente a \( F \) es:
		\[
		F_{\text{mín}} = \{ AB \rightarrow C, AB \rightarrow F, A \rightarrow D, A \rightarrow E, B \rightarrow G \}
		\]
	\end{itemize}
	\vspace*{.3cm}
	
	\item Normaliza de acuerdo con la 3NF. Indica claramente las relaciones resultantes y, en cada esquema, las dependencias funcionales que se cumplen.
	\begin{itemize}
		\item Relación 1: \( R_1(A, B, C, F) \) con las dependencias \( AB \rightarrow C \) y \( AB \rightarrow F \).
		\item Relación 2: \( R_2(A, D, E) \) con las dependencias \( A \rightarrow D \) y \( A \rightarrow E \).
		\item Relación 3: \( R_3(B, G) \) con la dependencia \( B \rightarrow G \).
	\end{itemize}
	Estas relaciones cumplen con la 3NF, ya que todas las dependencias están cubiertas.
\end{itemize}
\vspace*{.3cm}

\subsubsection*{Inciso b)}
\begin{itemize}
	\item Indica alguna llave candidata para la relación \( R \).
	
	Una posible llave candidata para la relación \( R \) es \( CD \), ya que:
	\begin{itemize}
		\item \( CD \rightarrow FG \): Nos da acceso a \( F \) y \( G \).
		\item \( G \rightarrow E \) implica que podemos determinar \( E \) si conocemos \( G \).
		\item \( E \rightarrow A \) permite determinar \( A \).
		\item \( A \rightarrow B \) y \( A \rightarrow C \) permiten determinar \( B \) y \( C \).
	\end{itemize}
	Por lo tanto, \( CD \) es una llave candidata.
	\vspace*{.3cm}
	
	\item Indica las violaciones a 3NF que encuentres en \( R \).
	
	Las siguientes dependencias violan la 3NF:
	\begin{itemize}
		\item \( G \rightarrow E \): \( E \) no es parte de una llave candidata y \( G \) no es superllave.
		\item \( B \rightarrow D \): \( D \) no es parte de una llave candidata y \( B \) no es superllave.
		\item \( A \rightarrow C \): \( C \) no es parte de una llave candidata y \( A \) no es superllave.
	\end{itemize}
	\vspace*{.3cm}
	
	\item Encuentra el conjunto mínimo de dependencias funcionales equivalente a \( F \).
	
	Pasos:
	\begin{itemize}
		\item \( A \rightarrow B \): No es redundante, ya que necesitamos \( A \) para determinar \( B \).
		\item \( CD \rightarrow FG \): No es redundante, necesitamos \( CD \) para determinar \( F \) y \( G \).
		\item \( G \rightarrow E \): No es redundante, necesitamos \( G \) para determinar \( E \).
		\item \( B \rightarrow D \): No es redundante, necesitamos \( B \) para determinar \( D \).
		\item \( A \rightarrow C \): No es redundante, necesitamos \( A \) para determinar \( C \).
		\item \( E \rightarrow A \): No es redundante, necesitamos \( E \) para determinar \( A \).
	\end{itemize}
	El conjunto mínimo equivalente a \( F \) es:
	\[
	F_{\text{mín}} = \{ A \rightarrow B, CD \rightarrow FG, G \rightarrow E, B \rightarrow D, A \rightarrow C, E \rightarrow A \}
	\]
	\vspace*{.3cm}
	
	\item Normaliza de acuerdo con la 3NF. Indica claramente las relaciones resultantes y, en cada esquema, las dependencias funcionales que se cumplen.
	\begin{itemize}
		\item Relación 1: \( R_1(A, B, C) \) con las dependencias \( A \rightarrow B \) y \( A \rightarrow C \).
		\item Relación 2: \( R_2(C, D, F, G) \) con la dependencia \( CD \rightarrow FG \).
		\item Relación 3: \( R_3(G, E) \) con la dependencia \( G \rightarrow E \).
		\item Relación 4: \( R_4(B, D) \) con la dependencia \( B \rightarrow D \).
		\item Relación 5: \( R_5(E, A) \) con la dependencia \( E \rightarrow A \).
	\end{itemize}
	Estas relaciones cumplen con la 3NF, ya que todas las dependencias están cubiertas.
\end{itemize}