Considera la siguiente tabla, donde cada \textbf{proyecto} tiene asignados \textbf{muchos empleados} y cada \textbf{empleado} trabaja en \textbf{muchos proyectos}. Se muestra a continuación un extracto de la tabla. Se muestra a continuación un extracto de la tabla  \textbf{Proyectos}: \vspace{1cm}


\begin{table}[h!]
    \centering
    \begin{tabular}{|c|c|c|c|c|c|}
        \hline
        \textbf{NumProy} & \textbf{NombreProy} & \textbf{Presupuesto} & \textbf{idEmp} & \textbf{NombreEmp} & \textbf{HrTrabajadas} \\ \hline
        P22 & Cyclone & 50000 & E1001 & Carlos & 12 \\ \hline
        P22 & Cyclone & 50000 & E2002 & Juan & 50 \\ \hline
        P21 & IBM & 20000 & E3003 & Patricia & 40 \\ \hline
        P21 & IBM & 20000 & E2002 & Juan & 30 \\ \hline
        P21 & IBM & 20000 & E1001 & Carlos & 70 \\ \hline
    \end{tabular}
    \label{tab:proyectos_empleados}
\end{table}

\vspace{1cm}

\begin{itemize}

    \item ¿Qué \textbf{problemas} consideras que puede haber al almacenar los datos en esta tabla? Describe los problemas en
    términos de las \textbf{anomalías} que se pueden presentar. \vspace{1cm}

    Algunos de los principales problemas que vemos en la tabla de este esquema donde cada proyecto puede tener multiples empleados y cada empleado puede trabajar en multiples proyectos son: \vspace{1cm}

    \begin{itemize}
        \item \textbf{Redundancia de datos}: Esto es debido a que los numeros de proyectots \textit{(NumProy)}, nombres de proyectos \textit{(NombreProy)}, presupuestos \textit{(Presupuesto)}, \textit{(idEmp)} y nombres de empleados \textit{(NombreEmp)} se repiten.
        \item \textbf{Anomalía de inserción}: No podemos agregar un proyecto sin asociarlo a alemos un empleado y viceversa.
        \item \textbf{Anomalía de actualización}: Si un empleado cambia de nombre, o si se actualiza el nombre de un proyecto o si se cambia de presupuesto, \textit{(en general, si actualiza alguna información)} se debe actualizar en todas las filas donde aparezca información relacionada lo que puede llevar a inconsistencias.
        \item \textbf{Anomalía de eliminación}: Si se elimina una fila, se puede perder información sobre el proyecto o el empleado si no hay otras filas que contengan esa información. \vspace{1cm}
    \end{itemize}

    \item ¿Cuáles son las dependencias funcionales que cumplen en la relación \textbf{Proyectos}? \vspace{1cm}
    
    Las dependencias funcionales en la tabla \textbf{Proyectos} se identifican observando cómo ciertos atributos determinan de forma única a otros. Analizamos la relación entre los atributos en cada fila, lo que nos permite inferir las siguientes dependencias:

\begin{itemize}
    \item \textbf{NumProy $\rightarrow$ NombreProy, Presupuesto}: Cada número de proyecto \textit{(NumProy)} está asociado de manera única a un nombre de proyecto \textit{(NombreProy)} y un presupuesto. Esto significa que, si conocemos el número de proyecto, podemos deducir su nombre y presupuesto. \textbf{Por Ejemplo:} El proyecto con \textit{NumProy = P22} siempre tiene el nombre Cyclone y el presupuesto de 50000 en todas las filas donde aparece. \vspace{1cm}
    
    \item \textbf{idEmp $\rightarrow$ NombreEmp}: Cada empleado tiene un identificador único \textit{(idEmp)} que determina su nombre. Esto significa que, con el identificador del empleado, es posible determinar un solo valor para el nombre del empleado. \textit{Veamos el siguiente ejemplo:} El empleado con \textit{idEmp = E1001} tiene siempre el nombre Carlos en las filas en las que aparece. \vspace{1cm}
    
    \item \textbf{NumProy, idEmp $\rightarrow$ HrTrabajadas}: Las \textit{(HrTrabajadas)} dependen tanto del proyecto como del empleado, ya que cada combinación de proyecto-empleado define una cantidad única de horas trabajadas. Por lo tanto, se requiere conocer ambos valores \textit{(NumProy e idEmp)} para determinar las horas trabajadas. Por ejemplo: La combinación de \textit{(P22, E1001)} da \textit{HrTrabajadas = 12}, mientras que \textit{(P22, E2002)} da \textit{HrTrabajadas = 50}. \vspace{1cm}
\end{itemize}
\vspace{1cm}
    
    \item ¿Cuál sería alguna llave para la relación \textbf{Proyectos}? \vspace{1cm}
    
    Una llave candidata para la relación \textbf{Proyectos} es el conjunto de atributos \textbf{\{NumProy, idEmp\}} ya que la combinación de \textbf{NumProy} (Número de Proyecto) y \textbf{idEmp} (Identificación del Empleado) es única para cada registro en la tabla. En otras palabras, cada par de \textbf{NumProy} e \textbf{idEmp} determina de manera única las \textbf{HrTrabajadas} (Horas Trabajadas) por un empleado en un proyecto específico. Por lo tanto, \textbf{\{NumProy, idEmp\}} cumple con los requisitos de una llave candidata, ya que no hay dos filas en la tabla que tengan la misma combinación de estos valores. \vspace{1cm}

    \item ¿La relación \textbf{Proyectos} cumple con \textbf{BCNF}? Justifica tu respuesta. \vspace{1cm}
    
    Para determinar si la relación \textbf{Proyectos} cumple con la (BCNF), necesitamos verificar si todas las dependencias funcionales no triviales en la relación tienen como lado izquierdo una superllave. Recordemos que una relación está en BCNF si, para toda dependencia funcional \( X \rightarrow Y \), \( X \) es una superllave de la relación. \vspace{1cm}

En esta relación \textbf{Proyectos}, identificamos las siguientes dependencias funcionales: \vspace{1cm}

\begin{itemize}
    \item \textbf{NumProy $\rightarrow$ NombreProy, Presupuesto}: Esto indica que el número de proyecto \textit{(NumProy)} determina de manera única tanto el nombre del proyecto como su presupuesto. Sin embargo, \textbf{NumProy} no es una superllave de la relación, ya que no determina de manera única todos los atributos de la tabla (por ejemplo, no determina \textit{idEmp} ni \textit{HrTrabajadas}). Esta dependencia funcional hace que la relación \textbf{Proyectos} no esté en BCNF. \vspace{1cm}
    
    \item \textbf{idEmp $\rightarrow$ NombreEmp}: Aquí, el identificador del empleado \textit{(idEmp)} determina de forma única el nombre del empleado \textit{(NombreEmp)}. Similar al caso anterior, \textbf{idEmp} no es una superllave de la relación, pues no determina atributos como \textit{NumProy} o \textit{HrTrabajadas}. Esto también viola la condición de BCNF. \vspace{1cm}
    
    \item \textbf{NumProy, idEmp $\rightarrow$ HrTrabajadas}: La combinación de \textbf{NumProy} e \textbf{idEmp} determina de manera única las horas trabajadas \textit{(HrTrabajadas)}, y esta combinación es una llave candidata de la relación, ya que identifica de forma única todas las tuplas en la tabla. Esta dependencia sí cumple con la condición de BCNF. \vspace{1cm}
\end{itemize}

Aunque \textbf{NumProy, idEmp $\rightarrow$ HrTrabajadas} cumple con BCNF, las dependencias \textbf{NumProy $\rightarrow$ NombreProy, Presupuesto} y \textbf{idEmp $\rightarrow$ NombreEmp} no lo hacen, ya que sus lados izquierdos no son superllaves. $\Longrightarrow$ \textbf{la relación Proyectos no está en BCNF} debido a que existen dependencias funcionales donde el determinante no es una superllave. Para cumplir con BCNF, sería necesario descomponer la tabla en relaciones que eliminen estas dependencias.  \vspace{1cm}
\vspace{1cm}

\end{itemize}


