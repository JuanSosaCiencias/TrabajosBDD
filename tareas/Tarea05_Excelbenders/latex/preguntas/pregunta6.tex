Se tiene la siguiente relación:
\begin{center}
    \textbf{R(idEnfermo, idCirujano, fechaCirugía, nombreEnfermo, direcciónEnfermo, nombreCirujano, nombreCirugía, medicinaSuministrada, efectosSecundarios)}
\end{center}
\begin{itemize}
    \item \textbf{Expresa las siguientes restricciones en forma de dependencias funcionales:}
    
    \begin{itemize}
        \item A un enfermo sólo se le da una medicina después de la operación. 
        \item Si existen efectos secundarios estos dependen sólo de la medicina suministrada.
        \item Sólo puede existir un efecto secundario por medicamento.
    \end{itemize}

    Podemos satisfacer las restricciones con:
    \begin{itemize}
        \item $idEnfermo, fechaCirugia \rightarrow medicinaSuministrada$

        Esto porque indica que para un enfermo específico en una fecha de cirugía específica, solo se le puede administrar una medicina. Lo cuál cubre la primera restricción. 

        Ocupamos tanto el idEnfermo como la fecha de la cirugía, o de otra manera no sabríamos a que enfermo administrarle la medicina después de su cirugía (fecha).

        \item $medicinaSuministrada \rightarrow efectosSecundarios$

        Aquí cumplimos con las dos restricciones que faltaban, porque asignamos a cada medicina un único efecto secundario. En este caso, es una dependencia funcional simple ya que solo ocuparemos la medicina para saber qué efecto secundario tiene. 
    \end{itemize}
    
    \item \textbf{Especifica otras dependencias funcionales o multivaluadas que deban satisfacerse en la relación \textbf{R}. Por cada una que definas, deberá aparecer un enunciado en español como en el inciso anterior.}

    Otras dependencias funcionales en \textbf{R} son:
    \begin{itemize}[label=$\heartsuit$]
        \item Cada enfermo tiene un único nombre y una única dirección
        \[idEnfermo \rightarrow nombreEnfermo, direccionEnfermo\]
        \item Cada cirujano tiene un único nombre
        \[idCirujano \rightarrow nombreCirujano\]
        \item Una cirugía realizada por un cirujano a un enfermo sólo puede ocurrir en una fecha determinada
        \[idEnfermo, idCirujano, nombreCirugia \rightarrow fechaCirugia\]
        \item Un enfermo en una fecha específica solo puede tener una cirugía y esta debe ser realizada por un único cirujano
        \[idEnfermo, fechaCirugia \rightarrow nombreCirugia, idCirujano\]
        \item El enfermo y la fecha de cirugía determina el cirujano que realizó la operación
        \[idEnfermo, fechaCirugia \rightarrow idCirujano\]
        \item El cirujano y la fecha determina qué cirugía se está realizando en ese momento
        \[idCirujano, fechaCirugia \rightarrow nombreCirugia\]
       
        \item El nombre de la cirugía determina qué medicinas pueden ser suministradas después de la operación
        \[nombreCirugia \twoheadrightarrow  medicinaSuministrada\]
        Esta es una dependencia multivaluada ya que una cirugía puede tener varias medicinas posibles
        \item La combinación de enfermo, cirujano y fecha determina unívocamente todos los demás atributos de la relación

        \begin{center}
            
        $idEnfermo, idCirujano, fechaCirugia \rightarrow nombreEnfermo, direccionEnfermo, nombreCirujano, nombreCirugia,$
        
        $medicinaSuministrada, efectosSecundarios$
        \end{center}
        

    \end{itemize}
    Estas dependencias surgen de restricciones pensadas por nosotros mismos, por lo que podrían no ser generales en un contexto de un hospital real.
    
    \item \textbf{Normaliza utilizando el conjunto de dependencias establecido en los puntos anteriores.}

    Tenemos el siguiente conjunto de dependencias:

    \textbf{F} = \{
    
    1. idEnfermo $\rightarrow$ nombreEnfermo, direccionEnfermo
    
    2. idCirujano $\rightarrow$ nombreCirujano
    
    3. idEnfermo, idCirujano, nombreCirugia $\rightarrow$ fechaCirugia
    
    4. idEnfermo, fechaCirugia $\rightarrow$ nombreCirugia, idCirujano
    
    5. idEnfermo, fechaCirugia $\rightarrow$ idCirujano
    
    6. idCirujano, fechaCirugia $\rightarrow$ nombreCirugia
    
    7. nombreCirugia $\twoheadrightarrow$ medicinaSuministrada
    
    8. idEnfermo, idCirujano, fechaCirugia $\rightarrow$ nombreEnfermo, direccionEnfermo, nombreCirujano, nombreCirugia, medicinaSuministrada, efectosSecundarios
    
    9. idEnfermo, fechaCirugia $\rightarrow$ medicinaSuministrada
    10. medicinaSuministrada $\rightarrow$ efectosSecundarios
\}

Como tenemos una dependencia multivaluada, usaremos $4FN$

Debido a la DF 8, la llave candidata es:

\{idEnfermo, fechaCirugia\}+=
\{idEnfermo fechaCirugia, nombreEnfermo direccionEnfermo, nombreCirugia idCirujano, nombreCirujano, medicinaSuministrada, efectosSecundarios\}


\textbf{PASO 1:} Primera violación a 4FN
\begin{itemize}
    \item La DMV nombreCirugia $\twoheadrightarrow$ medicinaSuministrada es una violación ya que nombreCirugia no es superllave
    
    \item Dividimos en dos relaciones:
    \begin{itemize}[label=\textcolor{magenta}{$\bigstar$}]
        \item $R_1$(nombreCirugia, medicinaSuministrada) 
        
        Ya está en $4FN$ pues no tiene DMVs no triviales
        \item $R_2$(idEnfermo, idCirujano, fechaCirugia, nombreEnfermo, direccionEnfermo, nombreCirujano, nombreCirugia, efectosSecundarios) con:
        \begin{itemize}
            \item idEnfermo $\rightarrow$ nombreEnfermo, direccionEnfermo
            \item idCirujano $\rightarrow$ nombreCirujano
            
            \item idEnfermo, fechaCirugia $\rightarrow$ nombreCirugia, idCirujano
        \end{itemize}
    \end{itemize}
\end{itemize}


\textbf{PASO 2}: Analizamos $R_2$

Obtenemos su llave:

\{idEnfermo fechaCirugia\}+= \{idEnfermo  fechaCirugia, nombreEnfermo  direccionEnfermo, nombreCirugia idCirujano,
nombreCirujano, efectosSecundarios\} 

Descomponemos $R_2$:
\begin{itemize}[label=\textcolor{blue}{$\clubsuit$}]
    \item Por $idEnfermo \rightarrow nombreEnfermo, direccionEnfermo$:
    \[
    R_3(idEnfermo, nombreEnfermo, direccionEnfermo)
    \]

    Llave: \{idEnfermo\}+=\{idEnfermo,nombreEnfermo  direccionEnfermo\} 

    Está en 4FN pues está en BCNF y no tiene DMVs
    \item Por $idCirujano \rightarrow nombreCirujano$:
    \[
    R_4(idCirujano, nombreCirujano)
    \]
    Llave: \{idCirujano\}+=\{idCirujano, nombreCirujano\}

    Está en 4FN 
    \item Por último obtendríamos
\[R_5(idEnfermo, idCirujano, fechaCirugia, nombreCirugia)\]
con
\begin{itemize}
    \item idEnfermo, fechaCirugia $\rightarrow$ nombreCirugia, idCirujano
    \item idEnfermo, idCirujano, fechaCirugia $\rightarrow$ nombreCirugia

\end{itemize}
Calculamos la llave 

\{idEnfermo idCirujano fechaCirugia\}+=
\{idEnfermo idCirujano fechaCirugia, nombreCirugia\}

\{idEnfermo fechaCirugia\}+=
\{idEnfermo fechaCirugia, nombreCirugia idCirujano\}

Por lo tanto la llave es \{idEnfermo, fechaCirugia\}

Está en 4FN pues está en BCNF y no tiene DMVs

\end{itemize}

El esquema final sería:
\begin{center}
    $R_1(nombreCirugia, medicinaSuministrada)$
    
    $R_3(idEnfermo, nombreEnfermo, direccionEnfermo)$
    
    $R_4(idCirujano, nombreCirujano)$
    
    $R_5(idEnfermo, idCirujano, fechaCirugia, nombreCirugia)$

\end{center}

Incluso podemos renombrar las relaciones para ser explícitos:
\begin{center}
    $MEDICINAS\_POR\_CIRUGIA(nombreCirugia, medicinaSuministrada)$
    
    $ENFERMO(idEnfermo, nombreEnfermo, direccionEnfermo)$
    
    $CIRUJANO(idCirujano, nombreCirujano)$
    
    $CIRUGIA(idEnfermo, idCirujano, fechaCirugia, nombreCirugia)$

\end{center}
\end{itemize}

