Para cada uno de los \textbf{esquemas} que se muestran a continuación, con su respectivo \textbf{conjunto de dependencias funcionales}:

\begin{enumerate}[label=\alph*.]
    \item \textbf{R(A, B, C, D, E, F, G)} con \textbf{F = \{AB $\rightarrow$ C, AB $\rightarrow$ F, A $\rightarrow$ D, A $\rightarrow$ E, B $\rightarrow$ G\}}
    \item \textbf{R(A, B, C, D, E, F)} con \textbf{F = \{AB $\rightarrow$ C, BC $\rightarrow$ AD, D $\rightarrow$ E, CF $\rightarrow$ B\}}
\end{enumerate}

\begin{itemize}
    \item Indica \textbf{alguna llave candidata} para la relación \textbf{R}. \vspace{.3cm}

    \begin{enumerate}[label=\alph*.]
        \item \{AB\}+ = \{ABCFDEG\} es llave candidata pues cumple con \textbf{identificación única} por tener a todos los atributos de \textbf{R} y \textbf{no redundancia.} pues si eliminamos a cualquiera de sus atributos, no se cumple con la identificación única.  \vspace{.2cm}
        
        \item \{CFA\}+ = \{CFABDE\} notemos que aqui tenemos que incluir a CF pues la unica manera de agregarlo es directamente (no existe DF con CF a la derecha) y no podemos quitar a ninguno de los atributos de la llave candidata pues no se cumple con la identificación única. \vspace{.2cm}
    \end{enumerate}
    \vspace{.3cm}

    \item Especifica \textbf{todas las violaciones} a la \textbf{BCNF}. \vspace{.3cm}
    
    \begin{enumerate}[label=\alph*.]
        \item \vspace{.2cm}
        \item \vspace{.2cm}
    \end{enumerate}
    \vspace{.3cm}

    \item \textbf{Normaliza} de acuerdo con \textbf{BCNF}, asegúrate de indicar cuáles son las \textbf{relaciones resultantes} con sus respectivas \textbf{dependencias funcionales}. \vspace{.3cm}
    
    \begin{enumerate}[label=\alph*.]
        \item \vspace{.2cm}
        \item \vspace{.2cm}
    \end{enumerate}
    \vspace{.3cm}
    

\end{itemize}