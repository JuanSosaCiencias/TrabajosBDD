Para cada uno de los \textbf{esquemas} que se muestran a continuación, con su respectivo \textbf{conjunto de dependencias funcionales}:

\begin{enumerate}[label=\alph*.]
    \item \textbf{R(A, B, C, D, E, F, G)} con \textbf{F = \{AB $\rightarrow$ C, AB $\rightarrow$ F, A $\rightarrow$ D, A $\rightarrow$ E, B $\rightarrow$ G\}}
    \item \textbf{R(A, B, C, D, E, F)} con \textbf{F = \{AB $\rightarrow$ C, BC $\rightarrow$ AD, D $\rightarrow$ E, CF $\rightarrow$ B\}}
\end{enumerate}

\begin{itemize}
    \item Indica \textbf{alguna llave candidata} para la relación \textbf{R}. \vspace{.3cm}

    \begin{enumerate}[label=\alph*.]
        \item \{AB\}+ = \{ABCFDEG\} es llave candidata pues cumple con \textbf{identificación única} por tener a todos los atributos de \textbf{R} y \textbf{no redundancia.} pues si eliminamos a cualquiera de sus atributos, no se cumple con la identificación única.  \vspace{.2cm}
        
        \item \{CF\}+ = \{CFABDE\} notemos que aqui tenemos que incluir a F pues la unica manera de agregarlo es directamente (no existe DF con F a la derecha) y no podemos quitar a ninguno de los atributos de la llave candidata pues no se cumple con la identificación única. \vspace{.2cm}
    \end{enumerate}
    \vspace{.3cm}

    \item Especifica \textbf{todas las violaciones} a la \textbf{BCNF}. \vspace{.3cm}
    
    \begin{enumerate}[label=\alph*.]
        \item 
        Calculamos la cerraduras de los lados izquierdos: (ninguna es trivial)
        \begin{align*}
            \{AB\}+ &= \{ABCFDEG\} \ \text{Es llave} \\
            \{AB\}+ &= \{ABCFDEG\} \ \text{Es llave} \\
            \{A\}+ &= \{ADE\} \ \text{Violación a BCNF, la elegimos para normalizar } \\
            \{A\}+ &= \{ADE\} \ \text{Violación a BCNF} \\
            \{B\}+ &= \{BG\} \ \text{Violación a BCNF} 
        \end{align*}

        \item 
        Calculamos la cerraduras de los lados izquierdos: (ninguna es trivial)
        \begin{align*}
            \{AB\}+ &= \{ABCDE\} \ \text{Violación a BCNF, la elegimos para normalizar} \\
            \{BC\}+ &= \{BCADE\} \ \text{Violación a BCNF} \\
            \{D\}+ &= \{DE\} \ \text{Violación a BCNF} \\
            \{CF\}+ &= \{CFBADE\} \ \text{Es llave}
        \end{align*}
    \end{enumerate}
    \vspace{.3cm}

    \item \textbf{Normaliza} de acuerdo con \textbf{BCNF}, asegúrate de indicar cuáles son las \textbf{relaciones resultantes} con sus respectivas \textbf{dependencias funcionales}. \vspace{.3cm}
    
    \begin{enumerate}[label=\alph*.]
        \item 
        Elegimos la primera violación y dividimos \textbf{R}:

        \begin{align*}
            &R_1 (A,D,E) \text{ con } F = \{ A \rightarrow D, A \rightarrow E \} ; \{A\}+ = \{ADE\} \xrightarrow{} \text{A es llave para } R_1 \\ % Lado derecho de la violación
            &R_2 (A,B,C,F,G) \text{ con } F = \{AB \rightarrow C, AB \rightarrow F, B \rightarrow G\} % lado izquierdo de la violación
        \end{align*}

        En $R_1$ no hay violación a BCNF, revisamos $R_2$:
        \begin{align*}
            \{AB\}+ &= \{ABCFG\} \ \text{Es llave para } R_2 \\
            \{B\}+ &= \{BG\} \ \text{Violación a BCNF, la elegimos para normalizar}
        \end{align*}
        Divido $R_2$:
        \begin{align*}
            &R_3 = (B,G) \text{ con } F = \{B \rightarrow G\} ; \{B\}+ = \{BG\} \xrightarrow{} \text{A es llave para } R_3\\
            &R_4 = (B,A,C,F) \text{ con } F = \{AB \rightarrow C, AB \rightarrow F\} ; \{AB\}+ = \{ABCF\} \xrightarrow{} \text{A es llave para } R_4
        \end{align*}

        Finalmente: 
        \begin{align*}
            R_1 (A,D,E) \text{ con } F &= \{ A \rightarrow D, A \rightarrow E \} \\
            R_3 (B,G) \text{ con } F &= \{B \rightarrow G\} \\
            R_4 (B,A,C,F) \text{ con } F &= \{AB \rightarrow C, AB \rightarrow F\}
        \end{align*}
        \vspace{.2cm}
        
        \item 
        Elegimos la primera violación y dividimos \textbf{R}:
        \begin{align*}
            R_1 (A,B,C,D,E) \text{ con } F &= \{AB \rightarrow C, BC \rightarrow AD, D \rightarrow E\} ; \\
            R_2 (A,B,F) \text{ con } F &= \{ABF \rightarrow ABF \} ; \xrightarrow{} \text{ABF es llave para } R_2 \\
        \end{align*}

        Aqui observamos 2 cosas, la primera es que $R_2$ al no cumplir ninguna dependencia funcional, solo se tiene la trivial y trivialmente es llave, ademas de esto, vemos que perdemos la DF $CF \rightarrow B$ por lo que deberiamos parar la normalización en $R_2$, pero como el profesor hizo vamos a seguir por fines didacticos. Revisamos $R_1$:

        \begin{align*}
            \{AB\}+ &= \{ABCDE\} \ \text{Es llave para } R_1 \\
            \{BC\}+ &= \{BCADE\} \ \text{Es llave para } R_1 \\
            \{D\}+ &= \{DE\} \ \text{Violación a BCNF, la elegimos para normalizar}
        \end{align*}

        Dividimos $R_1$:
        \begin{align*}
            R_3 (D,E) \text{ con } F &= \{D \rightarrow E\} ; \{D\}+ = \{DE\} \xrightarrow{} \text{D es llave para } R_3 \\
            R_4 (D,A,B,C) \text{ con } F &= \{AB \rightarrow C, BC \rightarrow AD\} ; 
        \end{align*}

        Revisamos $R_4$:
        \begin{align*}
            \{AB\}+ &= \{ABCD\} \ \text{Es llave para } R_4 \\
            \{BC\}+ &= \{BCAD\} \ \text{Es llave para } R_4 \\
        \end{align*}

        Finalmente:
        \begin{align*}
            R_2 (A,B,F) \text{ con } F &= \{ABF \rightarrow ABF \} \\
            R_3 (D,E) \text{ con } F &= \{D \rightarrow E\} \\
            R_4 (D,A,B,C) \text{ con } F &= \{AB \rightarrow C, BC \rightarrow AD\}
        \end{align*}

        Como nota importante, se presenta join con perdida, DF perdidas: $CF \rightarrow B$ 
        \vspace{.2cm}
    \end{enumerate}
    \vspace{.3cm}
    

\end{itemize}