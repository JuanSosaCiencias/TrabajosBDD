Para cada uno de los esquemas que se muestran a continuación, con su respectivo conjunto de dependencias funcionales:

\begin{itemize}
	\item[a)] \( R(A, B, C, D) \) con \( F = \{AB \twoheadrightarrow C, B \to D\} \).
	\item[b)] \( R(A, B, C, D, E) \) con \( F = \{A \twoheadrightarrow B, AB \to C, C \to D, D \to E\} \).
\end{itemize}

\begin{enumerate}
	\item Encuentra todas las violaciones a la BCNF.
	\item Normaliza de acuerdo con la BCNF.
\end{enumerate}

Para cada uno de los esquemas presentados, aplicaremos la cuarta forma normal sigiuendo estos pasos:
1. Identificamos la llave candidata.\\
2. Determinamos las violaciones a la 4NF.\\
3. Descomponemos la relación para eliminar dichas violaciones.

\subsubsection*{Esquema (a): \( R(A, B, C, D) \)}

\textbf{Paso 1: Determinar la llave candidata}

Las dependencias funcionales dadas son:
\[
AB \rightarrow C \quad \text{y} \quad B \rightarrow D
\]
Para determinar la llave candidata, observamos que \( AB \) es suficiente para determinar \( C \) y, dado \( B \rightarrow D \), concluimos que \( AB \) es la llave candidata para esta relación.

\textbf{Paso 2: Identificar violaciones a la 4NF}

La 4NF exige que cualquier dependencia multivaluada o funcional no trivial tenga como determinante una superllave. En este caso:
\begin{itemize}
	\item La dependencia funcional \( B \rightarrow D \) viola la 4NF porque \( B \) no es una superllave.
\end{itemize}

\textbf{Paso 3: Descomposición en 4NF}

Para normalizar esta relación, descomponemos \( R(A, B, C, D) \) en dos relaciones, de modo que cada dependencia funcional se maneje por separado:
\begin{align*}
	R_1(A, B, C) & : AB \rightarrow C \\
	R_2(B, D) & : B \rightarrow D
\end{align*}
En \( R_1 \), \( AB \) es la clave primaria, mientras que en \( R_2 \), la clave primaria es \( B \). Ambas relaciones ahora cumplen con la 4NF.

\textbf{Relaciones finales para el esquema (a):}
\begin{itemize}
	\item \( R_1(A, B, C) \): relación donde \( AB \) es la clave primaria y determina \( C \).
	\item \( R_2(B, D) \): relación donde \( B \) es la clave primaria y determina \( D \).
\end{itemize}

\subsubsection*{Esquema (b): \( R(A, B, C, D, E) \)}

\textbf{Paso 1: Determinar la llave candidata}

Las dependencias funcionales son:
\[
A \rightarrow B, \quad AB \rightarrow C, \quad A \rightarrow D, \quad AB \rightarrow E
\]
Dado que \( AB \rightarrow C \) y \( AB \rightarrow E \), concluimos que \( AB \) es una superllave para esta relación.

\textbf{Paso 2: Identificar violaciones a la 4NF}

Verificamos las dependencias para encontrar violaciones a la 4NF:
\begin{itemize}
	\item La dependencia \( A \rightarrow B \) es una violación a la 4NF porque \( A \) no es una superllave.
	\item La dependencia \( A \rightarrow D \) también viola la 4NF por la misma razón.
\end{itemize}

\textbf{Paso 3: Descomposición en 4NF}

Para normalizar el esquema, descomponemos la relación en varias sub-relaciones:
\begin{align*}
	R_1(A, B) & : A \rightarrow B \\
	R_2(A, D) & : A \rightarrow D \\
	R_3(A, B, C, E) & : AB \rightarrow C \quad \text{y} \quad AB \rightarrow E
\end{align*}
En \( R_1 \), \( A \) es la clave primaria y determina \( B \). En \( R_2 \), \( A \) es la clave primaria y determina \( D \). En \( R_3 \), \( AB \) es la clave primaria y determina \( C \) y \( E \). Todas estas relaciones cumplen con la 4NF.

\textbf{Relaciones finales para el esquema (b):}
\begin{itemize}
	\item \( R_1(A, B) \): relación donde \( A \) es la clave primaria y determina \( B \).
	\item \( R_2(A, D) \): relación donde \( A \) es la clave primaria y determina \( D \).
	\item \( R_3(A, B, C, E) \): relación donde \( AB \) es la clave primaria y determina \( C \) y \( E \).
\end{itemize}