\begin{center}
    \textbf{¿Cuál es el significado de un tipo de relación recursiva? Proporciona un par de ejemplos de este tipo de relación.}
\end{center}

\vspace{.3cm}

Un tipo de relación recursiva es una relación donde una entidad está relacionada consigo misma. Es decir, es cuando tenemos un conjunto de objetos del mismo tipo que pueden tener alguna relación entre ellos. Esto es útil cuando queremos modelar situaciones en las que un elemento necesita estar vinculado a otro del mismo tipo. \cite{lucidchart-2023} \\

Por ejemplo, imaginemos que en una empresa tenemos empleados y algunos empleados supervisan a otros empleados. Aquí, la entidad es \textit{Empleado}, y la relación recursiva sería \textit{supervisa}. Esta relación nos diría que un empleado puede supervisar a uno o más empleados, y a su vez, un empleado puede ser supervisado por uno o más empleados. La relación “supervisa” es recursiva porque conecta empleados con empleados. \\

Otro ejemplo sería en una biblioteca, donde un libro puede estar compuesto de varios capítulos, y a su vez, cada capítulo puede estar compuesto de subcapítulos. Aquí, la entidad es \textit{Capítulo}, y la relación recursiva sería \textit{compuesto de}. Esta relación nos dice que un capítulo puede contener otros capítulos, permitiendo representar una estructura jerárquica de contenidos dentro del mismo tipo de entidad. \\