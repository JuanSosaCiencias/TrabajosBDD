\begin{center} \textbf{¿Qué diferencia existe entre los diagramas de las figuras a) y c)?} \end{center}

\vspace{.3cm}

Veamos que la diferencia principal entre los diagramas de las figuras a) y c) radica en cómo representan las relaciones entre las entidades \textit{Materia}, \textit{Profesor} y \textit{Alumno}. \\

En la figura a), se utiliza una sola relación llamada \textit{"Tener"} para conectar las tres entidades. Esto significa que tanto los \textit{Profesores} como los \textit{Alumnos} están vinculados a las \textit{Materias} mediante la misma relación, sin especificar claramente cómo es esa relación. Por ejemplo, no nos dice si el \textit{Profesor} está enseñando la materia o si el \textit{Alumno} está inscrito en ella; simplemente dice que todos tienen una conexión con la materia. Se podría interpretar como que los alumnos tienen muchas materias inscritas y los profesores tienen muchas materias asignadas o que las materias tienen muchos alumnos y profesores. Sin embargo esto igualmente resulta ambiguo. \\

Por otro lado, en la figura c), se usan dos relaciones distintas: \textit{"Tener"} y \textit{"Manejar"}. Aquí, \textit{"Tener"} conecta a los \textit{Alumnos} con las \textit{Materias}, lo que podría significar que los \textit{Alumnos} están inscritos en esas \textit{Materias}. La relación \textit{"Manejar"} conecta a los \textit{Profesores} con la relacion binaria antes mencionada, siendo de esta manera mas específica, lo cual por ejemplo nos podría indicar que los \textit{Profesores} están a cargo o enseñan esas materias a los alumnos. Este enfoque es más claro porque muestra específicamente cómo se relaciona cada entidad. Ademas del hecho de señalar o remarcar la relacion entre materia y alumno dentro de un cuadro hace que podamos tomar esa relacion binaria como una entidad en si misma para conectarla con profesor mediante la relacion Manejar. \\

Entonces podemos decir que la figura a) es más simple, pero menos clara y la figura c) es más específica, ya que usa dos relaciones diferentes para mostrar claramente cómo los \textit{Profesores} y los \textit{Alumnos} se relacionan con las \textit{Materias}. \\