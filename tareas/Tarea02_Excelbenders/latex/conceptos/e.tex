Explica el concepto de categorías (herencia múltiple) en el modelo E-R y proporciona dos ejemplos de la vida real en donde se aplique este concepto.

Las \textbf{categorías} (o herencia múltiple) en el modelo E-R permite que una entidad pueda heredar sus atributos a otras entidades, donde la entidad hija \textbf{(subtipo)} hereda los atributos y relaciones de una o múltiples entidades padres \textbf{(supertipo)}. Esto significa que la entidad hija se relaciona con varias entidades padres, combinando sus características.

Ejemplos:
\begin{itemize}
    \item \textbf{Ejemplo 1:}
    \begin{itemize}
        \item \textbf{Supertipo:} Empleado y Consultor
        \item \textbf{Subtipo:} Empleado-Consultor
        
    \begin{itemize}
        \item La entidad \textbf{Empleado} contiene atributos como ``Número de empleado'', ``Salario'', ``Departamento''.
        \item La entidad \textbf{Consultor} tiene atributos como ``Número de proyecto'', ``Tarifa por hora'', ``Fecha de finalización''.
        \item La entidad \textbf{Empleado-Consultor} hereda atributos tanto de \textbf{Empleado} como de \textbf{Consultor}, y puede tener atributos adicionales como ``Horas dedicadas como consultor''. Esta entidad representa a un trabajador que es tanto un empleado fijo de una empresa como un consultor en proyectos específicos.
    \end{itemize}
    \end{itemize}

    \item \textbf{Ejemplo 2:}
    \begin{itemize}
         \item \textbf{Supertipo:} Persona
        \item \textbf{Subtipo:} Estudiante y Profesor
        
    \begin{itemize}
        \item La entidad \textbf{Persona} contiene atributos comunes como ``Nombre'', ``Edad'' y ``Dirección''.
        \item Las entidades hijas \textbf{Estudiante} y \textbf{Profesor} heredan esos atributos comunes, pero tienen atributos adicionales específicos:
        \begin{itemize}
            \item \textbf{Estudiante:} ``Número de matrícula'', ``Carrera''.
            \item \textbf{Profesor:} ``Número de empleado'', ``Departamento''.
        \end{itemize}
    \end{itemize}
    \end{itemize}
\end{itemize}

