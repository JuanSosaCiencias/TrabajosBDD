\begin{center}
    \textbf{¿Qué es un tipo de relación? Explica las diferencias con respecto a una instancia 
    de relación.}
\end{center}

\vspace{.3cm}

Un tipo de relación es una abstraccion que describe la relacion entre 2 o mas entidades, existe
en el modelo de conceptual de datos; es decir, es una definición general que nos dice la cardinalidad
y la participación de las entidades en la relación. Por otro lado, una instancia de relación es 
una ocurrencia de la relación, es decir, es un conjunto de tuplas que cumplen con las 
restricciones del tipo de relación.\\

Por ejemplo, si tenemos un tipo de relación \textit{Trabaja} entre las entidades \textit{Empleado} 
y \textit{Departamento}, el tipo de relación nos dirá que un empleado puede trabajar en 1 o mas
departamentos y que un departamento puede tener 1 o mas empleados. Por otro lado, una instancia de
relación sería una tabla que nos dice que el empleado \textit{Juan} trabaja en el departamento
\textit{Ventas} y el empleado \textit{Pedro} trabaja en el departamento de \textit{Computo}.\\