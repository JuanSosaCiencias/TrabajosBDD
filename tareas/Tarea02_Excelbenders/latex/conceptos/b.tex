¿En qué condiciones se puede migrar un atributo de algún tipo de entidad que participa en un tipo de relación binaria y convertirse en un atributo del tipo de relación? ¿Cuál sería en el efecto?

   Las \textbf{condiciones} en que se puede realizar esto es cuando la relación tiene una cardinalidad de muchos a muchos. Ademas el atributo a migrar debe tener sentido en el contexto de la relación y no solo de la entidad individual, y debe depender de la ocurrencia específica de la relación, no solo de una de las entidades participantes. 
   
   Una buena condicion para migrar es si el atributo es multivaluado para una misma instancia de la relación.

   Por otro lado, los \textbf{efectos} que tendría puede ser que, como movemos el atributo a la relación, se está indicando que este atributo es una propiedad de la interacción entre las dos entidades, no de una entidad individual, lo cual nos permite asociar diferentes valores del atributo a diferentes instancias de la relación entre las mismas entidades, esto nos permitiría mejorar la integridad de los datos al asociar el atributo directamente con la relación que lo determina y evitar redundancias. \\