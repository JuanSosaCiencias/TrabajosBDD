\documentclass{report}
\usepackage[utf8]{inputenc}

%----- Configuración del estilo del documento------%
\usepackage{epsfig,graphicx}
\usepackage[left=2.5cm,right=2.5cm,top=1.8cm,bottom=2.3cm]{geometry}
%------ Paquetes matematicos --------%
\usepackage{amsmath}
\usepackage{amssymb}
\usepackage{amsthm}
\usepackage{amsmath}
\usepackage{tabularx}
\usepackage{fancyhdr}
\usepackage{lastpage}
\usepackage{verbatim}
\usepackage[shortlabels]{enumitem}
\usepackage{venndiagram}
\usetikzlibrary{shapes.geometric}
\usepackage{cancel}
\usepackage{hyperref}
\usepackage[T1]{fontenc}
\usepackage[spanish,es-nodecimaldot,es-tabla]{babel}
\usepackage{csquotes}
\usepackage{graphicx}
\usepackage{tocloft}
\graphicspath{{./figs/}}
\usepackage{setspace}
\usepackage{xcolor}


\usepackage[backend=biber]{biblatex}
\addbibresource{resources/referencias/referencias.bib}




\begin{document}
	
	\begin{titlepage}
	\thispagestyle{empty}
	\begin{minipage}[c][0.17\textheight][c]{0.25\textwidth}
		\begin{center}
			\includegraphics[width=3.5cm, height=3.5cm]{resources/Logo_UNAM.png}
		\end{center}
	\end{minipage}
	\begin{minipage}[c][0.195\textheight][t]{0.75\textwidth}
		\begin{center}
			\vspace{0.3cm}
			\textsc{\large Universidad Nacional Aut\'onoma de M\'exico}\\[0.5cm]
			\vspace{0.3cm}
			\hrule height2.5pt
			\vspace{.2cm}
			\hrule height1pt
			\vspace{.8cm}
			\textsc{Facultad de Ciencias}\\[0.5cm] %
		\end{center}
	\end{minipage}
	
	\begin{minipage}[c][0.81\textheight][t]{0.25\textwidth}
		\vspace*{5mm}
		\begin{center}
			\hskip2.0mm
			\vrule width1pt height13cm 
			\vspace{5mm}
			\hskip2pt
			\vrule width2.5pt height13cm
			\hskip2mm
			\vrule width1pt height13cm \\
			\vspace{5mm}
			\includegraphics[height=4.0cm]{resources/Logo_FC.png}
		\end{center}
	\end{minipage}
	\begin{minipage}[c][0.81\textheight][t]{0.75\textwidth}
		\begin{center}
			\vspace{1cm}
			
			{\large\scshape Fundamentos de Bases de Datos - 7094}\\[.2in]
			
			\vspace{2cm}            
			
			\textsc{\LARGE \textbf{T}\hspace{1cm}\textbf{A}\hspace{1cm}\textbf{R}\hspace{1cm}\textbf{E}\hspace{1cm}\textbf{A}\hspace{1.3cm}\textbf{3}}\\[2cm]
			\textsc{\Large{Equipo:}\normalsize \\
                \vspace{.3cm}
				\textbf{Del Monte Ortega Maryam Michelle - 320083527 \\
                \vspace{.2cm}
				\href{https://github.com/JuanSosaCiencias}{\textcolor{blue}{Sosa Romo Juan Mario - 320051926}} \\
                \vspace{.2cm}
				Castillo Hernández Antonio - 320017438 \\
                \vspace{.2cm}
                Erik Eduardo Gómez López - 320258211 \\
                \vspace{.2cm}
                Julio César Islas Espino - 320340594}}\\[0.5cm]     
			
			\textsc{{Fecha de entrega: \\ \textbf{24 de Septiembre de 2024}}}\\[0.5cm]        
			
			\textsc{{Profesor: \\ \textbf{M. en I. Gerardo Avilés Rosas}}}\\[0.5cm]  
			
			\textsc{Ayudantes: \\ \textbf{Luis Enrique García Gómez \\ Kevin Jair Torres Valencia \\ Ricardo Badillo Macías \\ Rocío Aylin Huerta González
			} }
			
			
			\vspace{0.5cm}
		\end{center}
	\end{minipage}
\end{titlepage}

    \cite{martins2024}
	
	\begin{center}
		\section*{\LARGE{Tarea 1}}
	\end{center}

        \LARGE{\textbf{Analisis de requerimientos:}}\\
        \normalsize
        \begin{enumerate}
            \item 
\begin{center}
    \includegraphics[width = 17 cm]{1.jpg}
    \includegraphics[width = 17 cm]{2.jpg}
    \includegraphics[width = 17 cm]{3.jpg}
    \includegraphics[width = 17 cm]{4.jpg}
\end{center}

        \end{enumerate}

    % Preguntas  
    \begin{center}
        \LARGE{\textbf{Preguntas}}\\
    \end{center}
    \normalsize
    \begin{enumerate}%[label=\alph*.]
        \item \begin{center}
    \textbf{¿Qué es una política de mantenimiento de llaves foráneas?}   
\end{center}

Bien sabemos que en PostgreSQL, una \textbf{llave foránea} es una restricción que se utiliza para garantizar la integridad entre dos tablas. Entonces, en pocas palabras, en una tabla se referencia el campo clave primaria en otra tabla. \\

Son reglas que definen qué sucede con los registros relacionados entre dos tablas, a través de llaves foráneas, cuando se hacen cambios en la tabla que contiene la clave primaria. \\

Nos sirven para que exista coherencia entre los datos de ambas tablas cuando se eliminan o actualizan registros. \\

Las principales políticas son:

\begin{itemize}
    \item \textbf{ON DELETE}: Esta política especifica qué sucede cuando se elimina un registro de la tabla referenciada (la que contiene la clave primaria).
    \item \textbf{ON UPDATE}: Esta política especifica qué sucede cuando se actualiza un valor de clave primaria en la tabla referenciada.
\end{itemize}

PostgreSQL define diferentes políticas que pueden configurarse cuando
se crean las restricciones de llaves foráneas, las cuales determinan
cómo se gestionan los cambios en los registros referenciados. \\

Las opciones disponibles para ambas políticas son:

\begin{itemize}
    \item \textbf{CASCADE}: Si se elimina o actualiza un registro en la tabla referenciada, se eliminarán o actualizarán automáticamente los registros correspondientes en la tabla que contiene la llave foránea.
    \item \textbf{SET NULL}: Si se elimina o actualiza un registro en la tabla referenciada, los campos de la llave foránea en la tabla dependiente se establecerán en \texttt{NULL}.
    \item \textbf{SET DEFAULT}: Si se elimina o actualiza un registro en la tabla referenciada, los campos de la llave foránea en la tabla dependiente se establecerán en un valor por defecto predefinido.
    \item \textbf{RESTRICT}: Impide la eliminación o actualización del registro referenciado si hay registros dependientes en la tabla que contiene la llave foránea.
    \item \textbf{NO ACTION}: Similar a RESTRICT, pero la validación ocurre al final de la transacción. PostgreSQL no permite que se complete la transacción si la llave foránea se ve comprometida.
\end{itemize}

Estas políticas permiten mantener la consistencia e integridad de los datos en una base de datos relacional, asegurando que no haya referencias huérfanas o inconsistentes entre las tablas. \\

        \item \begin{center}
    \textbf{Describe cual es el mas conveniente utilizar.}
    \vspace{.5cm}
\end{center}

Depende de la complejidad de los datos y las necesidades del proyecto.

Un \textbf{Sistema de Datos} es adecuado cuando se requiere una estructura simple y poco procesamiento. Si solo necesitas almacenar archivos en carpetas y no hay relaciones complejas entre los datos, un sistema de archivos es una opción eficiente. Además, es más fácil de implementar y mantener en proyectos pequeños o específicos.

En cambio, una \textbf{Base de Datos} es preferible cuando se necesita manipular grandes volúmenes de datos estructurados, realizar consultas avanzadas, o cuando se requieren funcionalidades como seguridad, control de acceso, integridad referencial o escalabilidad. Las bases de datos permiten automatizar procesos complejos, mejorar la productividad y reducir errores, facilitando la toma de decisiones en empresas que manejan datos críticos y en constante evolución.\\

\cite{sooluciona}
\vspace{.5cm}

    \end{enumerate}
    \newpage
    
\printbibliography
  
\end{document}