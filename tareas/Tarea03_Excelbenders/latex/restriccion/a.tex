\textbf{Dos compañías con el nombre ‘Panaphonics’ podrían existir al mismo tiempo.}\vspace{.3cm}

\begin{quote}
    Voy a asumir que la relación \textbf{Proyecto} es donde se pondría el nombre de la compañía (que no se que tanto sentido tenga hacer esto para el departamento de RH de una empresa pero bueno). Vemos que en esta relación el único atributo que no se puede repetir es \textbf{NumProyecto} pues es la clave primaria. Por lo tanto, si se puede tener dos compañías con el nombre ‘Panaphonics’ al mismo tiempo. \vspace{.2cm}

    Si se quisiera que solo existiera una compañía con el nombre ‘Panaphonics’ al mismo tiempo, se podría definir el atributo \textbf{NombreCompañía} como clave primaria o una llave primaria compuesta con el atributo \textbf{NumProyecto}.
\end{quote}
\vspace{.3cm}