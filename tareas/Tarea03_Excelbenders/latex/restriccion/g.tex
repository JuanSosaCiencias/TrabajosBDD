\textbf{Los empleados y/o directores deben vivir en la misma Ciudad que la Compañía para la que laboran/dirigen.}\vspace{.3cm} \\

Para que ePara que esta afirmación fuera verdadera, debería existir una relación entre la ciudad donde reside un empleado o director y la ubicación del proyecto o compañía para la que trabajan o dirigen. Sin embargo, dado que no hay una tabla llamada \textit{compañía}, podemos suponer que esta información se encuentra en las otras tablas como \texttt{proyecto}, \texttt{asignar}, o \texttt{dirigir}. Con esto en mente, podemos analizar dos situaciones: \\

\textit{Caso 1: Si hay un campo Ciudad tanto en \texttt{proyecto}, \texttt{asignar} o \texttt{dirigir}, como en \texttt{empleado}:} \\

Si alguna de las tablas relacionadas con proyectos (\texttt{proyecto}, \texttt{asignar}, o \texttt{dirigir}) incluye un campo para la ubicación del proyecto, y la tabla \texttt{empleado} también tiene un campo Ciudad (que sabemos que tiene), podríamos crear una restricción que asegure que las ciudades coincidan. Si esto se cumple, la afirmación sería válida. \\

\textit{Caso 2: Si solo la tabla \texttt{empleado} tiene un campo Ciudad:} \\

Si únicamente la tabla \texttt{empleado} incluye un campo Ciudad y ninguna otra tabla contiene información sobre la ubicación de los proyectos o compañías, entonces no habría forma de verificar si las ciudades coinciden. En este caso, la afirmación no sería válida, ya que faltaría la información necesaria para hacer la comparación. \\

En conclusión, esta afirmación no se cumple de manera automática, ya que no hay una relación directa que conecte la ciudad del empleado o director con la ubicación de la compañía o proyecto. Para que la afirmación sea verdadera, sería necesario agregar un campo de ubicación en las tablas que gestionan proyectos y establecer una relación con los empleados. \\

$\therefore$ \text{La afirmación no se cumple en el modelo relacional actual.}\\
