\textbf{Los empleados y/o directores deben vivir en la misma Ciudad que la Compañía para la que laboran/dirigen.}\vspace{.3cm} \\

Para que esta afirmación sea verdadera, debería haber algún tipo de relación entre la ciudad en la que vive un empleado o director y la ciudad donde está ubicada la compañía para la cual trabajan o dirigen, pero dado que no existe una tabla llamada \textit{compañía}, podríamos asumir que dicha información, se almacena en alguna de las otras tablas. Por lo cual se pueden considerar los siguientes dos casos: \\

\textit{Caso 1: Si hay un campo ciuada en ambas tablas:}

Si alguna de las tablas \texttt{proyecto}, \texttt{asignar} o \texttt{dirigir} contiene un campo que indique la ubicación de la compañía o proyecto, y la tabla \texttt{empleado} incluye un campo Ciudad \textit{(como efectivamente lo tiene)}, entonces podríamos establecer una restricción de integridad referencial para asegurar que los valores de Ciudad coincidan. De lo contrario, la afirmación no sería válida automáticamente.\\

\textit{Caso 2: Si solo hay un campo Ciudad en la tabla de Empleado:} \\

Si solo hay un campo Ciudad en la tabla \texttt{empleado}, y ninguna otra tabla contiene información sobre la ubicación de la compañía o proyecto, no habría forma de comprobar la afirmación, ya que no habría un valor de ciudad en el lado de la empresa o proyecto. En este caso, la afirmación no podría ser validada. \\


Entonces analizando estos casos, se puede decir que esta afirmación no se cumple por sí sola, ya que no hay una relación directa entre la ciudad de la empresa y la ciudad de los empleados o directores. Para que esta afirmación sea verdadera, se necesitaría una tabla adicional que contenga información sobre la ubicación de la empresa o proyecto, y que esta información esté relacionada con las tablas de empleados y directores. \\

$\therefore$ \text{La afirmación no se cumple para el modelo relacional actual.} \\
