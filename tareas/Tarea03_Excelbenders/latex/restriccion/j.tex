\textbf{‘Mr. Plow’ no requiere tener definido algún empleado que la dirija.}\vspace{.3cm}

Esta afirmación depende de cómo se haya modelado la relación entre la entidad \texttt{Compañía} y la entidad \texttt{Director}. \\

\textit{Caso 1: Participación parcial} \\

Si la relación entre \texttt{Compañía} y \texttt{Director} es de participación parcial, es posible que \texttt{Mr. Plow} no tenga un director definido. En este caso, el campo \texttt{id\_director} en la tabla de compañías puede ser \texttt{NULL}, por lo que la afirmación se cumple.

\textit{Caso 2: Participación total} \\

Si la relación es de participación total, entonces todas las compañías deben tener un director, por lo que \texttt{Mr. Plow} debería tener uno asignado. En este caso, la afirmación no se cumple. \\


Por lo tanto veamos que la afirmación se cumple si el modelo relacional permite que algunas compañías existan sin tener un director asignado.