\textbf{Algunas tuplas en Trabaja podrían no tener valor para el atributo desde y ningún empleado asociado a ellas.}\vspace{.3cm}

\section{Afirmación i)}

La afirmación sugiere que en la relación \texttt{Trabaja} (que podría corresponder a la tabla \texttt{Asignar} en el modelo relacional), algunas tuplas podrían tener los siguientes problemas:
\begin{itemize}
    \item No contener un valor definido para el atributo \texttt{desde}.
    \item No tener asociado un empleado a través de la \textbf{clave foránea} correspondiente.
\end{itemize}

Analizemos ambos:


\subsection{Falta de valor en el atributo \texttt{desde}}

En el modelo relacional que se describe, se utiliza el período de asignación $[\texttt{desde}, \texttt{hasta})$ para representar el tiempo durante el cual un empleado está asignado a un proyecto. \\

El valor \texttt{desde} representa el \textbf{inicio} de dicha asignación, y según el enunciado, es \textbf{inclusivo}, lo que significa que el empleado comienza su trabajo en el proyecto a partir de esa fecha. \\

Si alguna tupla no tuviera un valor en el campo \texttt{desde}, no sería posible definir cuándo comenzó la asignación de un empleado a un proyecto. Esto genera un problema en términos de \textbf{consistencia de los datos}, ya que una asignación sin una fecha de inicio es \textbf{incompleta} y difícil de interpretar. \\

En un modelo relacional bien diseñado, el atributo \texttt{desde} debe ser \textbf{obligatorio}, ya que es un clave para definir el período de asignación. Permitir que este atributo sea nulo implicaría una \textbf{inconsistencia en tiempos} en el registro de las asignaciones.


\subsection{Ausencia de empleado relacionado a la tupla}

El campo \texttt{IdPersona} en la relación \texttt{Trabaja} (o \texttt{Asignar}) es una \textbf{clave foránea} que se refiere a la tabla \texttt{Empleado}. Esto significa que cada registro en \texttt{Trabaja} debería estar asociado a un empleado a través de esta \textbf{clave foránea}. \\

Si no existe un valor para \texttt{IdPersona} en alguna tupla, entonces dicha asignación no estaría asociada a ningún empleado, lo que \textbf{violaría la integridad referencial} del modelo. En este sistema de gestión de recursos humanos, no tendría sentido que existiera una asignación de trabajo sin un empleado asignado, sería algo ilógico. \\

En un \textbf{modelo relacional correcto}, este tipo de relaciones se define para garantizar que las tuplas de la tabla \texttt{Trabaja} siempre estén asociadas a un empleado válido. Es decir, no debería permitirse que el campo \texttt{IdPersona} sea nulo, ya que esto rompería la relación entre la tabla \texttt{Empleado} y \texttt{Trabaja}, y resultaría en una asignación \textbf{inválida}. \\
 


Para terminar, se concluye que la afirmación i) \textbf{no se cumple}. Permitir tuplas en la relación \texttt{Trabaja} sin un valor definido para el atributo \texttt{desde} o sin un empleado asociado violaría las reglas de \textbf{integridad} y \textbf{consistencia} de un modelo relacional.

\begin{itemize}
    \item El valor \texttt{desde} es un atributo esencial para definir el inicio del período de trabajo, por lo que no debe permitirse que sea nulo.
    \item El campo \texttt{IdPersona} es una \textbf{clave foránea} que asegura que cada asignación esté relacionada con un empleado. Su ausencia implicaría un error de \textbf{integridad referencial}.
\end{itemize}

\textbf {En conclusión, la afirmación i) no se ajusta a las reglas lógicas de un modelo relacional consistente.}
