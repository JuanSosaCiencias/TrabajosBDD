\textbf{Considera el mismo escenario del inciso b para las relaciones A y B. Toma como base el Modelo Relacional que
obtuviste para la cardinalidad 1:1. Supón que tu modelo tiene participación parcial de ambos lados. Propón un
conjunto de 4 tuplas que se pueda insertar en ab y un conjunto que no se pueda insertar (también de 4 tuplas).
Justifica tu respuesta en cada caso.}\vspace{.3cm}
\subsubsection*{Conjunto de 4 tuplas que se pueden insertar en \( AB \)}

Supongamos que las tuplas presentes en \( A \) y \( B \) son las siguientes:

\begin{itemize}
    \item \( A: (2, \text{'ww'}, \text{'a'}), (4, \text{'xx'}, \text{'b'}), (6, \text{'yy'}, \text{'c'}), (8, \text{'zz'}, \text{'d'}) \)
    \item \( B: (17, \text{'e'}), (27, \text{'f'}), (37, \text{'g'}), (47, \text{'h'}) \)
\end{itemize}

Propuestas de tuplas para \( AB \):

\begin{enumerate}
    \item (2, \text{'ww'}, 17, 10)
    \item (4, \text{'xx'}, 27, 20)
    \item (6, \text{'yy'}, 37, 30)
    \item (8, \text{'zz'}, 47, 40)
\end{enumerate}


Estas tuplas se pueden insertar porque:

\begin{itemize}
    \item Cada \( (a1, a2) \) de \( AB \) corresponde a una combinación válida de claves primarias de \( A \).
    \item Cada \( b \) en \( AB \) coincide con una clave primaria en \( B \).
    \item Las combinaciones de \( (a1, a2) \) y \( b \) son únicas, respetando la restricción de 1:1, sin repetir la asociación entre \( A \) y \( B \).
\end{itemize}

\subsubsection*{Conjunto de 4 tuplas que no se pueden insertar en \( AB \)}

\begin{enumerate}
    \item (2, \text{'ww'}, 17, 10)
    \item (2, \text{'ww'}, 27, 20)
    \item (4, \text{'xx'}, 27, 30)
    \item (4, \text{'xx'}, 47, 40)
\end{enumerate}


Estas tuplas no se pueden insertar debido a que:

\begin{itemize}
    \item Las primeras dos tuplas intentan asociar la misma tupla de \( A \) (\( 2, \text{'ww'} \)) con diferentes tuplas de \( B \) (17 y 27), violando la restricción de 1:1.
    \item Las últimas dos tuplas intentan asociar la misma tupla de \( A \) (\( 4, \text{'xx'} \)) con diferentes tuplas de \( B \) (27 y 47), lo cual también infringe la relación 1:1.
    \item La relación 1:1 implica que una tupla de \( A \) solo puede estar asociada con una única tupla de \( B \) y viceversa, y estas tuplas propuestas intentan establecer múltiples asociaciones.
\end{itemize}
