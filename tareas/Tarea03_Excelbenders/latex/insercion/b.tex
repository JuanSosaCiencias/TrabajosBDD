\textbf{Del inciso a) toma el MR que obtuviste para la cardinalidad M : N. Asume que los atributos a1, b y ab1 son de tipo
entero, mientras que a2, a3 y b1 son de tipo cadena. Supón que la relación A tiene 4 tuplas con los siguientes valores
(2,’ww’,’a’), (4,’xx’,’b’), (6,’yy’,’c’), (8,’zz’,’d’) y la relación B tiene 5 tuplas identificadas por
los valores 17, 27, 37, 47, 57. Los incisos que se presentan a continuación, representan un conjunto de tuplas
a insertar (en ese orden) en la relación AB, indica cuál conjunto se puede insertar completamente en dicha relación.
Justifica tu respuesta en cada caso.}\vspace{.3cm}

\begin{enumerate}
    \item (8,’zz’,17,5); (6,’yy’,57,10); (4,’xx’,27,15); (2,’ww’,37,20); (4,’xx’,27,15)
    
    Podemos insertar (8,’zz’,17,5); (6,’yy’,57,10); (4,’xx’,27,15); (2,’ww’,37,20) y ya, insertar \textbf{otra vez} (4,’xx’,27,15) es innecesario ya que existiría más de una tupla con la misma llave.

    \item (17,’zz’,2,’m’); (27,’yy’,4,’n’); (37,’xx’,6,’o’); (47,’ww’,8,’p’); (57,’zz’,4,’q’)
    
    Ninguna se puede insertar ya que \textbf{ab1} es de tipo entero según las especificaciones y para este conjunto de tuplas, en todas se representa a \textbf{ab1} como cadena o caracter. 

    \item (2,’a’,17,23); (4,’b’,27,24); (6,’c’,37,25); (8,’d’,47,26); (2,’a’,57,27)
    
    Todas se pueden insertar, ya que los tipos de datos están correctos y no hay tuplas duplicadas ya que todas tienen más de un atributo distinto para los identificadores. 

    \item (2,’ww’,57,’a’); (4,’xx’,37,’b’); (6,’yy’,17,’c’); (8,’zz’,37,’d’); (4,’xx’,47,’a’)
    
    Nuevamente no podemos insertar ninguna ya que \textbf{ab1} es de tipo entero y se le pasan cadenas. 
\end{enumerate}

\vspace{.5cm}
