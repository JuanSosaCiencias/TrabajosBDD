\textbf{Del inciso a) toma como base el MR que obtuviste para la cardinalidad 1 : N. Los incisos que se presentan a
continuación representan un conjunto de tuplas a insertar (en ese orden) en la relación B, indica cuál conjunto se
puede insertar completamente en dicha relación. Justifica tu respuesta en cada caso.}\vspace{.3cm}

\begin{enumerate}
    \item (2,’f’,57,’zz’); (4,’g’,47,’yy’); (6,’h’,37,’xx’); (8,’i’,27,’ww’); (2,’j’,17,’yy’)
    \item (57,8,’zz’,’f’); (47,6,’yy’,’g’); (37,4,’xx’,’h’); (27,2,’ww’,’i’); (17,6,’yy’,’j’)
    \item (57,’f’,8,’zz’); (47,’g’,6,’yy’); (37,’h’,4,’xx’); (27,’i’,2,’ww’); (17,’j’,6,’yy’)
    \item (57,’f’,8,’a’); (47,’g’,6,’b’); (37,’h’,4,’c’); (27,’i’,2,’d’); (17,’j’,6,’c’)
\end{enumerate}

\vspace{.5cm}