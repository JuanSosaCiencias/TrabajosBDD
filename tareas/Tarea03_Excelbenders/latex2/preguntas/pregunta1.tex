\textbf{¿Qué es una relación y qué características tiene?}\vspace{.2cm}
\begin{quote}
    Comenzamos por notar que una BDR no es más que un conjunto de relaciones.\vspace{.2cm}
    
    Una relación es un conjunto de tuplas, donde cada tupla es un conjunto de \textbf{atributos}. Cada atributo tiene un nombre y un \textbf{dominio}. Existe un conjunto no vacío de atributos que forman la \textbf{llave primaria} de la relación, esto garantiza que no existan tuplas repetidas; además de esta llave primaria, una relación puede tener \textbf{llaves foráneas} que hacen referencia a otras relaciones. Estas llaves deben coincidir con valores de la llave primaria de la relación referenciada; finalmente, existen las \textbf{llaves candidatas}, que son llaves primarias potenciales, es decir, que cumplen con las restricciones de llave primaria, pero no son seleccionadas como tal. Finalmente, una relación puede tener \textbf{restricciones de integridad}, que son condiciones que deben cumplir las tuplas de la relación. Hay 2 tipos de restricciones: inherentes y de usuario. \vspace{.2cm}

    Además de esto, sabemos que una relación tiene \textbf{nombre único} y que los atributos de una relación son \textbf{no ordenados}, es decir, que el orden de los atributos no importa. Lo mismo aplica para las tuplas. \textbf{No hay atributos multivaluados}, es decir, no hay atributos que tengan más de un valor en una tupla; además, los atributos deben ser \textbf{atómicos}, es decir, que no pueden ser descompuestos en atributos más pequeños, y tienen \textbf{nombre único} en la relación.
\end{quote}

Informacion tomada de la presentacion de la clase: "03ModeloR\texttt{\_}BD.pdf" de la materia "Fundamentos de Bases de Datos" paginas 3 a 6. \vspace{.2cm}