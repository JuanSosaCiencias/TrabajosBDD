\documentclass[11pt,letterpaper]{article}
\usepackage[utf8]{inputenc}

%ola 
%----- Configuración del estilo del documento------%
\usepackage{epsfig,graphicx}
\usepackage[left=2.5cm,right=2.5cm,top=1.8cm,bottom=2.3cm]{geometry}
%------ Paquetes matematicos --------%
\usepackage{amsmath}
\usepackage{amssymb}
\usepackage{amsthm}
\usepackage{amsmath}
\usepackage{tabularx}
\usepackage[numbers]{natbib}
\usepackage{fancyhdr}
\usepackage{lastpage}
\usepackage{verbatim}
\usepackage[shortlabels]{enumitem}
\usepackage{venndiagram}
\usepackage{xcolor}
\usetikzlibrary{shapes.geometric}
\renewcommand{\proofname}{Demostración}
\usepackage{cancel}
\usepackage{hyperref}

%%Para la portada
%\usepackage[top=1in, left=0.9in, right=1.25in, bottom=1in]{geometry}
%\usepackage[utf8]{inputenc}
\usepackage[T1]{fontenc}
\usepackage[utf8]{inputenc}
\usepackage[spanish,es-nodecimaldot,es-tabla]{babel}
\usepackage{graphicx}
\usepackage{tocloft}
\graphicspath{{./figs/}}
\usepackage{setspace}

%Color bibi
\definecolor{bibi}{RGB}{0,103,148}
% Otros colores
\definecolor{lightblue}{RGB}
\definecolor{myPurple}{RGB}{128, 0, 128}

\usepackage[dvipsnames]{xcolor}



\begin{document}
	
	\begin{titlepage}
	\thispagestyle{empty}
	\begin{minipage}[c][0.17\textheight][c]{0.25\textwidth}
		\begin{center}
			\includegraphics[width=3.5cm, height=3.5cm]{resources/Logo_UNAM.png}
		\end{center}
	\end{minipage}
	\begin{minipage}[c][0.195\textheight][t]{0.75\textwidth}
		\begin{center}
			\vspace{0.3cm}
			\textsc{\large Universidad Nacional Aut\'onoma de M\'exico}\\[0.5cm]
			\vspace{0.3cm}
			\hrule height2.5pt
			\vspace{.2cm}
			\hrule height1pt
			\vspace{.8cm}
			\textsc{Facultad de Ciencias}\\[0.5cm] %
		\end{center}
	\end{minipage}
	
	\begin{minipage}[c][0.81\textheight][t]{0.25\textwidth}
		\vspace*{5mm}
		\begin{center}
			\hskip2.0mm
			\vrule width1pt height13cm 
			\vspace{5mm}
			\hskip2pt
			\vrule width2.5pt height13cm
			\hskip2mm
			\vrule width1pt height13cm \\
			\vspace{5mm}
			\includegraphics[height=4.0cm]{resources/Logo_FC.png}
		\end{center}
	\end{minipage}
	\begin{minipage}[c][0.81\textheight][t]{0.75\textwidth}
		\begin{center}
			\vspace{1cm}
			
			{\large\scshape Fundamentos de Bases de Datos - 7094}\\[.2in]
			
			\vspace{2cm}            
			
			\textsc{\LARGE \textbf{T}\hspace{1cm}\textbf{A}\hspace{1cm}\textbf{R}\hspace{1cm}\textbf{E}\hspace{1cm}\textbf{A}\hspace{1.3cm}\textbf{3}}\\[2cm]
			\textsc{\Large{Equipo:}\normalsize \\
                \vspace{.3cm}
				\textbf{Del Monte Ortega Maryam Michelle - 320083527 \\
                \vspace{.2cm}
				\href{https://github.com/JuanSosaCiencias}{\textcolor{blue}{Sosa Romo Juan Mario - 320051926}} \\
                \vspace{.2cm}
				Castillo Hernández Antonio - 320017438 \\
                \vspace{.2cm}
                Erik Eduardo Gómez López - 320258211 \\
                \vspace{.2cm}
                Julio César Islas Espino - 320340594}}\\[0.5cm]     
			
			\textsc{{Fecha de entrega: \\ \textbf{24 de Septiembre de 2024}}}\\[0.5cm]        
			
			\textsc{{Profesor: \\ \textbf{M. en I. Gerardo Avilés Rosas}}}\\[0.5cm]  
			
			\textsc{Ayudantes: \\ \textbf{Luis Enrique García Gómez \\ Kevin Jair Torres Valencia \\ Ricardo Badillo Macías \\ Rocío Aylin Huerta González
			} }
			
			
			\vspace{0.5cm}
		\end{center}
	\end{minipage}
\end{titlepage}

	
	\begin{center}
		\section*{\LARGE{Tarea 1}}
	\end{center}

    % Conceptos generales  
    \begin{center}
        \LARGE{\textbf{Conceptos generales}}\\
    \end{center}
    \normalsize
    \begin{enumerate}[label=\alph*.]
        \item \begin{center}
    \textbf{¿Qué es una política de mantenimiento de llaves foráneas?}   
\end{center}

Bien sabemos que en PostgreSQL, una \textbf{llave foránea} es una restricción que se utiliza para garantizar la integridad entre dos tablas. Entonces, en pocas palabras, en una tabla se referencia el campo clave primaria en otra tabla. \\

Son reglas que definen qué sucede con los registros relacionados entre dos tablas, a través de llaves foráneas, cuando se hacen cambios en la tabla que contiene la clave primaria. \\

Nos sirven para que exista coherencia entre los datos de ambas tablas cuando se eliminan o actualizan registros. \\

Las principales políticas son:

\begin{itemize}
    \item \textbf{ON DELETE}: Esta política especifica qué sucede cuando se elimina un registro de la tabla referenciada (la que contiene la clave primaria).
    \item \textbf{ON UPDATE}: Esta política especifica qué sucede cuando se actualiza un valor de clave primaria en la tabla referenciada.
\end{itemize}

PostgreSQL define diferentes políticas que pueden configurarse cuando
se crean las restricciones de llaves foráneas, las cuales determinan
cómo se gestionan los cambios en los registros referenciados. \\

Las opciones disponibles para ambas políticas son:

\begin{itemize}
    \item \textbf{CASCADE}: Si se elimina o actualiza un registro en la tabla referenciada, se eliminarán o actualizarán automáticamente los registros correspondientes en la tabla que contiene la llave foránea.
    \item \textbf{SET NULL}: Si se elimina o actualiza un registro en la tabla referenciada, los campos de la llave foránea en la tabla dependiente se establecerán en \texttt{NULL}.
    \item \textbf{SET DEFAULT}: Si se elimina o actualiza un registro en la tabla referenciada, los campos de la llave foránea en la tabla dependiente se establecerán en un valor por defecto predefinido.
    \item \textbf{RESTRICT}: Impide la eliminación o actualización del registro referenciado si hay registros dependientes en la tabla que contiene la llave foránea.
    \item \textbf{NO ACTION}: Similar a RESTRICT, pero la validación ocurre al final de la transacción. PostgreSQL no permite que se complete la transacción si la llave foránea se ve comprometida.
\end{itemize}

Estas políticas permiten mantener la consistencia e integridad de los datos en una base de datos relacional, asegurando que no haya referencias huérfanas o inconsistentes entre las tablas. \\

        \item \begin{center}
    \textbf{Describe cual es el mas conveniente utilizar.}
    \vspace{.5cm}
\end{center}

Depende de la complejidad de los datos y las necesidades del proyecto.

Un \textbf{Sistema de Datos} es adecuado cuando se requiere una estructura simple y poco procesamiento. Si solo necesitas almacenar archivos en carpetas y no hay relaciones complejas entre los datos, un sistema de archivos es una opción eficiente. Además, es más fácil de implementar y mantener en proyectos pequeños o específicos.

En cambio, una \textbf{Base de Datos} es preferible cuando se necesita manipular grandes volúmenes de datos estructurados, realizar consultas avanzadas, o cuando se requieren funcionalidades como seguridad, control de acceso, integridad referencial o escalabilidad. Las bases de datos permiten automatizar procesos complejos, mejorar la productividad y reducir errores, facilitando la toma de decisiones en empresas que manejan datos críticos y en constante evolución.\\

\cite{sooluciona}
\vspace{.5cm}

        \item Para cada uno de los \textbf{esquemas} que se muestran a continuación, con su respectivo \textbf{conjunto de dependencias funcionales}:

\begin{enumerate}[label=\alph*.]
    \item \textbf{R(A, B, C, D, E, F, G)} con \textbf{F = \{AB $\rightarrow$ C, AB $\rightarrow$ F, A $\rightarrow$ D, A $\rightarrow$ E, B $\rightarrow$ G\}}
    \item \textbf{R(A, B, C, D, E, F)} con \textbf{F = \{AB $\rightarrow$ C, BC $\rightarrow$ AD, D $\rightarrow$ E, CF $\rightarrow$ B\}}
\end{enumerate}

\begin{itemize}
    \item Indica \textbf{alguna llave candidata} para la relación \textbf{R}. \vspace{.3cm}

    \begin{enumerate}[label=\alph*.]
        \item \{AB\}+ = \{ABCFDEG\} es llave candidata pues cumple con \textbf{identificación única} por tener a todos los atributos de \textbf{R} y \textbf{no redundancia.} pues si eliminamos a cualquiera de sus atributos, no se cumple con la identificación única.  \vspace{.2cm}
        
        \item \{CFA\}+ = \{CFABDE\} notemos que aqui tenemos que incluir a CF pues la unica manera de agregarlo es directamente (no existe DF con CF a la derecha) y no podemos quitar a ninguno de los atributos de la llave candidata pues no se cumple con la identificación única. \vspace{.2cm}
    \end{enumerate}
    \vspace{.3cm}

    \item Especifica \textbf{todas las violaciones} a la \textbf{BCNF}. \vspace{.3cm}
    
    \begin{enumerate}[label=\alph*.]
        \item \vspace{.2cm}
        \item \vspace{.2cm}
    \end{enumerate}
    \vspace{.3cm}

    \item \textbf{Normaliza} de acuerdo con \textbf{BCNF}, asegúrate de indicar cuáles son las \textbf{relaciones resultantes} con sus respectivas \textbf{dependencias funcionales}. \vspace{.3cm}
    
    \begin{enumerate}[label=\alph*.]
        \item \vspace{.2cm}
        \item \vspace{.2cm}
    \end{enumerate}
    \vspace{.3cm}
    

\end{itemize}
        \item Para cada uno de los esquemas que se muestran a continuación, con su respectivo conjunto de dependencias funcionales:

\begin{itemize}
	\item[a)] \( R(A, B, C, D, E, F, G) \) con \( F = \{AB \to C, AB \to F, A \to D, A \to E, B \to G\} \).
	\item[b)] \( R(A, B, C, D, E, F, G) \) con \( F = \{A \to B, CD \to FG, G \to E, B \to D, A \to C, E \to A\} \).
\end{itemize}

\subsubsection*{Inciso a)}
\begin{itemize}
	\item Indica alguna llave candidata para la relación \( R \).

		Una llave candidata para la relación \( R \) podría ser \( AB \), ya que:
		\begin{itemize}
			\item \( AB \rightarrow C \)
			\item \( AB \rightarrow F \)
			\item \( A \rightarrow D \) y \( A \rightarrow E \) implican que, conociendo \( A \), se puede determinar \( D \) y \( E \).
			\item \( B \rightarrow G \) implica que, conociendo \( B \), se puede determinar \( G \).
		\end{itemize}
		Por lo tanto, \( AB \) es una llave candidata.
	\vspace*{.3cm}
	\item Indica las violaciones a 3NF que encuentres en \( R \).
	
	Las siguientes dependencias violan la 3NF:
	\begin{itemize}
		\item \( A \rightarrow D \): \( D \) no es parte de una llave candidata y \( A \) no es superllave.
		\item \( A \rightarrow E \): \( E \) no es parte de una llave candidata y \( A \) no es superllave.
		\item \( B \rightarrow G \): \( G \) no es parte de una llave candidata y \( B \) no es superllave.
	\end{itemize}
	\vspace*{.3cm}
	
	\item Encuentra el conjunto mínimo de dependencias funcionales equivalente a \( F \).
	\begin{itemize}
		\item Paso a paso:
		\begin{itemize}
			\item \( AB \rightarrow C \): No es redundante, ya que necesitamos \( AB \) para determinar \( C \).
			\item \( AB \rightarrow F \): No es redundante, necesitamos \( AB \) para determinar \( F \).
			\item \( A \rightarrow D \): No es redundante, ya que no podemos deducir \( D \) de ninguna otra dependencia sin \( A \).
			\item \( A \rightarrow E \): No es redundante, ya que necesitamos \( A \) para determinar \( E \).
			\item \( B \rightarrow G \): No es redundante, necesitamos \( B \) para determinar \( G \).
		\end{itemize}
		Por lo tanto, el conjunto mínimo equivalente a \( F \) es:
		\[
		F_{\text{mín}} = \{ AB \rightarrow C, AB \rightarrow F, A \rightarrow D, A \rightarrow E, B \rightarrow G \}
		\]
	\end{itemize}
	\vspace*{.3cm}
	
	\item Normaliza de acuerdo con la 3NF. Indica claramente las relaciones resultantes y, en cada esquema, las dependencias funcionales que se cumplen.
	\begin{itemize}
		\item Relación 1: \( R_1(A, B, C, F) \) con las dependencias \( AB \rightarrow C \) y \( AB \rightarrow F \).
		\item Relación 2: \( R_2(A, D, E) \) con las dependencias \( A \rightarrow D \) y \( A \rightarrow E \).
		\item Relación 3: \( R_3(B, G) \) con la dependencia \( B \rightarrow G \).
	\end{itemize}
	Estas relaciones cumplen con la 3NF, ya que todas las dependencias están cubiertas.
\end{itemize}
\vspace*{.3cm}

\subsubsection*{Inciso b)}
\begin{itemize}
	\item Indica alguna llave candidata para la relación \( R \).
	
	Una posible llave candidata para la relación \( R \) es \( CD \), ya que:
	\begin{itemize}
		\item \( CD \rightarrow FG \): Nos da acceso a \( F \) y \( G \).
		\item \( G \rightarrow E \) implica que podemos determinar \( E \) si conocemos \( G \).
		\item \( E \rightarrow A \) permite determinar \( A \).
		\item \( A \rightarrow B \) y \( A \rightarrow C \) permiten determinar \( B \) y \( C \).
	\end{itemize}
	Por lo tanto, \( CD \) es una llave candidata.
	\vspace*{.3cm}
	
	\item Indica las violaciones a 3NF que encuentres en \( R \).
	
	Las siguientes dependencias violan la 3NF:
	\begin{itemize}
		\item \( G \rightarrow E \): \( E \) no es parte de una llave candidata y \( G \) no es superllave.
		\item \( B \rightarrow D \): \( D \) no es parte de una llave candidata y \( B \) no es superllave.
		\item \( A \rightarrow C \): \( C \) no es parte de una llave candidata y \( A \) no es superllave.
	\end{itemize}
	\vspace*{.3cm}
	
	\item Encuentra el conjunto mínimo de dependencias funcionales equivalente a \( F \).
	
	Pasos:
	\begin{itemize}
		\item \( A \rightarrow B \): No es redundante, ya que necesitamos \( A \) para determinar \( B \).
		\item \( CD \rightarrow FG \): No es redundante, necesitamos \( CD \) para determinar \( F \) y \( G \).
		\item \( G \rightarrow E \): No es redundante, necesitamos \( G \) para determinar \( E \).
		\item \( B \rightarrow D \): No es redundante, necesitamos \( B \) para determinar \( D \).
		\item \( A \rightarrow C \): No es redundante, necesitamos \( A \) para determinar \( C \).
		\item \( E \rightarrow A \): No es redundante, necesitamos \( E \) para determinar \( A \).
	\end{itemize}
	El conjunto mínimo equivalente a \( F \) es:
	\[
	F_{\text{mín}} = \{ A \rightarrow B, CD \rightarrow FG, G \rightarrow E, B \rightarrow D, A \rightarrow C, E \rightarrow A \}
	\]
	\vspace*{.3cm}
	
	\item Normaliza de acuerdo con la 3NF. Indica claramente las relaciones resultantes y, en cada esquema, las dependencias funcionales que se cumplen.
	\begin{itemize}
		\item Relación 1: \( R_1(A, B, C) \) con las dependencias \( A \rightarrow B \) y \( A \rightarrow C \).
		\item Relación 2: \( R_2(C, D, F, G) \) con la dependencia \( CD \rightarrow FG \).
		\item Relación 3: \( R_3(G, E) \) con la dependencia \( G \rightarrow E \).
		\item Relación 4: \( R_4(B, D) \) con la dependencia \( B \rightarrow D \).
		\item Relación 5: \( R_5(E, A) \) con la dependencia \( E \rightarrow A \).
	\end{itemize}
	Estas relaciones cumplen con la 3NF, ya que todas las dependencias están cubiertas.
\end{itemize}
        \item \begin{center}
    \textbf{Con base a lo anterior, ¿cuál política utilizarán para su esquema, y porqué motivo?}
\end{center}

Para todas nuestras llaves foránea usamos la política on delete cascade y on update cascade.

Lo anterior debido a 

        \item Se tiene la siguiente relación:
\begin{center}
    \textbf{R(idEnfermo, idCirujano, fechaCirugía, nombreEnfermo, direcciónEnfermo, nombreCirujano, nombreCirugía, medicinaSuministrada, efectosSecundarios)}
\end{center}
\begin{itemize}
    \item \textbf{Expresa las siguientes restricciones en forma de dependencias funcionales:}
    
    \begin{itemize}
        \item A un enfermo sólo se le da una medicina después de la operación. 
        \item Si existen efectos secundarios estos dependen sólo de la medicina suministrada.
        \item Sólo puede existir un efecto secundario por medicamento.
    \end{itemize}

    Podemos satisfacer las restricciones con:
    \begin{itemize}
        \item $idEnfermo, fechaCirugia \rightarrow medicinaSuministrada$

        Esto porque indica que para un enfermo específico en una fecha de cirugía específica, solo se le puede administrar una medicina. Lo cuál cubre la primera restricción. 

        Ocupamos tanto el idEnfermo como la fecha de la cirugía, o de otra manera no sabríamos a que enfermo administrarle la medicina después de su cirugía (fecha).

        \item $medicinaSuministrada \rightarrow efectosSecundarios$

        Aquí cumplimos con las dos restricciones que faltaban, porque asignamos a cada medicina un único efecto secundario. En este caso, es una dependencia funcional simple ya que solo ocuparemos la medicina para saber qué efecto secundario tiene. 
    \end{itemize}
    
    \item \textbf{Especifica otras dependencias funcionales o multivaluadas que deban satisfacerse en la relación \textbf{R}. Por cada una que definas, deberá aparecer un enunciado en español como en el inciso anterior.}

    Otras dependencias funcionales en \textbf{R} son:
    \begin{itemize}[label=$\heartsuit$]
        \item Cada enfermo tiene un único nombre y una única dirección
        \[idEnfermo \rightarrow nombreEnfermo, direccionEnfermo\]
        \item Cada cirujano tiene un único nombre
        \[idCirujano \rightarrow nombreCirujano\]
        \item Una cirugía realizada por un cirujano a un enfermo sólo puede ocurrir en una fecha determinada
        \[idEnfermo, idCirujano, nombreCirugia \rightarrow fechaCirugia\]
        \item Un enfermo en una fecha específica solo puede tener una cirugía y esta debe ser realizada por un único cirujano
        \[idEnfermo, fechaCirugia \rightarrow nombreCirugia, idCirujano\]
        \item El enfermo y la fecha de cirugía determina el cirujano que realizó la operación
        \[idEnfermo, fechaCirugia \rightarrow idCirujano\]
        \item El cirujano y la fecha determina qué cirugía se está realizando en ese momento
        \[idCirujano, fechaCirugia \rightarrow nombreCirugia\]
       
        \item El nombre de la cirugía determina qué medicinas pueden ser suministradas después de la operación
        \[nombreCirugia \twoheadrightarrow  medicinaSuministrada\]
        Esta es una dependencia multivaluada ya que una cirugía puede tener varias medicinas posibles
        \item La combinación de enfermo, cirujano y fecha determina unívocamente todos los demás atributos de la relación

        \begin{center}
            
        $idEnfermo, idCirujano, fechaCirugia \rightarrow nombreEnfermo, direccionEnfermo, nombreCirujano, nombreCirugia,$
        
        $medicinaSuministrada, efectosSecundarios$
        \end{center}
        

    \end{itemize}
    Estas dependencias surgen de restricciones pensadas por nosotros mismos, por lo que podrían no ser generales en un contexto de un hospital real.
    
    \item \textbf{Normaliza utilizando el conjunto de dependencias establecido en los puntos anteriores.}

    Tenemos el siguiente conjunto de dependencias:

    \textbf{F} = \{
    
    1. idEnfermo $\rightarrow$ nombreEnfermo, direccionEnfermo
    
    2. idCirujano $\rightarrow$ nombreCirujano
    
    3. idEnfermo, idCirujano, nombreCirugia $\rightarrow$ fechaCirugia
    
    4. idEnfermo, fechaCirugia $\rightarrow$ nombreCirugia, idCirujano
    
    5. idEnfermo, fechaCirugia $\rightarrow$ idCirujano
    
    6. idCirujano, fechaCirugia $\rightarrow$ nombreCirugia
    
    7. nombreCirugia $\twoheadrightarrow$ medicinaSuministrada
    
    8. idEnfermo, idCirujano, fechaCirugia $\rightarrow$ nombreEnfermo, direccionEnfermo, nombreCirujano, nombreCirugia, medicinaSuministrada, efectosSecundarios
    
    9. idEnfermo, fechaCirugia $\rightarrow$ medicinaSuministrada
    10. medicinaSuministrada $\rightarrow$ efectosSecundarios
\}

Como tenemos una dependencia multivaluada, usaremos $4FN$

Debido a la DF 8, la llave candidata es:

\{idEnfermo, fechaCirugia\}+=
\{idEnfermo fechaCirugia, nombreEnfermo direccionEnfermo, nombreCirugia idCirujano, nombreCirujano, medicinaSuministrada, efectosSecundarios\}


\textbf{PASO 1:} Primera violación a 4FN
\begin{itemize}
    \item La DMV nombreCirugia $\twoheadrightarrow$ medicinaSuministrada es una violación ya que nombreCirugia no es superllave
    
    \item Dividimos en dos relaciones:
    \begin{itemize}[label=\textcolor{magenta}{$\bigstar$}]
        \item $R_1$(nombreCirugia, medicinaSuministrada) 
        
        Ya está en $4FN$ pues no tiene DMVs no triviales
        \item $R_2$(idEnfermo, idCirujano, fechaCirugia, nombreEnfermo, direccionEnfermo, nombreCirujano, nombreCirugia, efectosSecundarios) con:
        \begin{itemize}
            \item idEnfermo $\rightarrow$ nombreEnfermo, direccionEnfermo
            \item idCirujano $\rightarrow$ nombreCirujano
            
            \item idEnfermo, fechaCirugia $\rightarrow$ nombreCirugia, idCirujano
        \end{itemize}
    \end{itemize}
\end{itemize}


\textbf{PASO 2}: Analizamos $R_2$

Obtenemos su llave:

\{idEnfermo fechaCirugia\}+= \{idEnfermo  fechaCirugia, nombreEnfermo  direccionEnfermo, nombreCirugia idCirujano,
nombreCirujano, efectosSecundarios\} 

Descomponemos $R_2$:
\begin{itemize}[label=\textcolor{blue}{$\clubsuit$}]
    \item Por $idEnfermo \rightarrow nombreEnfermo, direccionEnfermo$:
    \[
    R_3(idEnfermo, nombreEnfermo, direccionEnfermo)
    \]

    Llave: \{idEnfermo\}+=\{idEnfermo,nombreEnfermo  direccionEnfermo\} 

    Está en 4FN pues está en BCNF y no tiene DMVs
    \item Por $idCirujano \rightarrow nombreCirujano$:
    \[
    R_4(idCirujano, nombreCirujano)
    \]
    Llave: \{idCirujano\}+=\{idCirujano, nombreCirujano\}

    Está en 4FN 
    \item Por último obtendríamos
\[R_5(idEnfermo, idCirujano, fechaCirugia, nombreCirugia)\]
con
\begin{itemize}
    \item idEnfermo, fechaCirugia $\rightarrow$ nombreCirugia, idCirujano
    \item idEnfermo, idCirujano, fechaCirugia $\rightarrow$ nombreCirugia

\end{itemize}
Calculamos la llave 

\{idEnfermo idCirujano fechaCirugia\}+=
\{idEnfermo idCirujano fechaCirugia, nombreCirugia\}

\{idEnfermo fechaCirugia\}+=
\{idEnfermo fechaCirugia, nombreCirugia idCirujano\}

Por lo tanto la llave es \{idEnfermo, fechaCirugia\}

Está en 4FN pues está en BCNF y no tiene DMVs

\end{itemize}

El esquema final sería:
\begin{center}
    $R_1(nombreCirugia, medicinaSuministrada)$
    
    $R_3(idEnfermo, nombreEnfermo, direccionEnfermo)$
    
    $R_4(idCirujano, nombreCirujano)$
    
    $R_5(idEnfermo, idCirujano, fechaCirugia, nombreCirugia)$

\end{center}

Incluso podemos renombrar las relaciones para ser explícitos:
\begin{center}
    $MEDICINAS\_POR\_CIRUGIA(nombreCirugia, medicinaSuministrada)$
    
    $ENFERMO(idEnfermo, nombreEnfermo, direccionEnfermo)$
    
    $CIRUJANO(idCirujano, nombreCirujano)$
    
    $CIRUGIA(idEnfermo, idCirujano, fechaCirugia, nombreCirugia)$

\end{center}
\end{itemize}


        \item \begin{center}   
\textbf{Investiga cuáles son las responsabilidades de un DBA. Si asumimos que el DBA nunca está
interesado en ejecutar sus propias consultas, ¿necesita entender y/o conocer el modelo de datos
lógico de la base de datos? Justifica tu respuesta} \\
\end{center}

Las responsabilidades de un DBA, consiste en gestionar el software de las bases de datos, determina la organización y la almacena digitalmente, en la cual verifica la integridad de los datos y responsabilizándose de su seguridad.


\begin{table}[h]
\centering
\begin{tabular}{|p{6cm}|p{6cm}|}
\hline
\textbf{Tareas de BD} & \textbf{Permisos a los usuarios} \\
\hline
Se encarga del diseño, desarrollo y mantenimiento de las BD, de los cambios, de comprobar el funcionamiento correcto y de la eficacia del acceso de las bases de datos & Permite a los usuarios guardar, ordenar, extraer los datos y compartirlos a través de una red interna o por internet \\
\hline
\end{tabular}
\caption{Descripción de tareas y permisos}
\end{table}

Respondiendo a la pregunta de \textquotedblleft Si asumimos que el DBA
nunca está interesado en ejecutar sus propias consultas, ¿necesita entender y/o conocer el modelo de datos lógico de la base de datos?\textquotedblright
\\
\textcolor{blue}{Tod ala información de esta sección se sacó del trexto proporcionado\cite{Barcelona-Activa}} \\


Bueno, no necesariamente. Si el DBA (Administrador de Base de Datos) no está interesado en ejecutar sus propias consultas; entonces su principal enfoque estaría en la administración, seguridad, rendimiento y disponibilidad de la base de datos. Sin embargo el conocer el modelo de datos lógico puede ser útil para entender mejor cómo están organizados los datos, cómo se relacionan entre sí y cómo optimizar el rendimiento del sistema. Claramente, esto le ayudaría a tomar decisiones más informadas sobre cómo administrar la base de datos de manera eficiente. Pero bien sabemos que podría realizar su trabajo sin dicha actividad, sin embargo como se dijo anteriormente, es un plus que se tiene para realizar mejor la Base de Datos. \\

        \item \begin{center}
\textbf{Investiga por qué surgieron los sistemas NoSQL en la década de 2000 y compara a través de una tabla sus características vs. los sistemas de bases de datos relacionales.}
\end{center}

 Este tipo de sistemas se desarrollaron para abordar las limitaciones de los sistemas relacionales, que a menudo enfrentaban desafíos en términos de escalabilidad, flexibilidad y rendimiento al manejar grandes volúmenes de datos no estructurados. A medida que las aplicaciones web y móviles crecieron en popularidad, se necesitaban soluciones capaces de escalar horizontalmente, manejar datos diversos sin esquemas fijos y ofrecer alta disponibilidad con mejor tolerancia a fallos. Los sistemas NoSQL respondieron a estas necesidades al proporcionar modelos de datos más flexibles, mayor rendimiento y capacidades avanzadas para manejar datos en tiempo real y grandes volúmenes de información.

 Veamos la comparación de algunas características :

 \begin{table}[h!]
\centering
\small 
\begin{tabularx}{\textwidth}{|X|X|X|}
\hline
\textbf{Característica} & \textbf{\textcolor{myPurple}{Sistemas NoSQL}} & \textbf{\textcolor{lightblue}{Sistemas de Bases de Datos Relacionales}} \\
\hline
\textbf{Modelo de Datos} & Flexible (clave-valor, columnares, documentos, grafos) & Estructurado (tablas con esquema fijo) \\
\hline
\textbf{Escalabilidad} & Horizontal (agregar más servidores) & Vertical (mejorar hardware del servidor) \\
\hline
\textbf{Rendimiento} & Optimizado para alta disponibilidad y baja latencia & Puede ser más lento con grandes volúmenes de datos \\
\hline
\textbf{Flexibilidad del Esquema} & Esquema dinámico, datos semi-estructurados o no estructurados (más flexible) & Esquema fijo y predefinido (menos flexible) \\
\hline
\textbf{Garantías ACID} & A menudo sacrificadas por rendimiento y disponibilidad & Fuerte soporte para transacciones ACID \\
\hline
\textbf{Disponibilidad} & Diseñado para replicación y particionamiento distribuidos & Requiere configuraciones adicionales para alta disponibilidad \\
\hline
\textbf{Usos} & Aplicaciones web grandes, big data, datos en tiempo real & Aplicaciones empresariales tradicionales, sistemas transaccionales \\
\hline
\textbf{Cuándo usar} & Cuando el volumen de datos no crece tanto con el tiempo & Cuando el volumen de datos crece rápidamente \\
\hline
\end{tabularx}
\label{tab:nosql_vs_rdbms}
\end{table}

\textcolor{blue}{Toda la información de esta sección se ha tomado del texto proporcionado \cite{nosql}.} \\

        \item \begin{center}
\textbf{Investiga tres noticias relacionadas con las hojas de cálculo, en donde se muestre que hay errores en el procesamiento/manejo/distribución de datos y realiza una discusión breve de cada una de ellas, contrastándolas con el enfoque de las bases de datos. ¿Por qué consideras qué a pesar de los ejemplos mostrados, las hojas de cálculo siguen siendo la herramienta más prevaleciente en un ambiente de negocios típico?} \\
\end{center}

La primera noticia que tomamos como referencia nos habla acerca de la perdida financiera de 86 millones de euros por parte del \textit{Norges Bank} de Noruega, debido a una falla en una hoja de cálculo, que de acuerdo al medio que cubrió la noticia todo fue debido a un error humano en el cual se vio involucrado un empleado que según sus propias palabras \textit{"Fue un error manual. Utilicé la fecha equivocada, el 1 de diciembre en lugar del 1 de noviembre, como estaba claramente marcado"}. Por lo cual como podemos apreciar se trato de una casilla mal colocada en una hoja de Excel y debido a esto, se registro un impacto de 0.7 puntos en el fondo soberano noruego lo cual provoco la perdida de 900 millones de coronas o 86 millones de euros como se menciono al principio. 
\textcolor{blue}{Toda la información de esta sección se ha tomado del texto proporcionado\cite{dinero_en_imagen_2024}}
\\

Ahora pasando a la segunda noticia, esta nos habla sobre una situación que aconteció en la pandemia, mas en especifico en reino unido. Según el medio del cual es sacada esta noticia, las autoridades sanitarias británicas anunciaron que casi 16,000 casos de coronavirus no se informaron durante una semana. De acuerdo con \textit{Public Health England (PHE)}, que contabilizaba diariamente las nuevas infecciones por coronavirus en Excel, se omitieron 15,841 casos nuevos entre el 25 de septiembre y el 2 de octubre del 2020. Este fallo se debió a que la hoja de calculo de Excel había alcanzado su limite. \\

Esto ocasiono que el numero de personas infectadas de los casos nuevos diarios fueran significativamente más altos de lo que parecían, lo que daba como especulación que propagación de la infección en Inglaterra pareciese ser marcadamente mayor de lo que se informó. Lo cual afecto de manera mediática en como fue percibida la propagación del virus en reino unido, todo debido a un fallo en el software que se utilizaba para almacenar y consultar los datos.\textcolor{blue} \\
{Toda la información de esta sección se ha tomado del texto proporcionado\cite{deyden-2020}}
\\

Pasando ahora con la tercera noticia que se solicito, esta nos habla acerca de la necesidad de migrar del sistema convencional en el cual se utiliza Excel para las cuestiones relacionadas al almacenamiento y consulta de información en el ámbito del \textit{Big Data} ya que las hojas de calculo ya no son suficientes para las necesidades y requerimientos de las empresas en dicho ámbito. Por ejemplo, se sabe que hasta 2018, el volumen de información en el mundo se calculaba en 33 zettabytes. Al cierre de 2020, casi se duplicó al llegar a 59 zettabytes y esta cifra sigue creciendo conforme pasan los años, por lo cual se necesitan nuevos sistemas que soporten una mayor cantidad de datos como lo son las bases de datos. \\

Es por eso que el volumen casi incontrolable de información que se va generando segundo a segundo ha provocado que las hojas de calculo cada vez se queden más y más atrás.  \cite{escobar-2023} \\

Ahora bien, si contrastamos las 3 noticias anteriores con el enfoque de las bases de datos, se pueden observar en general algunas diferencias en términos de robustez, escalabilidad y manejo de errores vemos por ejemplo que en la primera noticia, por ejemplo; en la primera noticia, el error mas que nada humano de ingresar una fecha incorrecta podría haberse evitado mediante el uso de una base de datos con reglas de validación más estrictas y control de versiones, pudiendo así evitar el ingreso de datos erróneos.\\

En la segunda noticia, la limitación de Excel al alcanzar su capacidad máxima se podría haberse evitado utilizando una base de datos relacional o no relacional que puede manejar millones de registros sin problemas de rendimiento. Por ultimo en la tercera noticia, se discute la necesidad de migrar de hojas de cálculo a sistemas más avanzados para el manejo de Big Data y como ya lo mencionábamos las bases de datos están diseñadas para gestionar volúmenes masivos de información, permitiendo no solo su almacenamiento, sino también su procesamiento y análisis en tiempo real. \cite{kde-2023}\\

Ahora, respondiendo a la pregunta. Nosotros pensamos que las hojas de calculo siguen siendo la herramienta mas prevaleciente en un ambiente de negocios típico porque la infraestructura esta principalmente diseñada para tal, así como el personal de dichas instituciones pudiera estar mas familiarizado con el uso de Excel que con el manejo de una base de datos, todo esto aunado a que a lo largo de los años, Excel de la mano de Microsoft ha hecho que la mayoría de las empresas a nivel mundial utilicen su producto generando de esta manera un \textit{estatus quo} en las empresas y el ambiente de las mismas en general. \\

En conclusión, podemos decir que las hojas de calculo son mas utilizadas que las bases de datos debido a su popularidad y fácil acceso comparado con una base de datos tradicional, aunado a la infraestructura de las empresas mismas y el personal encargado de manejar los datos dentro de las organizaciones.\\

        \item \begin{center}
\textbf{Supón que deseas crear una red social orientada al uso empresarial, negocios y empleo, similar
a LinkedIn. Considera cada una de las desventajas indicadas en el documento “Purpose of Database Systems”, cuando se administran los datos en un sistema de archivos. Discute la relevancia de cada uno de los puntos indicados, con respecto al almacenamiento de datos de los perfiles profesionales: el usuario que lo subió, las fechas de los empleos que ha tenido, sus habilidades, sus postulaciones y los usuarios que vieron su perfil, cantidad de publicaciones realizadas, cantidad de reacciones, cantidad de compartidos, entre otros.}\\
\end{center}

En el documento nos explican como, para hacer nuestra base de información una universidad utilizo una serie de archivos y para modificar y manipular esta información se tienen que tener programas que hagan ciertas funciones como añadir estudiantes, materias o registrar alumnos, esto supone un problema a la hora de querer hacer cambios grandes o añadir mas información pues por ejemplo si necesitamos crear un curso nuevo o cambiar las reglas de la escuela habrá que cambiar varios archivos de programas y sera difícil de automatizar.\\

En nuestro caso suponemos que queremos crear un pseudo Linkedin; es claro, que en este caso, va a haber bastante mas manipulación de datos que en el caso ejemplo entonces para empezar, tendremos este problema de tener que reescribir programas cada que queramos cambiar cosas.\\

Ahora vamos a repasar las otras desventajas para ver como afectan a nuestro caso:
\begin{itemize}
    \item \textbf{Inconsistencia y redundancia de datos:} El texto explica que debido a que los programas probablemente se escriba entre varias personas a lo largo del tiempo quizás en diferentes lenguajes y con diferentes estructuras, es altamente probable que los archivos tengan grandes inconsistencias entre datos guardados y contengan varias veces la misma información, en nuestro ejemplo, por ejemplo al tener un programador que crea su archivo para registrar perfiles profesionales y otro que escribe por ejemplo el programa para agregar una persona en un perfil de empresa, es muy probable que ambos repitan el proceso de guardar muchos de los mismos datos y especialmente si no están coordinados van a guardar 2 veces el mismo perfil; aun así, el problema no acaba ahí, pues sabemos que la información al no ser estática, probablemente sera modificada y quizás por un tercer programador que si no trata con cuidado la información va a pasar que empiece a generar inconsistencias dentro del mismo archivo o a lo largo de varios archivos.\\
    \item \textbf{Dificultad para acceder a la información:} Este titulo se explica solo, pero, por ejemplo si una empresa quiere buscar a programadores que sepan utilizar lenguaje de consulta SQL, como en realidad no existe una manera simple de consultar esto, podemos o obtener todos los programadores de todo (si es que existe tal lista) que consumiría muchos recursos y seria luego tedioso filtrar, o alternativamente podríamos poner a alguien a hacer un programa para conseguir esta lista pero obviamente esto también tardaría y probablemente costaría mas.\\
    \item \textbf{Datos aislados:} Como ya dijimos los archivos de texto tienen limites informáticos mucho mas chicos que aquellos que tenemos si usamos BDD por tanto cuando estos se saturan hay que partir nuestra información, en nuestro ejemplo seria tener que dividir nuestra base de usuarios entre 2 o mas archivos.\\
    \item \textbf{Integridad de los datos:} El problema es que en los archivos es difícil mantener el formato de nuestros datos, entonces por ejemplo si en un archivo se guardan los nombres en formato <Nombre APaterno AMaterno> esto se tiene que mantener por el programador y siempre que haya un nuevo archivo o nuevo programa se debe verificar que se tenga esta restricción.\\
    \item \textbf{Falta de atomicidad:} Es importante que se mantengan partes independientes para prevenir hacer errores, lo mas importante es que si una operación se va a realizar debe poderse realizar completa o no realizarse, por ejemplo, si un usuario mete su CV a una postulación, no debe pasar que el lo suba y no le llegue, o que lo pueda subir mas de una vez porque no se registro.\\
    \item \textbf{Anomalías de concurrencia:} Sucede que en muchos caos ocupamos que muchos usuarios puedan acceder a la información al mismo tiempo y en muchos casos que modifiquen esta información, es así que surgen complicaciones, porque digamos en nuestro ejemplo que varios administradores de la pagina de nuestro "linkedin" quieren modificar la cantidad de personas que quieren en una postulación, si ambos modifican el dato y no se trata con cuidado es posible que solo se quede uno de los datos, pero también es posible que se solapen ambos cambios y quede mal el dato.\\
    \item \textbf{Seguridad:} Esto es simple, los archivos tienen medidas de seguridad muy débiles, controlar quien puede ver y modificarlos es muy complicado, además, varias de las medidas de seguridad por defecto son mucho mas débiles que medidas usadas en BDD, en nuestro ejemplo es esencial que pocas personas tengan acceso a los datos de acceso de los usuarios y es esencial que se tenga fino control sobre quien puede modificar que datos en que momento.
\end{itemize}

\textcolor{blue}{Toda la información de esta sección se saco del texto proporcionado. \cite{PuprposeOfDBS}}
    \end{enumerate}
    \newpage

    % ´Lectura del articulo
    \begin{center}
    \LARGE{\textbf{Lectura del articulo}}
    \normalsize
    \begin{enumerate}[label=\alph*.]
        \item \begin{center}
    \textbf{¿Qué es una política de mantenimiento de llaves foráneas?}   
\end{center}

Bien sabemos que en PostgreSQL, una \textbf{llave foránea} es una restricción que se utiliza para garantizar la integridad entre dos tablas. Entonces, en pocas palabras, en una tabla se referencia el campo clave primaria en otra tabla. \\

Son reglas que definen qué sucede con los registros relacionados entre dos tablas, a través de llaves foráneas, cuando se hacen cambios en la tabla que contiene la clave primaria. \\

Nos sirven para que exista coherencia entre los datos de ambas tablas cuando se eliminan o actualizan registros. \\

Las principales políticas son:

\begin{itemize}
    \item \textbf{ON DELETE}: Esta política especifica qué sucede cuando se elimina un registro de la tabla referenciada (la que contiene la clave primaria).
    \item \textbf{ON UPDATE}: Esta política especifica qué sucede cuando se actualiza un valor de clave primaria en la tabla referenciada.
\end{itemize}

PostgreSQL define diferentes políticas que pueden configurarse cuando
se crean las restricciones de llaves foráneas, las cuales determinan
cómo se gestionan los cambios en los registros referenciados. \\

Las opciones disponibles para ambas políticas son:

\begin{itemize}
    \item \textbf{CASCADE}: Si se elimina o actualiza un registro en la tabla referenciada, se eliminarán o actualizarán automáticamente los registros correspondientes en la tabla que contiene la llave foránea.
    \item \textbf{SET NULL}: Si se elimina o actualiza un registro en la tabla referenciada, los campos de la llave foránea en la tabla dependiente se establecerán en \texttt{NULL}.
    \item \textbf{SET DEFAULT}: Si se elimina o actualiza un registro en la tabla referenciada, los campos de la llave foránea en la tabla dependiente se establecerán en un valor por defecto predefinido.
    \item \textbf{RESTRICT}: Impide la eliminación o actualización del registro referenciado si hay registros dependientes en la tabla que contiene la llave foránea.
    \item \textbf{NO ACTION}: Similar a RESTRICT, pero la validación ocurre al final de la transacción. PostgreSQL no permite que se complete la transacción si la llave foránea se ve comprometida.
\end{itemize}

Estas políticas permiten mantener la consistencia e integridad de los datos en una base de datos relacional, asegurando que no haya referencias huérfanas o inconsistentes entre las tablas. \\

        \item \begin{center}
    \textbf{Describe cual es el mas conveniente utilizar.}
    \vspace{.5cm}
\end{center}

Depende de la complejidad de los datos y las necesidades del proyecto.

Un \textbf{Sistema de Datos} es adecuado cuando se requiere una estructura simple y poco procesamiento. Si solo necesitas almacenar archivos en carpetas y no hay relaciones complejas entre los datos, un sistema de archivos es una opción eficiente. Además, es más fácil de implementar y mantener en proyectos pequeños o específicos.

En cambio, una \textbf{Base de Datos} es preferible cuando se necesita manipular grandes volúmenes de datos estructurados, realizar consultas avanzadas, o cuando se requieren funcionalidades como seguridad, control de acceso, integridad referencial o escalabilidad. Las bases de datos permiten automatizar procesos complejos, mejorar la productividad y reducir errores, facilitando la toma de decisiones en empresas que manejan datos críticos y en constante evolución.\\

\cite{sooluciona}
\vspace{.5cm}

        \item Para cada uno de los \textbf{esquemas} que se muestran a continuación, con su respectivo \textbf{conjunto de dependencias funcionales}:

\begin{enumerate}[label=\alph*.]
    \item \textbf{R(A, B, C, D, E, F, G)} con \textbf{F = \{AB $\rightarrow$ C, AB $\rightarrow$ F, A $\rightarrow$ D, A $\rightarrow$ E, B $\rightarrow$ G\}}
    \item \textbf{R(A, B, C, D, E, F)} con \textbf{F = \{AB $\rightarrow$ C, BC $\rightarrow$ AD, D $\rightarrow$ E, CF $\rightarrow$ B\}}
\end{enumerate}

\begin{itemize}
    \item Indica \textbf{alguna llave candidata} para la relación \textbf{R}. \vspace{.3cm}

    \begin{enumerate}[label=\alph*.]
        \item \{AB\}+ = \{ABCFDEG\} es llave candidata pues cumple con \textbf{identificación única} por tener a todos los atributos de \textbf{R} y \textbf{no redundancia.} pues si eliminamos a cualquiera de sus atributos, no se cumple con la identificación única.  \vspace{.2cm}
        
        \item \{CFA\}+ = \{CFABDE\} notemos que aqui tenemos que incluir a CF pues la unica manera de agregarlo es directamente (no existe DF con CF a la derecha) y no podemos quitar a ninguno de los atributos de la llave candidata pues no se cumple con la identificación única. \vspace{.2cm}
    \end{enumerate}
    \vspace{.3cm}

    \item Especifica \textbf{todas las violaciones} a la \textbf{BCNF}. \vspace{.3cm}
    
    \begin{enumerate}[label=\alph*.]
        \item \vspace{.2cm}
        \item \vspace{.2cm}
    \end{enumerate}
    \vspace{.3cm}

    \item \textbf{Normaliza} de acuerdo con \textbf{BCNF}, asegúrate de indicar cuáles son las \textbf{relaciones resultantes} con sus respectivas \textbf{dependencias funcionales}. \vspace{.3cm}
    
    \begin{enumerate}[label=\alph*.]
        \item \vspace{.2cm}
        \item \vspace{.2cm}
    \end{enumerate}
    \vspace{.3cm}
    

\end{itemize}
    \end{enumerate}
    
\newpage

\Large \section*{Bibliografía}
  \bibliography{resources/referencias/referencias}
  \bibliographystyle{plainnat}
  
    
	

\end{document}