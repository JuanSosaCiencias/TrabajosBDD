\begin{center}
\textbf{Leer el articulo Data is the New Oil-Sort of: A View on Why This Comparison is Misleading and its Implications for Modern Data Administration y realizar un resumen del documento, destacando los puntos que a su consideración sean sean los mas relevantes (no mas de 2 cuartillas).} \\
\end{center}

El artículo "DATA IS THE NEW OIL-SORT OF: A VIEW ON WHY THIS COMPARISON IS MISLEADING AND ITS IMPLICATIONS FOR MODERN DATA ADMINISTRATION", 

El artículo "DATA IS THE NEW OIL-SORT OF: A VIEW ON WHY THIS COMPARISON IS MISLEADING AND ITS IMPLICATIONS FOR MODERN DATA ADMINISTRATION", nos cuenta de, bueno como el texto lo dice, nos comienza diciendo las similitudes entre el petróleo y los datos. Lo anterior se puede decir que es correcto pues tanto el petróleo como los datos se tienen que extraer, para después refinarlos y transformarlos en un recurso valioso.

Nos recalcan que el manejo de ambas materias primas implica los mismos pasos, se sabe que existen diferencias considerables en la implementación real; hay características especiales de los datos, que a continuación las enumero:

1.	Los datos no son consumibles: Bien se sabe que la materia prima al convertirla en producto con valor añadido, se consumen en el proceso. Pero para la materia “datos”, incluso después de sufrir modificaciones y se haya generado un producto de datos con valor añadido, los datos en bruto todavía están disponibles. 
No pierden su significado en el proceso, pues pueden procesarse de nuevo, de otra manera, y así dar nueva información de ellos. \\

2.	Los datos se pueden duplicar sin pérdida: Bien se sabe que las materias primas tangibles tienen un valor finito. Pues las gotas de petróleo son únicas, se pueden modificar, pero esto reduce su calidad. 
Los datos, por otro lado, se pueden duplicar sin ninguna pérdida. Pero en cuanto al valor de los datos, se determina por lo original; no por la oferta y la demanda como el petróleo; además también por lo único que sea su contenido. \\

3.	Los datos se generan a alta velocidad: Las materias primas bien se sabe que las encontramos dondequiera que estén de forma natural. 
Por otro lado, los datos, se generan en cualquier momento, se obtiene un objeto de datos procesable solo si se captura en ese momento exacto, sino se capturan se pierden. Sin embargo, hay varios dispositivos que tienen recursos de memoria limitadas, no pueden almacenar los datos capturados indefinidamente, pero hay algo que se llama “procesar los datos sobre la marcha”. 
Los datos requieren un modelo de transmisión, es decir, la fuente transmite los datos cuando se capturan.


Puse los 3 puntos que más me llamaron la atención, por supuesto que hay más, pero quise poner esos 3. A lo que quiero llegar es que el petróleo y los datos son los mismo, pero no iguales.

Hay un punto importante que es la evaluación de la Plataforma REFINERY, en la cual se revisa si cumple con los requisitos de seguridad, privacidad, funcionalidad y rendimiento par la administración de datos. Primero, se analiza la seguridad y privacidad del sistema, destacando que los datos están cifrados y firmados digitalmente para proteger la confidencialidad y la integridad durante su transmisión y almacenamiento. Además, se aseguran de que los datos solo sean accesibles de acuerdo con las políticas de acceso establecidas por los productores, y se utilizan filtros de privacidad para mantener la calidad de los datos sin comprometer la privacidad.


El artículo ofrece varias recomendaciones para mejorar la administración de datos, en lugar de basarse en la comparación con el petróleo:

Enfoque en la Calidad de los Datos: Las organizaciones deberían concentrarse en la calidad y relevancia de los datos más que en la cantidad. 

Inversión en Capacidades Analíticas: Es fundamental invertir en herramientas de análisis y en la formación del personal para obtener el máximo valor de los datos. Con las herramientas adecuadas y un personal bien capacitado, las organizaciones pueden transformar los datos en información útil y estratégica.

Estrategias de Privacidad y Seguridad: Las organizaciones deben desarrollar y aplicar estrategias sólidas para proteger la privacidad de los datos y gestionar el acceso de manera efectiva. Esto es crucial para mantener la confianza en el manejo de los datos y para cumplir con las regulaciones de privacidad.

Para cerrar, el artículo concluye que la comparación de los datos con el petróleo no es la mejor manera de entender la naturaleza y el valor de los datos. En lugar de ver los datos como un recurso finito que necesita ser extraído, es más útil adoptar un enfoque que reconozca que los datos son un recurso útil y continuo que necesita ser manejado y analizado de manera efectiva para obtener su verdadero valor.