\section*{Decisiones de diseño}
\section*{Modelo Entidad Relación}
A continuación, se detallan las principales entidades y relaciones, justificando las decisiones tomadas en el diseño.

\section*{Entidades principales}
\subsection*{Persona}
La entidad \textbf{Persona} es una entidad genérica que contiene atributos básicos como:
\begin{itemize}
    \item \textbf{IDPersona}: Identificador único; creamos uno sustituto porque va a ser complicado crear uno natural.
    \item \textbf{Nombre}, \textbf{Segundo Apellido}, \textbf{Fecha de Nacimiento}, \textbf{Genero}, \textbf{Teléfono}, \textbf{Correo}, y \textbf{Nacionalidad}.
\end{itemize}

\textbf{Atleta}, \textbf{Entrenador} y \textbf{Árbitro} son entidades hijas de \textbf{Persona}, ya que todas comparten estos datos básicos. Esta generalización permite evitar redundancia en el diseño. Aunque en implementación, al ser especializacion total se pierde gran parte de la ventaja. \vspace{0.3cm}

Probablemente, tendriamos que agregar un atributo que indique el equipo al que pertenece, ya que en implementación, si hay una diciplina que es en equipos esta dificil saber a que equipo pertenece cada atleta.

\subsection*{Disciplina}
La entidad \textbf{Disciplina} representa las actividades deportivas que forman parte de los Juegos Olímpicos. Sus atributos incluyen:
\begin{itemize}
    \item \textbf{IDDisciplina}: Identificador único, podriamos usar tambien el nombreDisciplina como llave primaria, si garantizamos que todas sean unicas por ejemplo "100 metros planos hombres", pero nos limitamos a que solo ocurra el evento una vez (no por temporada) y comparar strings.
    \item \textbf{NombreDisciplina}: Nombre oficial de la disciplina.
    \item \textbf{Categoría}: Clasificación de la disciplina.
    \item \textbf{NumParticipantes}: Número de atletas participantes. Este atributo es calculado a partir de saber si es en equipo y contar el número de atletas que participan en cada equipo.
\end{itemize}

\subsection*{Evento}
La entidad \textbf{Evento} representa las competiciones específicas. Sus atributos son:
\begin{itemize}
    \item \textbf{IDEvento}: Identificador único, usamos este identificador por las mismas razones que el de disciplina.
    \item \textbf{NombreEvento}, \textbf{FechaEvento}, \textbf{DuraciónMax}, \textbf{HorarioInicio}, \textbf{Fase}, \textbf{Precio}.
\end{itemize}
El precio de un evento puede variar dependiendo de la fase en la que se encuentre.

\subsection*{Localidad}
La \textbf{Localidad} es una entidad débil dependiente de \textbf{Disciplina}. Sus atributos son:
\begin{itemize}
    \item \textbf{NombreLocalidad}: Nombre del lugar.
    \item \textbf{Dirección}, incluyendo \textbf{Calle}, \textbf{Número}, \textbf{Ciudad} y \textbf{País}.
    \item \textbf{Aforo}: Capacidad máxima del lugar.
    \item \textbf{Tipo}: Clasificación del lugar.
\end{itemize}
La dependencia entre \textbf{Localidad} y \textbf{Disciplina} asegura que una localidad solo exista si está asociada a una disciplina. Localidad la consideramos entidad débil dependiendo de Disciplina en el sentido de que cada localidad estará determinada por la existencia de una disciplina, no tiene sentido que guardemos una ubicación de un lugar si no habrá nada relacionado a los juegos. \vspace{0.3cm}

Al final nos dimos cuenta que probablemente la relacion entre localidad y disciplina debería ser multivaluada, ya que una localidad puede albergar varias disciplinas.

\section*{Relaciones}

Algunas de las relaciones más importantes que hay que resaltar, junto a sus restricciones:

\subsection*{Entrenar (Entrenador - Atleta)}
La relación \textbf{Entrenar} conecta a los \textbf{Entrenadores} con los \textbf{Atletas}. Es una relación \textbf{uno a muchos}, ya que cada atleta debe tener un entrenador, y un entrenador puede preparar a varios atletas.

\subsection*{Representa (Atleta - País)}
La relación \textbf{Representa} indica que cada \textbf{Atleta} pertenece a un único \textbf{País}. Esto asegura que un atleta solo pueda competir representando a una nación específica. Además, se incluye el atributo \textbf{Temporada}, que indica el año en el que el atleta está participando.

\subsection*{Supervisa (Árbitro - Disciplina)}
La relación \textbf{Supervisa} conecta a los \textbf{Árbitros} con las \textbf{Disciplinas}. Es una relación \textbf{muchos a muchos}, ya que diferentes árbitros pueden supervisar una misma disciplina, y un árbitro puede supervisar varias disciplinas.

\subsection*{Tiene (Disciplina - Localidad)}
La relación \textbf{Tiene} indica que cada \textbf{Disciplina} tiene asociada una o más \textbf{Localidades}. Esto asegura que cada disciplina esté vinculada a un lugar donde se llevará a cabo.

\subsection*{Participa (Atleta - Disciplina)}
La relación \textbf{Participa} conecta a los \textbf{Atletas} con las \textbf{Disciplinas} en las que compiten. Es una relación \textbf{muchos a muchos}, ya que un atleta puede participar en varias disciplinas, y una disciplina puede incluir a varios atletas.

\subsection*{Consigue (Atleta - Medalla)}
La relación \textbf{Consigue} conecta a los \textbf{Atletas} con las \textbf{Medallas} obtenidas. Las medallas pueden ser de oro, plata o bronce, y están asociadas a una disciplina específica.

\subsection*{Patrocina (Patrocinador - Disciplina)}
La relación \textbf{Patrocina} conecta a los \textbf{Patrocinadores} con las \textbf{Disciplinas} que apoyan económicamente.

\subsection*{CompraEntrada (Cliente - Evento)}
La relación \textbf{CompraEntrada} conecta a los \textbf{Clientes} con los \textbf{Eventos} a los que asisten. Incluye la compra de entradas y la relación directa con el evento.

Este modelo entidad-relación busca representar de manera clara y eficiente la información necesaria para gestionar los Juegos Olímpicos. Cada relación y entidad ha sido diseñada considerando restricciones lógicas y necesidades del Caso de Uso.

\section*{Dependencias Funcionales}

\subsection*{1. País}
Relación: \texttt{País(idPaís, nombrePaís)}

\begin{itemize}
    \item $idPaís \rightarrow nombrePaís$
\end{itemize}

\subsection*{2. Atleta}
Relación: \texttt{Atleta(IDAtleta, Nombre, PrimerApellido, SegundoApellido, FechaNacimiento, Nacionalidad, Genero, Temporada, NombrePaís, IDEntrenador)}

\begin{itemize}
    \item $IDAtleta \rightarrow Nombre, PrimerApellido, SegundoApellido, FechaNacimiento, Nacionalidad, Genero,$

$ Temporada, NombrePaís, IDEntrenador$
    \item $NombrePaís \rightarrow Nacionalidad$
\end{itemize}

\subsection*{3. Entrenador}
Relación: \texttt{Entrenador(IDEntrenador, Nombre, PrimerApellido, SegundoApellido, FechaNacimiento, Nacionalidad, Genero, IDDisciplina)}

\begin{itemize}
    \item $IDEntrenador \rightarrow Nombre, PrimerApellido, SegundoApellido, FechaNacimiento, Nacionalidad, $

$Genero, IDDisciplina$
\end{itemize}

\subsection*{4. Árbitro}
Relación: \texttt{Árbitro(IDArbitro, Nombre, PrimerApellido, SegundoApellido, FechaNacimiento, Nacionalidad, Genero, IDDisciplina)}

\begin{itemize}
    \item $IDArbitro \rightarrow Nombre, PrimerApellido, SegundoApellido, FechaNacimiento, Nacionalidad, Genero, IDDisciplina$
\end{itemize}

\subsection*{5. Disciplina}
Relación: \texttt{Disciplina(IDDisciplina, NombreDisciplina, Categoría)}

\begin{itemize}
    \item $IDDisciplina \rightarrow NombreDisciplina, Categoria$
\end{itemize}

\subsection*{6. Evento}
Relación: \texttt{Evento(IDEvento, NombreEvento, FechaEvento, HoraInicio, DuraciónMax, Precio, Fase, NombreLocalidad, IDDisciplina)}

\begin{itemize}
    \item $IDEvento \rightarrow NombreEvento, FechaEvento, HoraInicio, DuraciónMax, Precio, Fase, $

$NombreLocalidad, IDDisciplina$
\end{itemize}

\subsection*{7. Localidad}
Relación: \texttt{Localidad(NombreLocalidad, Calle, Número, Ciudad, País, Aforo, Tipo)}

\begin{itemize}
    \item $NombreLocalidad \rightarrow Calle, Número, Ciudad, Pais, Aforo, Tipo$
\end{itemize}


\subsection*{8. CompraEntrada}
Relación: \texttt{CompraEntrada(IDCliente, IDEvento)}

\begin{itemize}
    \item $IDCliente \rightarrow IDEvento$
\end{itemize}


\subsection*{9. Patrocina}
Relación: \texttt{Patrocina(NombrePatrocinador, IDDisciplina)}

\begin{itemize}
    \item $NombrePatrocinador \rightarrow IDDisciplina$
\end{itemize}

\subsection*{10. Medalla}
Relación: \texttt{Medalla(TipoMedalla, IDDisciplina, IDAtleta)}

\begin{itemize}
    \item $IDDisciplina, IDAtleta \rightarrow TipoMedalla$
\end{itemize}

\subsection*{11. Concursa}
Relación: \texttt{Concursa(IDAtleta, IDEvento)}

\begin{itemize}
    \item $IDAtleta \rightarrow  IDEvento$
\end{itemize}

\subsection*{12. Participa}
Relación: \texttt{Participa(IDAtleta, IDDisciplina)}

\begin{itemize}
    \item $IDAtleta \rightarrow IDDisciplina$
\end{itemize}

\section*{Modelo Relacional}
\section*{Restricciones del Modelo Relacional}

\subsection*{Entidad: País}
\textbf{Atributos:}
\begin{itemize}
    \item \textbf{NombrePaís:}
    \begin{itemize}
        \item Dominio: Cadena de caracteres (\texttt{varchar(50)}).
        \item Restricción: Llave primaria (PK).
    \end{itemize}
\end{itemize}
\textbf{Llave primaria:}
\begin{itemize}
    \item \textbf{NombrePaís} identifica de forma única a cada país.
\end{itemize}

\subsection*{Entidad: Atleta}
\textbf{Atributos:}
\begin{itemize}
    \item \textbf{IDAtleta:}
    \begin{itemize}
        \item Dominio: Entero (\texttt{int}).
        \item Restricción: Llave primaria (PK).
    \end{itemize}
    \item \textbf{Nombre:}
    \begin{itemize}
        \item Dominio: Cadena de caracteres (\texttt{varchar(50)}).
    \end{itemize}
    \item \textbf{PrimerApellido:}
    \begin{itemize}
        \item Dominio: Cadena de caracteres (\texttt{varchar(50)}).
    \end{itemize}
    \item \textbf{SegundoApellido:}
    \begin{itemize}
        \item Dominio: Cadena de caracteres (\texttt{varchar(50)}).
    \end{itemize}
    \item \textbf{FechaNacimiento:}
    \begin{itemize}
        \item Dominio: Fecha (\texttt{date}).
    \end{itemize}
    \item \textbf{Nacionalidad:}
    \begin{itemize}
        \item Dominio: Cadena de caracteres (\texttt{varchar(50)}).
    \end{itemize}
    \item \textbf{Género:}
    \begin{itemize}
        \item Dominio: caracter (\texttt{char(1)}).
    \end{itemize}
    \item \textbf{Temporada:}
    \begin{itemize}
        \item Dominio: int.
    \end{itemize}
    \item \textbf{NombrePaís:}
    \begin{itemize}
        \item Dominio: Cadena de caracteres (\texttt{varchar(50)}).
        \item Restricción: Llave foránea (FK) que referencia a \textbf{País(NombrePaís)}.
    \end{itemize}
    \item \textbf{IDEntrenador:}
    \begin{itemize}
        \item Dominio: Entero (\texttt{int}).
        \item Restricción: Llave foránea (FK) que referencia a \textbf{Entrenador(IDEntrenador)}.
    \end{itemize}
\end{itemize}
\textbf{Llave primaria:}
\begin{itemize}
    \item \textbf{IDAtleta} identifica de forma única a cada atleta.
\end{itemize}

\subsection*{Entidad: TeléfonosAtleta}
\textbf{Atributos:}
\begin{itemize}
    \item \textbf{idAtleta:}
    \begin{itemize}
        \item Dominio: Entero (\texttt{int}).
        \item Restricción: Llave foránea (FK) que referencia a \textbf{Atleta(IDAtleta)}.
    \end{itemize}
    \item \textbf{númeroTeléfono:}
    \begin{itemize}
        \item Dominio: Cadena de caracteres (\texttt{varchar(15)}).
    \end{itemize}
\end{itemize}
\textbf{Llave primaria:}
\begin{itemize}
    \item \textbf{idAtleta, númeroTeléfono} forman una llave primaria compuesta.
\end{itemize}

\subsection*{Entidad: CorreosAtleta}
\textbf{Atributos:}
\begin{itemize}
    \item \textbf{idAtleta:}
    \begin{itemize}
        \item Dominio: Entero (\texttt{int}).
        \item Restricción: Llave foránea (FK) que referencia a \textbf{Atleta(IDAtleta)}.
    \end{itemize}
    \item \textbf{correoElectrónico:}
    \begin{itemize}
        \item Dominio: Cadena de caracteres (\texttt{varchar(50)}).
    \end{itemize}
\end{itemize}
\textbf{Llave primaria:}
\begin{itemize}
    \item \textbf{idAtleta, correoElectrónico} forman una llave primaria compuesta.
\end{itemize}

\subsection*{Entidad: Entrenador}
\textbf{Atributos:}
\begin{itemize}
    \item \textbf{IDEntrenador:}
    \begin{itemize}
        \item Dominio: Entero (\texttt{int}).
        \item Restricción: Llave primaria (PK).
    \end{itemize}
    \item \textbf{Nombre:}
    \begin{itemize}
        \item Dominio: Cadena de caracteres (\texttt{varchar(50)}).
    \end{itemize}
    \item \textbf{PrimerApellido:}
    \begin{itemize}
        \item Dominio: Cadena de caracteres (\texttt{varchar(50)}).
    \end{itemize}
    \item \textbf{SegundoApellido:}
    \begin{itemize}
        \item Dominio: Cadena de caracteres (\texttt{varchar(50)}).
    \end{itemize}
    \item \textbf{FechaNacimiento:}
    \begin{itemize}
        \item Dominio: Fecha (\texttt{date}).
    \end{itemize}
    \item \textbf{Nacionalidad:}
    \begin{itemize}
        \item Dominio: Cadena de caracteres (\texttt{varchar(50)}).
    \end{itemize}
    \item \textbf{Género:}
    \begin{itemize}
        \item Dominio: caracter (\texttt{char(1)}).
    \end{itemize}
    \item \textbf{IDDisciplina:}
    \begin{itemize}
        \item Dominio: Entero (\texttt{int}).
        \item Restricción: Llave foránea (FK) que referencia a \textbf{Disciplina(IDDisciplina)}.
    \end{itemize}
\end{itemize}
\textbf{Llave primaria:}
\begin{itemize}
    \item \textbf{IDEntrenador} identifica de forma única a cada entrenador.
\end{itemize}
\subsection*{Entidad: TeléfonosEntrenador}
\textbf{Atributos:}
\begin{itemize}
    \item \textbf{IDTeléfono:}
    \begin{itemize}
        \item Dominio: Entero (\texttt{int}).
        \item Restricción: Llave primaria (PK).
    \end{itemize}
    \item \textbf{IDEntrenador:}
    \begin{itemize}
        \item Dominio: Entero (\texttt{int}).
        \item Restricción: Llave foránea (FK) que referencia a \textbf{Entrenador(IDEntrenador)}.
    \end{itemize}
\end{itemize}

\subsection*{Entidad: CorreosEntrenador}
\textbf{Atributos:}
\begin{itemize}
    \item \textbf{IDCorreo:}
    \begin{itemize}
        \item Dominio: Entero (\texttt{int}).
        \item Restricción: Llave primaria (PK).
    \end{itemize}
    \item \textbf{IDEntrenador:}
    \begin{itemize}
        \item Dominio: Entero (\texttt{int}).
        \item Restricción: Llave foránea (FK) que referencia a \textbf{Entrenador(IDEntrenador)}.
    \end{itemize}
\end{itemize}

\subsection*{Entidad: Árbitro}
\textbf{Atributos:}
\begin{itemize}
    \item \textbf{IDArbitro:}
    \begin{itemize}
        \item Dominio: Entero (\texttt{int}).
        \item Restricción: Llave primaria (PK).
    \end{itemize}
    \item \textbf{Nombre:}
    \begin{itemize}
        \item Dominio: Cadena de caracteres (\texttt{varchar(50)}).
    \end{itemize}
    \item \textbf{PrimerApellido:}
    \begin{itemize}
        \item Dominio: Cadena de caracteres (\texttt{varchar(50)}).
    \end{itemize}
    \item \textbf{SegundoApellido:}
    \begin{itemize}
        \item Dominio: Cadena de caracteres (\texttt{varchar(50)}).
    \end{itemize}
    \item \textbf{FechaNacimiento:}
    \begin{itemize}
        \item Dominio: Fecha (\texttt{date}).
    \end{itemize}
    \item \textbf{Nacionalidad:}
    \begin{itemize}
        \item Dominio: Cadena de caracteres (\texttt{varchar(50)}).
    \end{itemize}
    \item \textbf{Género:}
    \begin{itemize}
        \item Dominio: caracter (char(1)).
    \end{itemize}
    \item \textbf{IDDisciplina:}
    \begin{itemize}
        \item Dominio: Entero (\texttt{int}).
        \item Restricción: Llave foránea (FK) que referencia a \textbf{Disciplina(IDDisciplina)}.
    \end{itemize}
\end{itemize}

\subsection*{Entidad: TeléfonosÁrbitro}
\textbf{Atributos:}
\begin{itemize}
    \item \textbf{IDTeléfono:}
    \begin{itemize}
        \item Dominio: Entero (\texttt{int}).
        \item Restricción: Llave primaria (PK).
    \end{itemize}
    \item \textbf{IDArbitro:}
    \begin{itemize}
        \item Dominio: Entero (\texttt{int}).
        \item Restricción: Llave foránea (FK) que referencia a \textbf{Árbitro(IDArbitro)}.
    \end{itemize}
\end{itemize}

\subsection*{Entidad: CorreosÁrbitro}
\textbf{Atributos:}
\begin{itemize}
    \item \textbf{IDCorreo:}
    \begin{itemize}
        \item Dominio: Entero (\texttt{int}).
        \item Restricción: Llave primaria (PK).
    \end{itemize}
    \item \textbf{IDArbitro:}
    \begin{itemize}
        \item Dominio: Entero (\texttt{int}).
        \item Restricción: Llave foránea (FK) que referencia a \textbf{Árbitro(IDArbitro)}.
    \end{itemize}
\end{itemize}

\subsection*{Entidad: Disciplina}
\textbf{Atributos:}
\begin{itemize}
    \item \textbf{IDDisciplina:}
    \begin{itemize}
        \item Dominio: Entero (\texttt{int}).
        \item Restricción: Llave primaria (PK).
    \end{itemize}
    \item \textbf{NombreDisciplina:}
    \begin{itemize}
        \item Dominio: Cadena de caracteres (\texttt{varchar(50)}).
    \end{itemize}
    \item \textbf{Categoría:}
    \begin{itemize}
        \item Dominio: Cadena de caracteres (\texttt{varchar(50)}).
    \end{itemize}
\end{itemize}

\subsection*{Entidad: Evento}
\textbf{Atributos:}
\begin{itemize}
    \item \textbf{IDEvento:}
    \begin{itemize}
        \item Dominio: Entero (\texttt{int}).
        \item Restricción: Llave primaria (PK).
    \end{itemize}
    \item \textbf{NombreEvento:}
    \begin{itemize}
        \item Dominio: Cadena de caracteres (\texttt{varchar(50)}).
    \end{itemize}
    \item \textbf{FechaEvento:}
    \begin{itemize}
        \item Dominio: Fecha (\texttt{date}).
    \end{itemize}
    \item \textbf{HoraInicio:}
    \begin{itemize}
        \item Dominio: Hora (\texttt{time}).
    \end{itemize}
    \item \textbf{DuraciónMax:}
    \begin{itemize}
        \item Dominio: Entero (\texttt{int}) en minutos.
    \end{itemize}
    \item \textbf{Precio:}
    \begin{itemize}
        \item Dominio: Decimal (\texttt{decimal(10,2)}).
    \end{itemize}
    \item \textbf{Fase:}
    \begin{itemize}
        \item Dominio: Cadena de caracteres (\texttt{varchar(50)}).
    \end{itemize}
    \item \textbf{NombreLocalidad:}
    \begin{itemize}
        \item Dominio: Cadena de caracteres (\texttt{varchar(50)}).
        \item Restricción: Llave foránea (FK) que referencia a \textbf{Localidad(NombreLocalidad)}.
    \end{itemize}
    \item \textbf{IDDisciplina:}
    \begin{itemize}
        \item Dominio: Entero (\texttt{int}).
        \item Restricción: Llave foránea (FK) que referencia a \textbf{Disciplina(IDDisciplina)}.
    \end{itemize}
\end{itemize}

\subsection*{Entidad: Localidad}
\textbf{Atributos:}
\begin{itemize}
    \item \textbf{NombreLocalidad:}
    \begin{itemize}
        \item Dominio: Cadena de caracteres (\texttt{varchar(50)}).
        \item Restricción: Llave primaria (PK).
    \end{itemize}
    \item \textbf{Calle:}
    \begin{itemize}
        \item Dominio: Cadena de caracteres (\texttt{varchar(50)}).
    \end{itemize}
    \item \textbf{Número:}
    \begin{itemize}
        \item Dominio: Entero (\texttt{int}).
    \end{itemize}
    \item \textbf{Ciudad:}
    \begin{itemize}
        \item Dominio: Cadena de caracteres (\texttt{varchar(50)}).
    \end{itemize}
    \item \textbf{País:}
    \begin{itemize}
        \item Dominio: Cadena de caracteres (\texttt{varchar(50)}).
    \end{itemize}
    \item \textbf{Aforo:}
    \begin{itemize}
        \item Dominio: Entero (\texttt{int}).
    \end{itemize}
    \item \textbf{Tipo:}
    \begin{itemize}
        \item Dominio: Cadena de caracteres (\texttt{varchar(50)}).
    \end{itemize}
\end{itemize}


\subsection*{Entidad: Patrocinador}
\textbf{Atributos:}
\begin{itemize}
    \item \textbf{NombrePatrocinador:}
    \begin{itemize}
        \item Dominio: Cadena de caracteres (\texttt{varchar(50)}).
        \item Restricción: Llave primaria (PK).
    \end{itemize}
\end{itemize}

\subsection*{Entidad: Patrocina}
\textbf{Atributos:}
\begin{itemize}
    \item \textbf{NOmbrePatrocinador:}
    \begin{itemize}
        \item Dominio: Cadena de caracteres (\texttt{varchar(50)}).
        \item Restricción: Llave foránea (FK) que referencia a \textbf{Patrocinador(NombrePatrocinador)}.
    \end{itemize}
    \item \textbf{IDAtleta:}
    \begin{itemize}
        \item Dominio: Entero (\texttt{int}).
        \item Restricción: Llave foránea (FK) que referencia a \textbf{Atleta(IDAtleta)}.
    \end{itemize}
\end{itemize}
\textbf{Llave primaria:}
\begin{itemize}
    \item \textbf{IDPatrocinador, IDAtleta} forman una llave primaria compuesta.
\end{itemize}

\subsection*{Entidad: CompraEntrada}
\textbf{Atributos:}
\begin{itemize}
    \item \textbf{IDCliente:}
    \begin{itemize}
        \item Dominio: Entero (\texttt{int}).
        \item Restricción: Llave foránea (FK) que referencia a \textbf{Cliente(IDCliente)}.
    \end{itemize}
    \item \textbf{IDEvento:}
    \begin{itemize}
        \item Dominio: Entero (\texttt{int}).
        \item Restricción: Llave foránea (FK) que referencia a \textbf{Evento(IDEvento)}.
    \end{itemize}
\end{itemize}

\subsection*{Entidad: Cliente}
\textbf{Atributos:}
\begin{itemize}
    \item \textbf{IDCliente:}
    \begin{itemize}
        \item Dominio: Entero (\texttt{int}).
        \item Restricción: Llave primaria (PK).
    \end{itemize}
    \end{itemize}



