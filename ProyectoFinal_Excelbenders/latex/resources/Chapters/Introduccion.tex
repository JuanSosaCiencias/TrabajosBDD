El Comité Olímpico Internacional (COI), reconociendo la necesidad de modernizar la gestión de información de los Juegos Olímpicos, ha decidido implementar un sistema robusto y centralizado que permita superar los problemas históricos asociados al uso de registros físicos. Este proyecto, desarrollado por estudiantes de la Facultad de Ciencias de la UNAM, busca sentar las bases para una administración de datos más eficiente y consistente, asegurando que los Juegos Olímpicos de Los Ángeles 2028 se beneficien de un manejo más profesional y organizado de la información. \vspace{0.5cm}

    El presente reporte ejecutivo resume el trabajo realizado en el diseño e implementación de una base de datos integral para el COI. A lo largo del documento, se abordan los aspectos técnicos y estratégicos clave, desde la conceptualización del modelo Entidad-Relación hasta la creación de un esquema lógico y físico en PostgreSQL. Asimismo, se destacan las funcionalidades avanzadas del sistema, como procedimientos almacenados, disparadores, y un conjunto de consultas SQL diseñadas para generar reportes ejecutivos que proporcionen información valiosa para la toma de decisiones.\vspace{0.5cm}

    Este sistema no solo representa una solución tecnológica, sino también un paso hacia la profesionalización y digitalización de la administración de los Juegos Olímpicos, marcando un precedente para eventos futuros. El Comité Olímpico Internacional ha confiado en esta propuesta como un pilar esencial para el éxito organizativo de los próximos Juegos.\vspace{0.5cm}