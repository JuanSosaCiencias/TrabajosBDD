\begin{itemize}
    \item \textbf{Conclusiones técnicas:}\vspace{0.5cm}
    
    Hemos notado la importancia de contar con un buen diseño de base de datos, ya que esto nos permitió establecer una base sólida para el desarrollo. Además, los errores en el diseño se traducen en un alto costo en tiempo y esfuerzo. Apreciamos que existan herramientas como estas, ya que nos proporcionan una mejor comprensión de la base de datos y la información que gestionamos. \vspace{0.5cm}

    Creemos que nuestra base de datos puede mejorarse y escalarse fácilmente, lo que permite agregar nuevas funcionalidades y adaptarse tanto a nuevas reglas de negocio como a cambios en las reglas actuales. \vspace{0.5cm}

    Personalmente (Juan Sosa), estoy muy agradecido por el uso de triggers y procedimientos almacenados, ya que en la primera iteración del proyecto, al poblar con datos de prueba, no implementamos validaciones y quedaron datos inconsistentes. El tener que rehacer este proceso utilizando Python para validar fue muy difícil, por lo que contar con estas herramientas que lo hacen automáticamente es un verdadero alivio. \vspace{0.5cm}

    \item \textbf{Conclusiones funcionales:} \vspace{0.5cm}
    
    Tener una base de datos que considere todos los datos necesarios facilitará en gran medida el trabajo del usuario final al manipular y consultar la información. Creemos que este proyecto puede generar mejoras sustanciales en el sistema. Asimismo, es importante destacar que, al implementar un sistema de registros históricos y, si se añaden roles y permisos, podemos obtener un sistema mucho más seguro y confiable. \vspace{0.5cm}

    \item \textbf{Conclusiones sobre los datos:} \vspace{0.5cm}
    
    Es inevitable pensar que, con tantos datos disponibles, será mucho más fácil utilizar técnicas de análisis de datos para obtener información valiosa. Actualmente, podemos obtener el podio de los países con mejor desempeño, pero también podríamos analizar otras tendencias, como las disciplinas que practican estos países y su relación con el desempeño. \vspace{0.5cm}

    Además, al contar con un registro directo de los eventos, podríamos analizar su impacto económico y social, lo que nos permitiría ajustar y mejorar la organización de los próximos juegos o eventos. \vspace{0.5cm}

    \item \textbf{Conclusiones generales:} \vspace{0.5cm}
    
    Creemos que este tipo de desarollos pueden ser altamente beneficiosos para cualquier organización que maneje una gran cantidad de información. La implementación de una base de datos bien diseñada puede mejorar significativamente la eficiencia y la eficacia de la organización. \vspace{0.5cm}
\end{itemize}
