\section{Proceso de desarrollo}

El proyecto se desarollo en varias fases, que basicamente seguian el curso de la materia, especificamente del laboratorio de bases de datos. A continuación se describen las fases del proyecto:
\begin{enumerate}
    \item \textbf{Fase 1:} Instalacion de PostgreSQL y dbeaver, \texttt{creación de la base de datos} dentro de un docker.
    \item  \textbf{Fase 2:} \texttt{Analisis de requerimientos} funcionales y no funcionales y consideracion de alternativas.
    \item \textbf{Fase 3:} Diseño del \texttt{modelo Entidad-Relación} usando draw.io.
    \item \textbf{Fase 4:} Creación del \texttt{modelo relacional} usando draw.io.
    \item \textbf{Fase 5:} Creacion de tablas, definidas en el archivo \texttt{DDL.sql}.
    \item \textbf{Fase 6:} Mantenimiento de llaves foraneas y llaves primarias, también en el mismo archivo.
    \item \textbf{Fase 7:} Población de tablas con datos de prueba, se encuentra en el archivo \texttt{DML.sql}.
    \item \textbf{Fase 8:} Prueba de la base de datos usando una app en python usando Psycopg2 y Django.
    \item \textbf{Fase 9:} \texttt{Consultas} útiles para el COI, para mas detalles ver el capitulo 4 o el anexo \texttt{Consultas.sql} o  \texttt{PruebasDeFuncionalidad}.
    \item \textbf{Fase 10:} Creación de \texttt{triggers} y \texttt{procedimientos almacenados} para la base de datos, ver en los anexos con los mismos nombres para mas detalles.
\end{enumerate}

\section{Herramientas utilizadas}

Las herramientas que utilizamos para la elaboración de este proyecto son las siguientes:
\begin{itemize}
    \item \textbf{PostgreSQL:} Sistema de gestión de bases de datos relacional orientado a objetos.
    \item \textbf{dbeaver:} Herramienta de administración de bases de datos que permite la conexión a múltiples sistemas de gestión de bases de datos.
    \item \textbf{Docker:} Plataforma de código abierto que facilita la creación, implement
    \item \textbf{Git:} Sistema de control de versiones distribuido.
    \item \textbf{GitHub:} Plataforma de desarrollo colaborativo de software para alojar proyectos utilizando el sistema de control de versiones Git.
    \item \textbf{LaTeX:} Sistema de composición de textos.
    \item \textbf{draw.io} Herramienta en línea para la creación de diagramas.
    \item \textbf{Visual Studio Code:} Editor de código fuente desarrollado por Microsoft.
    \item \textbf{Python:} Lenguaje de programación interpretado.
    \item \textbf{Psycopg2:} Adaptador de base de datos PostgreSQL para el lenguaje de programación Python.
    \item \textbf{Plpgsql:} Lenguaje de programación procedural que se utiliza en PostgreSQL.
    \item \textbf{Django:} Framework de desarrollo web de código abierto, escrito en Python.
\end{itemize}