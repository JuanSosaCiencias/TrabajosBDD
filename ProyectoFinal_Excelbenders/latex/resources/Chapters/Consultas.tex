Aqui vamos a detallar los resultados de las consultas que se realizaron en la base de datos, con el fin de obtener información relevante para la toma de decisiones. Cabe mencionar que las consultas se realizaron en PostgreSQL, y se utilizaron las tablas y vistas creadas en el esquema lógico de la base de datos. A continuación, se presentan las consultas realizadas y el analisis de los resultados obtenidos:
\begin{itemize}
    \item \textbf{Consulta 1:} \textbf{A continuación, se muestran tres representaciones posibles referidas a las relaciones entre Materias, Profesores y Alumnos. Analiza las ventajas y desventajas de cada propuesta, contestando las preguntas que se presentan a continuación:}

\begin{center}
    \includegraphics[width=16cm]{resources/ER_2.1.png}
\end{center}

Indica qué diagramas representan la información requerida por las siguientes solicitudes de información:

\begin{enumerate}
    \item \textbf{¿A qué alumnos imparte clases el profesor Carlos Sánchez en la materia Bases de Datos?} \\

    Como dice la pregunta hay 3 relaciones, entonces debemos identificar la relación entre Profesor, Materia y Alumno. \\

    Por lo que el diagrama \textbf{$"b"$} es el más adecuado; pues como se puede ver incluye una relación directa de \textbf{Enseñar} entre \textbf{Profesor} y \textbf{Alumno}, lo cual nos permite conocer a qué \textbf{Alumnos} enseña el profesor en una \textbf{Materia} específica. \\

    \item \textbf{¿Qué materias imparte la profesora Patricia Ríos?} \\

    Como dice la pregunta, requerimos identificar la relación entre \textbf{Profesor} y \textbf{Materia}. \\

    Por lo que el diagrama \textbf{$"a"$} es el más adecuado; pues como se puede ver hay una relación \textbf{Tener} entre \textbf{Profesor} y \textbf{Materia} que esta directamente representada, y esto nos permite conocer qué materias tiene a su cargo la profesora Patrícia Ríos. \\
    
    \item \textbf{¿Qué alumnos están inscritos en la materia Ingeniería de Software?} \\

    Para esta pregunta, el diagrama \textbf{$"a"$} también es útil, ya que la relación entre \textbf{Materia} y \textbf{Alumno} está directamente representada. Esto permite saber qué \textbf{alumnos} están relacionados con la \textbf{materia} Ingeniería de Software.
    
\end{enumerate}

    \newpage
    \item \textbf{Consulta 2:} \textbf{La información de eventos cuyo precio base sea mayor a 2500. Deberan ordenar la información a partir
del precio.}\vspace{.3cm}

En este caso, necesitamos obtener toda la información de eventos con precio mayor a 2500, por lo que nos interesan datos como el ID del evento, nombre de la localidad donde se llevará a cabo, el precio del evento, nombre de la disciplina de la que trata el evento, la duración máxima en minutos, la fecha, la fase del evento, la ciudad y el país. 

Todos estos datos están en diferentes tablas (no solo en Evento), por lo que ocuparemos realizar un join entre ellas para obtener la información. De Disciplina ocupamos el nombre de la disciplina, de localidad el nombre y país, y todos los demás datos de Evento. Luego filtramos los que tengan precio mayor a 2500 y finalmente ordenamos de mayor a menor precio. 

Nuestra consulta sería:

\begin{center}
    \includegraphics[width=10cm]{resources/consulta2.png}
\end{center}

Usamos join natural que solo esta en postgres:D

El resultado es:
\begin{center}
    
\includegraphics[width=15cm]{resources/consulta2.1.png}
\end{center}
    \newpage
    \item  \textbf{Consulta 3:} \begin{center} \textbf{¿Qué diferencia existe entre los diagramas de las figuras a) y c)?} \end{center}

\vspace{.3cm}

Veamos que la diferencia principal entre los diagramas de las figuras a) y c) radica en cómo representan las relaciones entre las entidades \textit{Materia}, \textit{Profesor} y \textit{Alumno}. \\

En la figura a), se utiliza una sola relación llamada \textit{"Tener"} para conectar las tres entidades. Esto significa que tanto los \textit{Profesores} como los \textit{Alumnos} están vinculados a las \textit{Materias} mediante la misma relación, sin especificar claramente cómo es esa relación. Por ejemplo, no nos dice si el \textit{Profesor} está enseñando la materia o si el \textit{Alumno} está inscrito en ella; simplemente dice que todos tienen una conexión con la materia. Se podría interpretar como que los alumnos tienen muchas materias inscritas y los profesores tienen muchas materias asignadas o que las materias tienen muchos alumnos y profesores. Sin embargo esto igualmente resulta ambiguo. \\

Por otro lado, en la figura c), se usan dos relaciones distintas: \textit{"Tener"} y \textit{"Manejar"}. Aquí, \textit{"Tener"} conecta a los \textit{Alumnos} con las \textit{Materias}, lo que podría significar que los \textit{Alumnos} están inscritos en esas \textit{Materias}. La relación \textit{"Manejar"} conecta a los \textit{Profesores} con la relacion binaria antes mencionada, siendo de esta manera mas específica, lo cual por ejemplo nos podría indicar que los \textit{Profesores} están a cargo o enseñan esas materias a los alumnos. Este enfoque es más claro porque muestra específicamente cómo se relaciona cada entidad. Ademas del hecho de señalar o remarcar la relacion entre materia y alumno dentro de un cuadro hace que podamos tomar esa relacion binaria como una entidad en si misma para conectarla con profesor mediante la relacion Manejar. \\

Entonces podemos decir que la figura a) es más simple, pero menos clara y la figura c) es más específica, ya que usa dos relaciones diferentes para mostrar claramente cómo los \textit{Profesores} y los \textit{Alumnos} se relacionan con las \textit{Materias}. \\
    \newpage
    \item \textbf{Consulta 4:} \textbf{Los Jueces y Entrenadores que tengan la misma nacionalidad pero que no se encuentren participando en
el mismo evento.}\vspace{.3cm}
    \newpage
    \item \textbf{Consulta 5:} \textbf{Patrocinadores que solo esten patrocinando a una disciplina.}\vspace{.3cm}

\begin{center}
	\includegraphics[width=1.05\textwidth]{resources/consulta5.png}
\end{center} 

\textbf{Explicación:} \\
Aquí seleccionamos los datos de la tabla patrocina, luego agrupamos por nombrepatrocinador e iddisciplina para organizar la información de patrocinadores y disciplinas. Con el uso de having count(patrocina.iddisciplina) = 1, filtramos los patrocinadores que solo tienen una entrada en la tabla patrocina, lo cual significa que están patrocinando exclusivamente una disciplina.\vspace{.3cm}

\textbf{Resultado:}
\begin{center}
	\includegraphics[width=1.05\textwidth]{resources/resultados/r5.png}
\end{center} 
    \newpage
    \item \textbf{Consulta 6:} \textbf{El número de medallas de oro ganadas por México.}\vspace{.3cm}

\begin{center}
	\includegraphics[width=1.05\textwidth]{resources/consulta6.png}
\end{center} 

\textbf{Explicación:} \\
En esta consulta, seleccionamos la tabla Medalla y la enlazamos con la tabla Atleta mediante un join usando IDAtleta para relacionar medallas con los atletas. Luego, aplicamos un filtro con where para incluir solo los atletas de nacionalidad mexicana y medallas de tipo oro. Finalmente, usamos count(*) para contar las filas que cumplen estas condiciones, obteniendo el total de medallas de oro ganadas por atletas mexicanos.
\vspace{.3cm}

\textbf{Resultado:}
\begin{center}
	\includegraphics[width=1.05\textwidth]{resources/resultados/r6.png}
\end{center} 
    \newpage
    \item \textbf{Consulta 7:} \textbf{El número de medallas de plata ganadas por Japon.}\vspace{.3cm}
    \newpage
    \item \textbf{Consulta 8:} \begin{center}
	\includegraphics[width=16.5cm]{resources/Chapters/Consultas/Imagenes/Consulta8.jpg} 
	
	Consulta 8. Países que han ganado más medallas.
\end{center}

\textbf{Propósito de la consulta}

La consulta tiene como objetivo identificar los países que han ganado más medallas, mostrando su nombre y la cantidad de medallas ganadas, ordenados de mayor a menor. Esto permite observar el desempeño acumulado de los países en términos de logros deportivos.

\textbf{Desglose de la consulta}

\begin{itemize} \item \textbf{Selección de columnas (\texttt{SELECT}):} \begin{itemize} \item \texttt{p.NombrePais}: Muestra el nombre del país al que se asignan las medallas. \item \texttt{COUNT(m.TipoMedalla) AS CantidadMedallas}: Cuenta la cantidad total de medallas ganadas por cada país, independientemente del tipo (oro, plata o bronce). \end{itemize}
	
	\item \textbf{Tablas involucradas (\texttt{FROM} y \texttt{JOIN}):} \begin{itemize} \item \texttt{Medalla (m)}: Tabla que registra las medallas ganadas, incluyendo información sobre el atleta que las obtuvo. \item \texttt{Atleta (a)}: Tabla que relaciona a cada medalla con un atleta específico mediante la clave \texttt{m.IDAtleta = a.IDAtleta}. \item \texttt{Pais (p)}: Tabla que asocia cada atleta con su respectivo país mediante la relación \texttt{a.NombrePais = p.NombrePais}. \end{itemize}
	
	\item \textbf{Agrupación de resultados (\texttt{GROUP BY}):} \begin{itemize} \item La agrupación se realiza por \texttt{p.NombrePais}, permitiendo que las medallas se contabilicen para cada país de forma independiente. \end{itemize}
	
	\item \textbf{Ordenamiento de resultados (\texttt{ORDER BY}):} \begin{itemize} \item Los resultados se ordenan por la columna \texttt{CantidadMedallas} en orden descendente (\texttt{DESC}), para que los países con más medallas aparezcan primero. \end{itemize} \end{itemize}

\textbf{Análisis detallado}

\begin{itemize} \item \textbf{Relación entre tablas:} \begin{itemize} \item Existe una relación jerárquica que conecta \texttt{Medalla} con \texttt{Atleta} y \texttt{Atleta} con \texttt{Pais}: \begin{itemize} \item Cada medalla (\texttt{Medalla}) se asigna a un atleta específico (\texttt{Atleta}). \item Cada atleta pertenece a un país (\texttt{Pais}). \item Por lo tanto, las medallas de cada país se calculan mediante esta relación indirecta. \end{itemize} \end{itemize}
	
	\item \textbf{Uso de la función agregada \texttt{COUNT}:} \begin{itemize} \item La función \texttt{COUNT(m.TipoMedalla)} cuenta cuántas medallas (de cualquier tipo) están asociadas a cada país. \item Esto implica que no se diferencia entre tipos de medalla, sino que todas las medallas tienen el mismo peso en el conteo. \end{itemize}
	
	\item \textbf{Agrupación por país:} \begin{itemize} \item Agrupar los resultados por \texttt{p.NombrePais} asegura que las medallas se contabilicen de manera acumulativa para cada país. \end{itemize}
	
	\item \textbf{Ordenamiento por cantidad de medallas:} \begin{itemize} \item Ordenar los resultados por \texttt{CantidadMedallas} en orden descendente facilita la identificación de los países con mejor desempeño en términos de medallas ganadas. \end{itemize} \end{itemize}

\textbf{Posibles escenarios y consideraciones}

\begin{itemize} \item \textbf{Países sin medallas:} \begin{itemize} \item Los países sin medallas no aparecerán en los resultados, ya que la consulta utiliza la función \texttt{COUNT}, que excluye filas sin registros asociados. \end{itemize}
	
	\item \textbf{Medallas compartidas:} \begin{itemize} \item Si un sistema permite que una medalla sea compartida por varios atletas de diferentes países, la consulta no maneja este caso explícitamente. \end{itemize}
	
	\item \textbf{Empates en el conteo:} \begin{itemize} \item Si dos países tienen la misma cantidad de medallas, el orden relativo entre ellos no está definido, pero esto no afecta el propósito principal de la consulta. \end{itemize} \end{itemize}

Esta consulta es útil para analizar el desempeño de cada país en términos de medallas ganadas, proporcionando información valiosa para comparaciones y estudios de rendimiento deportivo.
    \newpage
    \item \textbf{Consulta 9:} \begin{center}
	\includegraphics[width=16.5cm]{resources/Chapters/Consultas/Imagenes/Consulta9.jpg} 
	
	Consulta 9. Porcentaje de aforo utilizado en las localidades.
\end{center}

\textbf{Propósito de la consulta}

La consulta tiene como objetivo calcular el porcentaje de aforo utilizado en cada localidad para eventos realizados, considerando el número total de entradas vendidas y la capacidad máxima (aforo) de cada localidad. Además, se ordenan los resultados de mayor a menor porcentaje de aforo utilizado.

\textbf{Desglose de la consulta}

\begin{itemize} \item \textbf{Selección de columnas (\texttt{SELECT}):} \begin{itemize} \item \texttt{Localidad.NombreLocalidad}: Identifica el nombre de la localidad donde se realizó cada evento. \item \texttt{Localidad.Aforo}: Representa la capacidad máxima (número total de asientos) de cada localidad. \item \texttt{COUNT(CompraEntrada.IDEvento) AS EntradasVendidas}: Cuenta la cantidad total de entradas vendidas para eventos realizados en cada localidad. \item \texttt{ROUND((COUNT(CompraEntrada.IDEvento) * 100.0) / Localidad.Aforo, 2) AS PorcentajeAforoUtilizado}: Calcula el porcentaje de aforo utilizado en cada localidad, dividiendo las entradas vendidas entre el aforo y multiplicando por 100. La función \texttt{ROUND} redondea este valor a dos decimales. \end{itemize}
	
	\item \textbf{Tablas involucradas (\texttt{FROM} y \texttt{JOIN}):} \begin{itemize} \item \texttt{Evento}: Tabla que contiene información sobre los eventos realizados, incluyendo su relación con las localidades. \item \texttt{Localidad}: Tabla que almacena datos sobre las localidades, incluyendo su nombre y capacidad máxima (\texttt{Aforo}). \item \texttt{CompraEntrada}: Tabla que registra las entradas compradas para cada evento. \item \textbf{Unión de tablas (\texttt{JOIN})}: \begin{itemize} \item \texttt{Evento.NombreLocalidad = Localidad.NombreLocalidad}: Relaciona cada evento con la localidad donde se realizó. \item \texttt{Evento.IDEvento = CompraEntrada.IDEvento}: Conecta cada evento con las entradas vendidas correspondientes. \end{itemize} \end{itemize}
	
	\item \textbf{Agrupación de resultados (\texttt{GROUP BY}):} \begin{itemize} \item La agrupación se realiza por: \begin{itemize} \item \texttt{Localidad.NombreLocalidad}: Para obtener estadísticas específicas para cada localidad. \item \texttt{Localidad.Aforo}: Para incluir el aforo máximo en los cálculos por localidad. \end{itemize} \end{itemize}
	
	\item \textbf{Ordenamiento de resultados (\texttt{ORDER BY}):} \begin{itemize} \item Los resultados se ordenan por \texttt{PorcentajeAforoUtilizado} en orden descendente (\texttt{DESC}), destacando las localidades con mayor porcentaje de uso de su capacidad. \end{itemize} \end{itemize}

\textbf{Análisis detallado}

\begin{itemize} \item \textbf{Relación entre tablas:} \begin{itemize} \item Cada entrada registrada en \texttt{CompraEntrada} está asociada a un evento específico (\texttt{Evento}). \item Cada evento se realiza en una localidad particular (\texttt{Localidad}), estableciendo la conexión para calcular el porcentaje de entradas vendidas respecto al aforo de esa localidad. \end{itemize}
	
	\item \textbf{Uso de funciones agregadas:} \begin{itemize} \item \texttt{COUNT(CompraEntrada.IDEvento)}: Cuenta el total de entradas vendidas para eventos realizados en cada localidad. \item \texttt{ROUND}: Redondea el cálculo del porcentaje a dos decimales para mejorar la presentación de los datos. \end{itemize}
	
	\item \textbf{Cálculo del porcentaje de aforo utilizado:} \begin{itemize} \item La fórmula \texttt{(COUNT(CompraEntrada.IDEvento) * 100.0) / Localidad.Aforo} calcula la proporción de entradas vendidas respecto al aforo máximo y la convierte en un porcentaje. \end{itemize}
	
	\item \textbf{Ordenamiento por porcentaje:} \begin{itemize} \item Ordenar los resultados por \texttt{PorcentajeAforoUtilizado} permite identificar las localidades donde se aprovechó más la capacidad disponible. \end{itemize} \end{itemize}

\textbf{Posibles escenarios y consideraciones}

\begin{itemize} \item \textbf{Localidades sin entradas vendidas:} \begin{itemize} \item Si una localidad no tiene eventos con entradas vendidas, no aparecerá en los resultados debido al uso de \texttt{COUNT}, que excluye valores nulos. \end{itemize}
	
	\item \textbf{Capacidad máxima (aforo):} \begin{itemize} \item Localidades con un aforo bajo pueden tener un porcentaje de uso alto incluso con pocas entradas vendidas, lo que podría sesgar los análisis si no se consideran otros factores. \end{itemize}
	
	\item \textbf{Localidades con eventos múltiples:} \begin{itemize} \item Si una localidad ha albergado varios eventos, el porcentaje de aforo utilizado considera la suma total de entradas vendidas en todos los eventos realizados en esa localidad. \end{itemize} \end{itemize}

Esta consulta permite evaluar la eficiencia en el uso de las capacidades de las localidades, proporcionando información útil para optimizar la planificación de eventos futuros.
    \newpage
    \item \textbf{Consulta 10:} \textbf{La información de todos los atletas que hayan ganado alguna medalla. Asi como un conteo de las medallas
de oro, plata y bronce que ganaron. La información debera ser ordenada con respecto a las medallas, es
decir primero oro, despues plata y al final bronce.}\vspace{.3cm}
    \newpage
    \item \textbf{Consulta 11:} \begin{center}
	\includegraphics[width=16.5cm]{resources/Chapters/Consultas/Imagenes/Consulta11.jpg} 
	
	Consulta 11. Promedio de edad por disciplina.
\end{center}

\textbf{Propósito de la consulta}

La consulta tiene como objetivo calcular el promedio de edad de los atletas por cada disciplina deportiva. Los resultados están ordenados alfabéticamente por el nombre de la disciplina para facilitar su interpretación.

\textbf{Desglose de la consulta}

\begin{itemize} \item \textbf{Selección de columnas (\texttt{SELECT}):} \begin{itemize} \item \texttt{d.NombreDisciplina}: El nombre de la disciplina, que identifica de forma única cada deporte o actividad. \item \texttt{AVG(EXTRACT(YEAR FROM evento.FechaEvento) - EXTRACT(YEAR FROM a.FechaNacimiento)) AS PromedioEdad}: Calcula el promedio de la diferencia de años entre la fecha del evento y la fecha de nacimiento de los atletas, representando el promedio de edad para cada disciplina. \end{itemize}
	
	\item \textbf{Tablas involucradas (\texttt{FROM} y \texttt{JOIN}):} \begin{itemize} \item \texttt{Evento}: Tabla que contiene información sobre los eventos, incluida la fecha en que se llevaron a cabo. \item \texttt{Disciplina (d)}: Tabla que registra las disciplinas deportivas disponibles. \item \texttt{Participa (p)}: Relaciona a los atletas con las disciplinas en las que participan. \item \texttt{Atleta (a)}: Tabla que almacena información sobre los atletas, incluida su fecha de nacimiento. \item \textbf{Uniones (\texttt{JOIN})}: \begin{itemize} \item Se realiza un \texttt{JOIN} entre \texttt{Disciplina (d)} y \texttt{Participa (p)} usando la clave \texttt{d.IDDisciplina = p.IDDisciplina}. \item Posteriormente, se une \texttt{Atleta (a)} con \texttt{Participa (p)} usando \texttt{p.IDAtleta = a.IDAtleta}. \end{itemize} \end{itemize}
	
	\item \textbf{Agrupación de resultados (\texttt{GROUP BY}):} \begin{itemize} \item La agrupación se realiza por \texttt{d.NombreDisciplina}, lo que permite calcular el promedio de edad de los atletas de forma independiente para cada disciplina. \end{itemize}
	
	\item \textbf{Ordenamiento de resultados (\texttt{ORDER BY}):} \begin{itemize} \item Los resultados se ordenan alfabéticamente por el nombre de la disciplina (\texttt{d.NombreDisciplina}), facilitando su organización y lectura. \end{itemize} \end{itemize}

\textbf{Análisis detallado}

\begin{itemize} \item \textbf{Relación entre tablas:} \begin{itemize} \item La consulta utiliza varias tablas relacionadas: \begin{itemize} \item La tabla \texttt{Disciplina (d)} se conecta con \texttt{Participa (p)} para identificar las disciplinas en las que participan los atletas. \item La tabla \texttt{Atleta (a)} proporciona la información de la fecha de nacimiento de cada atleta, necesaria para calcular su edad. \end{itemize} \item La tabla \texttt{Evento} se utiliza para calcular la edad de los atletas en el año en que ocurrió el evento. \end{itemize}
	
	\item \textbf{Cálculo del promedio de edad:} \begin{itemize} \item La función \texttt{AVG()} calcula el promedio de la diferencia entre: \begin{itemize} \item El año del evento (\texttt{EXTRACT(YEAR FROM evento.FechaEvento)}). \item El año de nacimiento del atleta (\texttt{EXTRACT(YEAR FROM a.FechaNacimiento)}). \end{itemize} \item Esto representa el promedio de edad de los atletas para cada disciplina en el año del evento. \end{itemize}
	
	\item \textbf{Ordenamiento alfabético:} \begin{itemize} \item Ordenar los resultados por \texttt{d.NombreDisciplina} garantiza que las disciplinas estén organizadas de forma alfabética, mejorando la presentación de los datos. \end{itemize} \end{itemize}

\textbf{Posibles escenarios y consideraciones}

\begin{itemize} \item \textbf{Disciplinas sin participación:} \begin{itemize} \item Si una disciplina no tiene atletas registrados en \texttt{Participa}, no aparecerá en los resultados debido al uso de \texttt{JOIN}, lo que implica que solo se consideran disciplinas con participantes. \end{itemize}
	
	\item \textbf{Edad promedio en decimal:} \begin{itemize} \item Los resultados del promedio de edad pueden incluir decimales, lo que representa una aproximación más precisa. \end{itemize}
	
	\item \textbf{Datos de fechas:} \begin{itemize} \item Es importante que las fechas (\texttt{evento.FechaEvento} y \texttt{a.FechaNacimiento}) estén correctamente registradas para evitar errores en el cálculo de la edad. \end{itemize} \end{itemize}

La consulta está diseñada para calcular de manera eficiente el promedio de edad de los atletas por disciplina, proporcionando información valiosa para análisis demográficos y de participación en los eventos deportivos.
    \newpage
    \item \textbf{Consulta 12:} \begin{center}
    \includegraphics[width=16.5cm]{resources/Consulta12.jpeg} 
    
   Consulta 12. Árbitros asignados por disciplina y su nacionalidad.
\end{center}

\textbf{Propósito de la consulta}

El objetivo de esta consulta es obtener una lista de los árbitros asignados a cada disciplina, incluyendo su nacionalidad. Esto permite analizar la distribución de los árbitros entre las disciplinas y verificar la diversidad cultural representada en el equipo de árbitros.

\textbf{Desglose de la consulta}

\begin{itemize}
   \item \textbf{Selección de columnas (\texttt{SELECT})}:
   \begin{itemize}
       \item \texttt{d.NombreDisciplina}: Nombre de la disciplina deportiva.
       \item \texttt{ar.Nombre}: Nombre del árbitro.
       \item \texttt{ar.PrimerApellido} y \texttt{ar.SegundoApellido}: Apellidos del árbitro.
       \item \texttt{ar.Nacionalidad}: Nacionalidad del árbitro.
   \end{itemize}

   \item \textbf{Tablas involucradas (\texttt{FROM} y \texttt{JOIN})}:
   \begin{itemize}
       \item \texttt{Disciplina (d)}: Contiene información sobre las disciplinas deportivas.
       \item \texttt{Arbitro (ar)}: Contiene información sobre los árbitros.
       \item Se realiza un \texttt{JOIN} entre ambas tablas usando la relación \texttt{d.IDDisciplina = ar.IDDisciplina}, lo que asocia cada árbitro con su respectiva disciplina.
   \end{itemize}

   \item \textbf{Ordenamiento de resultados (\texttt{ORDER BY})}:
   \begin{itemize}
       \item Los resultados se ordenan primero por \texttt{d.NombreDisciplina} (nombre de la disciplina) y luego por \texttt{ar.Nacionalidad} (nacionalidad del árbitro).
   \end{itemize}
\end{itemize}

\textbf{Análisis detallado}

\begin{enumerate}
   \item \textbf{Relación entre tablas:}
   \begin{itemize}
       \item Existe una relación directa entre las tablas \texttt{Disciplina} y \texttt{Arbitro} a través de la clave foránea \texttt{ar.IDDisciplina}, que apunta a \texttt{d.IDDisciplina}.
       \item Esto implica que cada árbitro está asignado a una única disciplina.
   \end{itemize}
   
   \item \textbf{Uso de columnas seleccionadas:}
   \begin{itemize}
       \item Se seleccionan tanto datos descriptivos de las disciplinas como la información personal y de nacionalidad de los árbitros, para proporcionar un contexto completo de las asignaciones.
   \end{itemize}
   
   \item \textbf{Ordenamiento:}
   \begin{itemize}
       \item El ordenamiento jerárquico (por disciplina y nacionalidad) facilita la visualización de los árbitros por disciplina y permite detectar rápidamente patrones o diversidad nacional en cada deporte.
   \end{itemize}
\end{enumerate}

\textbf{Consideraciones}

\begin{itemize}
   \item \textbf{Árbitros sin asignación:}
   \begin{itemize}
       \item La consulta no incluye árbitros que no estén asignados a una disciplina, debido a la naturaleza del \texttt{JOIN}.
   \end{itemize}
   \item \textbf{Empates en la nacionalidad:}
   \begin{itemize}
       \item Si varios árbitros de la misma disciplina comparten nacionalidad, el orden entre ellos no está definido. Se puede agregar un criterio adicional en el \texttt{ORDER BY}, como el nombre completo del árbitro.
   \end{itemize}
\end{itemize}

\textbf{Utilidad de la consulta}

Esta consulta es útil para:
\begin{itemize}
    \item Monitorear la asignación de árbitros por disciplina y garantizar una distribución equitativa de recursos humanos.
    \item Identificar posibles carencias o exceso de árbitros en una disciplina específica.
    \item Analizar la diversidad nacional de los árbitros, lo que puede ser un indicador importante en eventos deportivos internacionales.
    \item Facilitar la planeación logística y la gestión de recursos para competencias futuras.
\end{itemize}
    \newpage
    \item \textbf{Consulta 13:} \begin{center}
    \includegraphics[width=16.5cm]{resources/Consulta13.jpeg} 
    
   Consulta 13. Entradas vendidas por disciplina en un rango de fechas.
\end{center}

\textbf{Propósito de la consulta}

El objetivo de esta consulta es determinar la cantidad de entradas vendidas por disciplina durante un rango específico de fechas. Esto permite analizar la popularidad de las disciplinas y apoyar la toma de decisiones en la planificación de futuros eventos.

\textbf{Desglose de la consulta}

\begin{itemize}
   \item \textbf{Selección de columnas (\texttt{SELECT})}:
   \begin{itemize}
       \item \texttt{d.NombreDisciplina}: Nombre de la disciplina deportiva.
       \item \texttt{COUNT(ce.IDCliente)}: Calcula la cantidad de entradas vendidas para cada disciplina. Esta columna se denomina \texttt{EntradasVendidas}.
   \end{itemize}

   \item \textbf{Tablas involucradas (\texttt{FROM} y \texttt{JOIN})}:
   \begin{itemize}
       \item \texttt{Evento (e)}: Contiene información sobre los eventos deportivos.
       \item \texttt{CompraEntrada (ce)}: Contiene información sobre las compras de entradas.
       \item \texttt{Disciplina (d)}: Contiene información sobre las disciplinas deportivas.
       \item Se realizan los siguientes \texttt{JOINs}:
       \begin{itemize}
           \item \texttt{Evento} con \texttt{CompraEntrada} usando \texttt{e.IDEvento = ce.IDEvento}, para relacionar las entradas con los eventos.
           \item \texttt{Evento} con \texttt{Disciplina} usando \texttt{e.IDDisciplina = d.IDDisciplina}, para asociar los eventos con las disciplinas correspondientes.
       \end{itemize}
   \end{itemize}

   \item \textbf{Filtrado de datos (\texttt{WHERE})}:
   \begin{itemize}
       \item Se filtran los eventos cuya fecha (\texttt{e.FechaEvento}) esté dentro del rango especificado: entre el 1 de enero y el 31 de diciembre de 2025.
   \end{itemize}

   \item \textbf{Agrupación de resultados (\texttt{GROUP BY})}:
   \begin{itemize}
       \item Los resultados se agrupan por \texttt{d.NombreDisciplina}, para calcular la cantidad total de entradas vendidas por cada disciplina.
   \end{itemize}

   \item \textbf{Ordenamiento de resultados (\texttt{ORDER BY})}:
   \begin{itemize}
       \item Los resultados se ordenan en orden descendente (\texttt{DESC}) según la cantidad de entradas vendidas (\texttt{EntradasVendidas}), mostrando primero las disciplinas más populares.
   \end{itemize}
\end{itemize}

\textbf{Análisis detallado}

\begin{enumerate}
   \item \textbf{Relación entre tablas:}
   \begin{itemize}
       \item Existe una relación entre las tablas \texttt{Evento}, \texttt{CompraEntrada} y \texttt{Disciplina}:
       \begin{itemize}
           \item Cada entrada comprada se asocia con un evento a través de \texttt{IDEvento}.
           \item Cada evento está vinculado a una disciplina mediante \texttt{IDDisciplina}.
       \end{itemize}
   \end{itemize}
   
   \item \textbf{Cálculo de entradas vendidas:}
   \begin{itemize}
       \item La función agregada \texttt{COUNT(ce.IDCliente)} cuenta el número de entradas vendidas asociadas con cada disciplina.
   \end{itemize}
   
   \item \textbf{Filtrado por rango de fechas:}
   \begin{itemize}
       \item El filtro en el \texttt{WHERE} asegura que solo se incluyan eventos ocurridos en 2025, excluyendo datos fuera de este rango temporal.
   \end{itemize}
   
   \item \textbf{Ordenamiento:}
   \begin{itemize}
       \item Ordenar por \texttt{EntradasVendidas DESC} permite identificar las disciplinas con mayor éxito en ventas de entradas.
   \end{itemize}
\end{enumerate}

\textbf{Consideraciones}

\begin{itemize}
   \item \textbf{Eventos sin ventas:}
   \begin{itemize}
       \item Si una disciplina no tuvo ventas de entradas, no aparecerá en los resultados.
   \end{itemize}
   \item \textbf{Empates en las ventas:}
   \begin{itemize}
       \item Si dos disciplinas tienen la misma cantidad de entradas vendidas, el orden relativo entre ellas no está definido. Se podría agregar un criterio adicional en el \texttt{ORDER BY}, como el nombre de la disciplina.
   \end{itemize}
\end{itemize}

\textbf{Utilidad de la consulta}

Esta consulta es útil para:
\begin{itemize}
    \item Evaluar la popularidad de las disciplinas en función de las ventas de entradas.
    \item Identificar disciplinas que podrían necesitar estrategias de promoción o mejor planificación logística.
    \item Ayudar en la asignación de recursos y espacios para futuras competencias basadas en la demanda histórica.
    \item Determinar patrones de participación del público durante un rango de fechas específico.
\end{itemize}

    \newpage
    \item \textbf{Consulta 14:} \begin{center}
    \includegraphics[width=16.5cm]{resources/Consulta14.jpeg} 
    
   Consulta 14. Relación entre atletas y sus países de origen.
\end{center}

\textbf{Propósito de la consulta}

El objetivo de esta consulta es obtener la cantidad de atletas de cada país que participan en cada disciplina. Esto permite analizar la representación de los países en diferentes disciplinas y evaluar la diversidad en las competencias deportivas.

\textbf{Desglose de la consulta}

\begin{itemize}
   \item \textbf{Selección de columnas (\texttt{SELECT})}:
   \begin{itemize}
       \item \texttt{p.NombrePais}: Nombre del país de origen de los atletas.
       \item \texttt{d.NombreDisciplina}: Nombre de la disciplina deportiva.
       \item \texttt{COUNT(a.IDAtleta)}: Calcula la cantidad de atletas por país en cada disciplina, generando la columna \texttt{CantidadAtletas}.
   \end{itemize}

   \item \textbf{Tablas involucradas (\texttt{FROM} y \texttt{JOIN})}:
   \begin{itemize}
       \item \texttt{Atleta (a)}: Contiene información sobre los atletas.
       \item \texttt{Pais (p)}: Contiene información sobre los países de origen.
       \item \texttt{Participa (pa)}: Relaciona a los atletas con las disciplinas en las que participan.
       \item \texttt{Disciplina (d)}: Contiene información sobre las disciplinas deportivas.
       \item Se realizan los siguientes \texttt{JOINs}:
       \begin{itemize}
           \item \texttt{Atleta} con \texttt{Pais} usando \texttt{a.NombrePais = p.NombrePais}, para asociar a cada atleta con su país de origen.
           \item \texttt{Atleta} con \texttt{Participa} usando \texttt{a.IDAtleta = pa.IDAtleta}, para identificar las disciplinas en las que participa cada atleta.
           \item \texttt{Participa} con \texttt{Disciplina} usando \texttt{pa.IDDisciplina = d.IDDisciplina}, para relacionar cada participación con una disciplina específica.
       \end{itemize}
   \end{itemize}

   \item \textbf{Agrupación de resultados (\texttt{GROUP BY})}:
   \begin{itemize}
       \item Los resultados se agrupan por \texttt{p.NombrePais} y \texttt{d.NombreDisciplina}, para calcular la cantidad de atletas por país en cada disciplina.
   \end{itemize}

   \item \textbf{Ordenamiento de resultados (\texttt{ORDER BY})}:
   \begin{itemize}
       \item Los resultados se ordenan primero por \texttt{p.NombrePais} (nombre del país) y luego por \texttt{d.NombreDisciplina} (nombre de la disciplina), facilitando la lectura y análisis.
   \end{itemize}
\end{itemize}

\textbf{Análisis detallado}

\begin{enumerate}
   \item \textbf{Relación entre tablas:}
   \begin{itemize}
       \item La consulta establece relaciones entre las tablas \texttt{Atleta}, \texttt{Pais}, \texttt{Participa} y \texttt{Disciplina}, conectando a cada atleta con su país y las disciplinas en las que participa.
   \end{itemize}
   
   \item \textbf{Cálculo de atletas:}
   \begin{itemize}
       \item La función agregada \texttt{COUNT(a.IDAtleta)} cuenta el número de atletas de un país que participan en cada disciplina.
   \end{itemize}
   
   \item \textbf{Agrupación:}
   \begin{itemize}
       \item El \texttt{GROUP BY} asegura que los datos estén organizados de manera que cada combinación de país y disciplina tenga su correspondiente conteo.
   \end{itemize}
   
   \item \textbf{Ordenamiento:}
   \begin{itemize}
       \item El orden jerárquico por país y disciplina facilita la interpretación, permitiendo identificar rápidamente la representación por país en cada deporte.
   \end{itemize}
\end{enumerate}

\textbf{Consideraciones}

\begin{itemize}
   \item \textbf{Atletas sin participación:}
   \begin{itemize}
       \item Si un atleta no está asociado a una disciplina, no aparecerá en los resultados.
   \end{itemize}
   \item \textbf{Empates en cantidad de atletas:}
   \begin{itemize}
       \item Si dos países tienen la misma cantidad de atletas en una disciplina, el orden entre ellos no está definido. Se podría agregar un criterio adicional en el \texttt{ORDER BY}, como el nombre del país o de la disciplina.
   \end{itemize}
\end{itemize}

\textbf{Utilidad de la consulta}

Esta consulta es útil para:
\begin{itemize}
    \item Analizar la diversidad de participación en cada disciplina y verificar la representación equitativa de diferentes países.
    \item Identificar posibles tendencias en la participación de atletas de determinados países en disciplinas específicas.
    \item Planificar estrategias de inclusión y promoción para fomentar la participación de países con poca representación.
    \item Facilitar reportes y estadísticas sobre la participación internacional en competencias deportivas.
\end{itemize}

    \newpage
    \item \textbf{Consulta 15:} \begin{center}
    \includegraphics[width=16.5cm]{resources/Consulta15.jpeg} 
    
   Consulta 15. Número de atletas por cada país.
\end{center}

\textbf{Propósito de la consulta}

El objetivo de esta consulta es determinar cuántos atletas están registrados en cada país, proporcionando una visión general de la distribución de atletas a nivel internacional.

\textbf{Desglose de la consulta}

\begin{itemize}
   \item \textbf{Selección de columnas (\texttt{SELECT})}:
   \begin{itemize}
       \item \texttt{p.NombrePais}: Nombre del país de origen de los atletas.
       \item \texttt{COUNT(a.IDAtleta)}: Calcula el número total de atletas registrados en cada país. Esta columna se denomina \texttt{NumeroAtletas}.
   \end{itemize}

   \item \textbf{Tablas involucradas (\texttt{FROM} y \texttt{JOIN})}:
   \begin{itemize}
       \item \texttt{Pais (p)}: Contiene información sobre los países.
       \item \texttt{Atleta (a)}: Contiene información sobre los atletas.
       \item Se realiza un \texttt{JOIN} entre \texttt{Pais} y \texttt{Atleta} utilizando la relación \texttt{p.NombrePais = a.NombrePais}, que vincula a cada atleta con su país de origen.
   \end{itemize}

   \item \textbf{Agrupación de resultados (\texttt{GROUP BY})}:
   \begin{itemize}
       \item Los resultados se agrupan por \texttt{p.NombrePais}, para calcular el número de atletas registrados en cada país.
   \end{itemize}

   \item \textbf{Ordenamiento de resultados (\texttt{ORDER BY})}:
   \begin{itemize}
       \item Los resultados se ordenan en orden descendente (\texttt{DESC}) según la cantidad de atletas (\texttt{NumeroAtletas}), mostrando primero los países con más atletas registrados.
   \end{itemize}
\end{itemize}

\textbf{Análisis detallado}

\begin{enumerate}
   \item \textbf{Relación entre tablas:}
   \begin{itemize}
       \item La consulta utiliza la relación entre las tablas \texttt{Pais} y \texttt{Atleta} para asociar a cada atleta con su país de origen.
   \end{itemize}
   
   \item \textbf{Cálculo de atletas:}
   \begin{itemize}
       \item La función agregada \texttt{COUNT(a.IDAtleta)} cuenta el número de atletas registrados en cada país.
   \end{itemize}
   
   \item \textbf{Agrupación:}
   \begin{itemize}
       \item El uso de \texttt{GROUP BY} organiza los resultados por país, asegurando que cada país tenga un registro único con su conteo correspondiente de atletas.
   \end{itemize}
   
   \item \textbf{Ordenamiento:}
   \begin{itemize}
       \item El orden descendente por \texttt{NumeroAtletas} facilita la identificación de los países con mayor cantidad de atletas registrados.
   \end{itemize}
\end{enumerate}

\textbf{Consideraciones}

\begin{itemize}
   \item \textbf{Países sin atletas registrados:}
   \begin{itemize}
       \item Los países que no tienen atletas registrados no aparecerán en los resultados.
   \end{itemize}
   \item \textbf{Empates en cantidad de atletas:}
   \begin{itemize}
       \item Si dos o más países tienen el mismo número de atletas registrados, el orden relativo entre ellos no está definido. Se podría agregar un criterio adicional en el \texttt{ORDER BY}, como el nombre del país.
   \end{itemize}
\end{itemize}

\textbf{Utilidad de la consulta}

Esta consulta es útil para:
\begin{itemize}
    \item Analizar la distribución de atletas a nivel internacional.
    \item Identificar países con una alta o baja representación en términos de atletas registrados.
    \item Planificar estrategias para fomentar la participación en países con menor número de atletas.
    \item Evaluar la diversidad y alcance global del registro de atletas.
\end{itemize}

    \newpage
    \item \textbf{Consulta Extra:} \begin{center}
    \includegraphics[width=16.5cm]{resources/ConsultaExtra.jpeg} 
    
   Consulta Extra. Medallero de los Juegos Olímpicos.
\end{center}

\textbf{Propósito de la consulta}

El objetivo de esta consulta es generar el medallero de los Juegos Olímpicos, mostrando la cantidad de medallas de oro, plata y bronce obtenidas por cada país, con un orden que prioriza el número de medallas de oro, seguido por las de plata y bronce.

\textbf{Desglose de la consulta}

\begin{itemize}
   \item \textbf{Selección de columnas (\texttt{SELECT})}:
   \begin{itemize}
       \item \texttt{a.NombrePais}: Nombre del país que será representado en el medallero.
       \item \texttt{COALESCE(SUM(CASE ...))}: Suma la cantidad de medallas de cada tipo (oro, plata, bronce), reemplazando valores nulos con 0.
       \item \texttt{TotalMedallas}: Suma total de medallas (oro, plata y bronce) obtenidas por el país.
   \end{itemize}

   \item \textbf{Tablas involucradas (\texttt{FROM} y \texttt{JOIN})}:
   \begin{itemize}
       \item \texttt{Atleta (a)}: Contiene información sobre los atletas y su país de origen.
       \item \texttt{Medalla (m)}: Contiene información sobre las medallas obtenidas por los atletas.
       \item Se utiliza un \texttt{LEFT JOIN} entre \texttt{Atleta} y \texttt{Medalla} para incluir a los países incluso si no tienen medallas registradas.
   \end{itemize}

   \item \textbf{Agrupación de resultados (\texttt{GROUP BY})}:
   \begin{itemize}
       \item Los resultados se agrupan por \texttt{a.NombrePais}, asegurando que cada país tenga un único registro con el conteo de sus medallas.
   \end{itemize}

   \item \textbf{Ordenamiento de resultados (\texttt{ORDER BY})}:
   \begin{itemize}
       \item Los resultados se ordenan jerárquicamente:
       \begin{enumerate}
           \item Por \texttt{MedallasOro} en orden descendente.
           \item En caso de empate, por \texttt{MedallasPlata}.
           \item En caso de persistir el empate, por \texttt{MedallasBronce}.
       \end{enumerate}
   \end{itemize}

   \item \textbf{Límite de resultados (\texttt{LIMIT 1})}:
   \begin{itemize}
       \item La consulta devuelve solo el país con más medallas de oro (y desempates por plata y bronce), ya que está limitada a un solo resultado.
   \end{itemize}
\end{itemize}

\textbf{Análisis detallado}

\begin{enumerate}
   \item \textbf{Relación entre tablas:}
   \begin{itemize}
       \item La consulta vincula a los atletas con las medallas que han ganado utilizando \texttt{a.IDAtleta = m.IDAtleta}.
       \item El \texttt{LEFT JOIN} asegura que los países sin medallas aún se incluyan en la lista, aunque con valores de medallas igual a cero.
   \end{itemize}

   \item \textbf{Cálculo de medallas:}
   \begin{itemize}
       \item Las expresiones \texttt{CASE WHEN} cuentan medallas específicas (oro, plata, bronce), sumando los valores que corresponden.
       \item \texttt{COALESCE} reemplaza valores nulos con 0, útil para países que no tienen medallas de cierto tipo.
   \end{itemize}

   \item \textbf{Orden y desempate:}
   \begin{itemize}
       \item El orden por tipos de medallas asegura que los países con mejor desempeño sean priorizados correctamente.
   \end{itemize}
\end{enumerate}

\textbf{Consideraciones}

\begin{itemize}
   \item \textbf{Países sin medallas:}
   \begin{itemize}
       \item Gracias al \texttt{LEFT JOIN}, los países sin medallas no son excluidos, pero tendrán valores de medallas igual a cero.
   \end{itemize}
   \item \textbf{Empates:}
   \begin{itemize}
       \item Si dos países tienen exactamente el mismo número de medallas de oro, plata y bronce, el orden relativo entre ellos no está definido.
   \end{itemize}
\end{itemize}

\textbf{Utilidad de la consulta}

Esta consulta permite:
\begin{itemize}
    \item Generar estadísticas sobre el rendimiento de cada país en los Juegos Olímpicos.
    \item Identificar rápidamente el país más exitoso basado en el conteo de medallas.
    \item Evaluar tendencias en el desempeño de países durante la competencia.
\end{itemize}

\textbf{Nota sobre el \texttt{LIMIT 1}}

Si se elimina la cláusula \texttt{LIMIT 1}, la consulta generará el medallero completo, mostrando los resultados para todos los países en orden de prioridad basado en el número de medallas de oro, plata y bronce.

\end{itemize}